%%%%%%%%%%%%%%%%%%%%%%%%%%%%%%%%%%%%%%%%
% Structured General Purpose Assignment
% LaTeX Template
%
% This template has been downloaded from:
% http://www.latextemplates.com
%
% Original author:
% Ted Pavlic (http://www.tedpavlic.com)
%
% Note:
% The \lipsum[#] commands throughout this template generate dummy text
% to fill the template out. These commands should all be removed when 
% writing assignment content.
%
%%%%%%%%%%%%%%%%%%%%%%%%%%%%%%%%%%%%%%%%%

%----------------------------------------------------------------------------------------
%       PACKAGES AND OTHER DOCUMENT CONFIGURATIONS
%----------------------------------------------------------------------------------------

\documentclass{article}

\usepackage{fancyhdr} % Required for custom headers
\usepackage{lastpage} % Required to determine the last page for the footer
\usepackage{extramarks} % Required for headers and footers
\usepackage{graphicx} % Required to insert images
\usepackage{lipsum} % Used for inserting dummy 'Lorem ipsum' text into the template

\usepackage{amsmath}

% Margins
\topmargin=-0.45in
\evensidemargin=0in
\oddsidemargin=0in
\textwidth=6.5in
\textheight=9.0in
\headsep=0.25in 

\linespread{1.1} % Line spacing

% Set up the header and footer
\pagestyle{fancy}
\lhead{\hmwkAuthorName} % Top left header
\chead{\hmwkClass\ : \hmwkTitle} % Top center header
\rhead{\firstxmark} % Top right header
\lfoot{\lastxmark} % Bottom left footer
\cfoot{} % Bottom center footer
\rfoot{Page\ \thepage\ of\ \pageref{LastPage}} % Bottom right footer
\renewcommand\headrulewidth{0.4pt} % Size of the header rule
\renewcommand\footrulewidth{0.4pt} % Size of the footer rule
\newcommand\numberthis{\addtocounter{equation}{1}\tag{\theequation}}
\setlength\parindent{0pt} % Removes all indentation from paragraphs

%----------------------------------------------------------------------------------------
%       DOCUMENT STRUCTURE COMMANDS
%       Skip this unless you know what you're doing
%----------------------------------------------------------------------------------------

% Header and footer for when a page split occurs within a problem environment
\newcommand{\enterProblemHeader}[1]{
\nobreak\extramarks{#1}{#1 continued on next page\ldots}\nobreak
\nobreak\extramarks{#1 (continued)}{#1 continued on next page\ldots}\nobreak
}

% Header and footer for when a page split occurs between problem environments
\newcommand{\exitProblemHeader}[1]{
\nobreak\extramarks{#1 (continued)}{#1 continued on next page\ldots}\nobreak
\nobreak\extramarks{#1}{}\nobreak
}

\setcounter{secnumdepth}{0} % Removes default section numbers
\newcounter{homeworkProblemCounter} % Creates a counter to keep track of the number of problems

\newcommand{\homeworkProblemName}{}
\newenvironment{homeworkProblem}[1][Problem \arabic{homeworkProblemCounter}]{ % Makes a new environment called homeworkProblem which takes 1 argument (custom name) but the default is "Problem #"
\stepcounter{homeworkProblemCounter} % Increase counter for number of problems
\renewcommand{\homeworkProblemName}{#1} % Assign \homeworkProblemName the name of the problem
\section{\homeworkProblemName} % Make a section in the document with the custom problem count
\enterProblemHeader{\homeworkProblemName} % Header and footer within the environment
}{
\exitProblemHeader{\homeworkProblemName} % Header and footer after the environment
}

\newcommand{\problemAnswer}[1]{ % Defines the problem answer command with the content as the only argument
\noindent\framebox[\columnwidth][c]{\begin{minipage}{0.98\columnwidth}#1\end{minipage}} % Makes the box around the problem answer and puts the content inside
}

\newcommand{\homeworkSectionName}{}
\newenvironment{homeworkSection}[1]{ % New environment for sections within homework problems, takes 1 argument - the name of the section
\renewcommand{\homeworkSectionName}{#1} % Assign \homeworkSectionName to the name of the section from the environment argument
\subsection{\homeworkSectionName} % Make a subsection with the custom name of the subsection
\enterProblemHeader{\homeworkProblemName\ [\homeworkSectionName]} % Header and footer within the environment
}{
\enterProblemHeader{\homeworkProblemName} % Header and footer after the environment
}
   
%----------------------------------------------------------------------------------------
%       NAME AND CLASS SECTION
%----------------------------------------------------------------------------------------

\newcommand{\hmwkTitle}{Midterm } % Assignment title
\newcommand{\hmwkDueDate}{Thursday,\ November\ 5,\ 2015} % Due date
\newcommand{\hmwkClass}{MATH-578B} % Course/class
\newcommand{\hmwkClassTime}{} % Class/lecture time
\newcommand{\hmwkAuthorName}{Saket Choudhary} % Your name
\newcommand{\hmwkAuthorID}{2170058637} % Teacher/lecturer
%----------------------------------------------------------------------------------------
%       TITLE PAGE
%----------------------------------------------------------------------------------------

\title{
\vspace{2in}
\textmd{\textbf{\hmwkClass:\ \hmwkTitle}}\\
\normalsize\vspace{0.1in}\small{Due\ on\ \hmwkDueDate}\\
\vspace{0.1in}\large{\textit{\hmwkClassTime}}
\vspace{3in}
}

\author{\textbf{\hmwkAuthorName} \\
        \textbf{\hmwkAuthorID}
        }
\date{} % Insert date here if you want it to appear below your name

%----------------------------------------------------------------------------------------

\begin{document}

\maketitle

%----------------------------------------------------------------------------------------
%       TABLE OF CONTENTS
%----------------------------------------------------------------------------------------

%\setcounter{tocdepth}{1} % Uncomment this line if you don't want subsections listed in the ToC

\newpage
\tableofcontents
\newpage



\begin{homeworkProblem}[Problem 1] % Roman numerals

\begin{homeworkSection}{\homeworkProblemName: ~(a)}
	\problemAnswer{
	$$
	P =	\begin{bmatrix}
			1-\alpha & \alpha\\
			\beta & 1-\beta
		\end{bmatrix}
	$$
		Let the stationary state be given by $\pi = (\pi_1,\pi_2)$, then:
		\begin{eqnarray*}
		\pi. P = \pi\\
		\pi_1 + \pi_2 =1
		\end{eqnarray*}
		
	Solving which gives:
	\begin{eqnarray*}
		(1-\alpha)\pi_1+\pi_2 = 1\\
		\pi_1 + \pi_2 = 1\\
		\implies (\pi_1,\pi_2) = (\frac{\beta}{\alpha+\beta}, \frac{\alpha}{\alpha+\beta}) 
	\end{eqnarray*}
	
	
	
		}
\end{homeworkSection}

\begin{homeworkSection}{\homeworkProblemName: ~(b)}
\problemAnswer{
	$\mathbf{w} = 101$ 
	\begin{eqnarray*}
	\beta_{w,w}(0) = 1\\
	\beta_{w,w}(1) = 0\\
	\beta_{w,w}(2) = 1
	\end{eqnarray*}
	
	\begin{eqnarray*}
	P_u(0) = 1\\
	P_u(1) = p_{w_2w_3} = p_{01} = \alpha\\
	P_U(2) = p_{w_1w_2}p_{w_2w_3} = p_{10}p_{01} = \beta\alpha\\
	\end{eqnarray*}
	
	Now,
	$$G_{w,w}(t) = \sum_{j=0}^2t^j\beta_{w,w}(j)P_{w,w}(j)$$
	
	Thus,
	\begin{align*}
	G_{w,w}(t) &= 1 \times 1 \times 1 + t \times 0 \times \alpha + t^2 \times 1 \times \beta \alpha\\ 
	&= 1+\alpha\beta t^2
	\end{align*}
	}
\end{homeworkSection}

\begin{homeworkSection}{\homeworkProblemName: ~(c)}
	\problemAnswer{
		$X_n$ : Number of occurrences(overlaps allowed)  in $A_1A_2A_3\dots A_n$
		Using Theorem $12.1$:
		$$
		\lim_{n \rightarrow \infty} \frac{1}{n} E(X_n) = \pi_w = \pi_1 \times p_{10} \times p_{01} = 1 \times \beta \times \alpha 
		$$ 
		
		Thus $	\lim_{n \rightarrow \infty} \frac{X_n}{n} = \alpha \beta$
		
		}
\end{homeworkSection}

\begin{homeworkSection}{\homeworkProblemName: ~(d)}

\problemAnswer{ % Answer
Spectral decomposition of $P$:

$$
det\begin{bmatrix}
\alpha-\lambda & 1-\alpha\\
1-\beta & \beta-\lambda
\end{bmatrix} = 0
$$

$$
\lambda^2 +(\alpha + \beta-2) \lambda + (1-\alpha -\beta) = 0
$$

Thus, $\lambda_1 = 1$ and $\lambda_2 = 1-\alpha-\beta$

Eigenvectors are given by:

$v_1^T = \big( x_1\ x_1 \big)\ \forall\ x_1 \in R$

and for $\lambda_2$ , $v_2 = \big( x_1\ \frac{-\beta x_1}{\alpha} \big)$

Now using Markov property: $P(X_n=1|X_0=0) = (P^n)_{01}$

Now, 

$P^n = VD^nV^{-1}$

where:
$$
V = \begin{bmatrix}
1 & 1\\
1 & \frac{-\beta}{\alpha}
\end{bmatrix}
$$

and 

$$
D = \begin{bmatrix}
1 & 0 \\
0 & (1-\alpha-\beta)
\end{bmatrix}
$$

$$
V^{-1} = \frac{-1}{\frac{\beta}{\alpha}+1}\begin{bmatrix}
-\frac{\beta}{\alpha} & -1 \\
-1 & 1
\end{bmatrix}
$$

Thus,

$$
P^n = \begin{bmatrix}
1 & 1\\
1 & \frac{-\beta}{\alpha}
\end{bmatrix} \times \begin{bmatrix}
1 & 0 \\
0 & (1-\alpha-\beta)^n
\end{bmatrix} \times \frac{-1}{\frac{\beta}{\alpha}+1}\begin{bmatrix}
-\frac{\beta}{\alpha} & -1 \\
-1 & 1
\end{bmatrix}
$$

$$
P^n = \frac{1}{\alpha+\beta} \begin{bmatrix}
\beta + \alpha(1-\alpha-\beta)^n & \alpha-\alpha(1-\alpha-\beta)^n\\
\beta - \beta(1-\alpha-\beta)^n & \alpha + \beta(1-\alpha-\beta)^n
\end{bmatrix}
$$


\underline{\textbf{ALITER}}

We consider the following identity: $P^{n+1}=PP^n$

then:

$$
\begin{bmatrix}
p_{00}^{n+1} & p_{01}^{n+1}\\
p_{10}^{n+1} & p_{11}^{n+1}\\
\end{bmatrix} = \begin{bmatrix} 
p_{00}^n & p_{01}^n\\
p_{10}^n & p_{11}^n
\end{bmatrix} \times \begin{bmatrix}
1-\alpha & \alpha\\
\beta & 1-\beta
\end{bmatrix}
$$

$\implies$


\begin{align*}
p_{11}^{n+1} &= p_{10}^n(\alpha) + p_{11}^n(1-\beta)\\
&= (1-p_{11}^n)(\alpha) +(p_{11}^n)(1-\beta)\\
&= \alpha + (1-\alpha-\beta)p_{11}^n   \tag{1d.1}
\end{align*}

}
\clearpage
\problemAnswer{

On similar lines:

\begin{align*}
p_{00}^{n+1} &= (1-\alpha-\beta)p_{00}^n + \beta   \tag{1d.2}
\end{align*}

In order to solve equations of type $1d.1$ and $1d.2$ we take the following strategy:

By substituting $p_{00}^{n+1} = p_{00}^n$ (and thus obtaining the stationary solution at $\frac{\beta}{\alpha+\beta}$), $1d.2$ can be reduced to the following form:

\begin{align*}
p_{00}^{n+1} &= \frac{\beta}{\alpha+\beta}= (1-\alpha-\beta)(p_{00}^n -\frac{\beta}{\alpha+\beta})  \label{1d.3}
\end{align*}




Let's call $y^(n) =  p_{00}^n -\frac{\beta}{\alpha+\beta}$


Then $1d.3$ is similar to:

\begin{eqnarray*}
y^{(n+1)} = (1-\alpha-\beta)y^(n) \\
y^{(n+1)} = (1-\alpha-\beta)^ny^(0)
\end{eqnarray*}

$y^(0) = p_{00}^{(0)} - \frac{\beta}{\alpha+\beta}$

Assume $p_{00}^{(0)}=1$ $\implies$ $y^{(0)} = \frac{\alpha}{\alpha+\beta }$

Which gives:

$$
p_{00}^n = (1-\alpha-\beta)^n \frac{\alpha}{\alpha+\beta} + \frac{\beta}{\alpha+\beta} \text{if $\alpha+\beta > 0$}
$$

Similarly,

$$
p_{11}^n = (1-\alpha-\beta)^n \frac{\beta}{\alpha+\beta} + \frac{\alpha}{\alpha+\beta} \text{if $\alpha+\beta > 0$}
$$

\textbf{NOTE:} If $\alpha+\beta = 0$, we get:
$$
P^n = \begin{bmatrix}
1 & 0\\
0 & 1
\end{bmatrix}
$$

}

\end{homeworkSection}

\begin{homeworkSection}{\homeworkProblemName: ~(e)}
\problemAnswer{
	$\pi_w =\pi_i p_{10}p_{01} = 1 \times \beta \alpha  = \alpha\beta$
	  
	\begin{align*}
	\lim_{n \rightarrow \infty} \frac{Var(X_n)}{n} &= 2\pi_w \big((\beta_{w,w}(0)P_w(0) -\pi_w ) + (\beta_{w,w}(1)P_w(1) - \pi_w ) + (\beta_{w,w}(2)P_w(2) - \pi_w )	\big)\\
	 &+ 2\pi_w P_w(2)\sum_{j=0}^\infty \{p^{j+1}_{11} - \pi_1 \} + \pi_w^2-\pi_w \\
	&= 2\alpha\beta (1+\alpha\beta - 3\alpha\beta )  + 2\alpha\beta   \sum_{j=0}^\infty \{ \beta^{j+1} - 1 \} - \alpha\beta + (\alpha\beta)^2\\
	&= \alpha\beta \big(2(1-2\alpha\beta) + 2 \sum_{j=0}^\infty \{ \beta^{j+1} - 1 \}  -1 + \alpha\beta)
	\end{align*}
		
}
\end{homeworkSection}

\end{homeworkProblem}

\begin{homeworkSection}{\homeworkProblemName: ~(f)}
	\problemAnswer{
		$Y_n$: Number of occurences of word $w$ in all words, thus,
		$Y_n \approx c X_n$
		
		\textbf{NOTE:} I assume the problem should be to estimate $\lim_{n \rightarrow \infty} \frac{Y_n}{n}$ and $\lim_{n \rightarrow \infty } \frac{Var(Y_n)}{n}$ In the problem it mentions $X_n$ instead of $Y_n$
		
		Thus,
		\begin{align*}
			\lim_{n \rightarrow \infty} \frac{Y_n}{n} &= c \times \lim_{n \rightarrow} \frac{X_n}{n}\\  
			&= c\alpha\beta \\
			\lim_{n \rightarrow \infty } \frac{Var(Y_n)}{n} &= c^2 \times \lim_{n \rightarrow \infty } \frac{Var(X_n)}{n}\\ 
			&= c^2 \times \big( \alpha\beta \big(2(1-2\alpha\beta) + 2 \sum_{j=0}^\infty \{ \beta^{j+1} - 1 \}  -1 + \alpha\beta) \big)
		\end{align*} 
		
		
		}
\end{homeworkSection}

\begin{homeworkProblem}{Problem 2}
\begin{homeworkSection}{\homeworkProblemName: ~(a)}
	\problemAnswer{
		Expected number of squares of side length $t$ such that all $X_v$ are 1 in the square: We simply choose a position on the positive lattice $x$ axis and then construct a square around it $[(x_0,y_0) (x_0+t,y_0+t)]$ so we have $n-t$ choices for $x_0$ and $n-t$ choices for $y_0$, the constraint being that all points inside are all $1$, there are approximately $t^2$ integer points inside 
		
		$$
		E(\# \text{number of squares of side length $t$ such that all $X_v$ are 1 inside}) = (n-t) \times (n-t) p^{t^2} = (n-t)^2 p^{t^2}
		$$
		
	}
\end{homeworkSection}

	
	\begin{homeworkSection}{\homeworkProblemName: ~(b)}
		\problemAnswer{
			
			}
	\end{homeworkSection}
\end{homeworkProblem}

\end{document}