%%%%%%%%%%%%%%%%%%%%%%%%%%%%%%%%%%%%%%%%
% Structured General Purpose Assignment
% LaTeX Template
%
% This template has been downloaded from:
% http://www.latextemplates.com
%
% Original author:
% Ted Pavlic (http://www.tedpavlic.com)
%
% Note:
% The \lipsum[#] commands throughout this template generate dummy text
% to fill the template out. These commands should all be removed when 
% writing assignment content.
%
%%%%%%%%%%%%%%%%%%%%%%%%%%%%%%%%%%%%%%%%%

%----------------------------------------------------------------------------------------
%       PACKAGES AND OTHER DOCUMENT CONFIGURATIONS
%----------------------------------------------------------------------------------------

\documentclass{article}

\usepackage{fancyhdr} % Required for custom headers
\usepackage{lastpage} % Required to determine the last page for the footer
\usepackage{extramarks} % Required for headers and footers
\usepackage{graphicx} % Required to insert images
\usepackage{lipsum} % Used for inserting dummy 'Lorem ipsum' text into the template

\usepackage{amsmath}

% Margins
\topmargin=-0.45in
\evensidemargin=0in
\oddsidemargin=0in
\textwidth=6.5in
\textheight=9.0in
\headsep=0.25in 

\linespread{1.1} % Line spacing

% Set up the header and footer
\pagestyle{fancy}
\lhead{\hmwkAuthorName} % Top left header
\chead{\hmwkClass\ : \hmwkTitle} % Top center header
\rhead{\firstxmark} % Top right header
\lfoot{\lastxmark} % Bottom left footer
\cfoot{} % Bottom center footer
\rfoot{Page\ \thepage\ of\ \pageref{LastPage}} % Bottom right footer
\renewcommand\headrulewidth{0.4pt} % Size of the header rule
\renewcommand\footrulewidth{0.4pt} % Size of the footer rule

\setlength\parindent{0pt} % Removes all indentation from paragraphs

%----------------------------------------------------------------------------------------
%       DOCUMENT STRUCTURE COMMANDS
%       Skip this unless you know what you're doing
%----------------------------------------------------------------------------------------

% Header and footer for when a page split occurs within a problem environment
\newcommand{\enterProblemHeader}[1]{
\nobreak\extramarks{#1}{#1 continued on next page\ldots}\nobreak
\nobreak\extramarks{#1 (continued)}{#1 continued on next page\ldots}\nobreak
}

% Header and footer for when a page split occurs between problem environments
\newcommand{\exitProblemHeader}[1]{
\nobreak\extramarks{#1 (continued)}{#1 continued on next page\ldots}\nobreak
\nobreak\extramarks{#1}{}\nobreak
}

\setcounter{secnumdepth}{0} % Removes default section numbers
\newcounter{homeworkProblemCounter} % Creates a counter to keep track of the number of problems

\newcommand{\homeworkProblemName}{}
\newenvironment{homeworkProblem}[1][Problem \arabic{homeworkProblemCounter}]{ % Makes a new environment called homeworkProblem which takes 1 argument (custom name) but the default is "Problem #"
\stepcounter{homeworkProblemCounter} % Increase counter for number of problems
\renewcommand{\homeworkProblemName}{#1} % Assign \homeworkProblemName the name of the problem
\section{\homeworkProblemName} % Make a section in the document with the custom problem count
\enterProblemHeader{\homeworkProblemName} % Header and footer within the environment
}{
\exitProblemHeader{\homeworkProblemName} % Header and footer after the environment
}

\newcommand{\problemAnswer}[1]{ % Defines the problem answer command with the content as the only argument
\noindent\framebox[\columnwidth][c]{\begin{minipage}{0.98\columnwidth}#1\end{minipage}} % Makes the box around the problem answer and puts the content inside
}

\newcommand{\homeworkSectionName}{}
\newenvironment{homeworkSection}[1]{ % New environment for sections within homework problems, takes 1 argument - the name of the section
\renewcommand{\homeworkSectionName}{#1} % Assign \homeworkSectionName to the name of the section from the environment argument
\subsection{\homeworkSectionName} % Make a subsection with the custom name of the subsection
\enterProblemHeader{\homeworkProblemName\ [\homeworkSectionName]} % Header and footer within the environment
}{
\enterProblemHeader{\homeworkProblemName} % Header and footer after the environment
}
   
%----------------------------------------------------------------------------------------
%       NAME AND CLASS SECTION
%----------------------------------------------------------------------------------------

\newcommand{\hmwkTitle}{Assignment\ \# 2 } % Assignment title
\newcommand{\hmwkDueDate}{Friday,\ October\ 15,\ 2015} % Due date
\newcommand{\hmwkClass}{MATH-578B} % Course/class
\newcommand{\hmwkClassTime}{} % Class/lecture time
\newcommand{\hmwkAuthorName}{Saket Choudhary} % Your name
\newcommand{\hmwkAuthorID}{2170058637} % Teacher/lecturer
%----------------------------------------------------------------------------------------
%       TITLE PAGE
%----------------------------------------------------------------------------------------

\title{
\vspace{2in}
\textmd{\textbf{\hmwkClass:\ \hmwkTitle}}\\
\normalsize\vspace{0.1in}\small{Due\ on\ \hmwkDueDate}\\
\vspace{0.1in}\large{\textit{\hmwkClassTime}}
\vspace{3in}
}

\author{\textbf{\hmwkAuthorName} \\
        \textbf{\hmwkAuthorID}
        }
\date{} % Insert date here if you want it to appear below your name

%----------------------------------------------------------------------------------------

\begin{document}

\maketitle

%----------------------------------------------------------------------------------------
%       TABLE OF CONTENTS
%----------------------------------------------------------------------------------------

%\setcounter{tocdepth}{1} % Uncomment this line if you don't want subsections listed in the ToC

\newpage
\tableofcontents
\newpage



\begin{homeworkProblem}[Problem 1] % Roman numerals


\problemAnswer{ % Answer
Define $h(w)$ to be an indicator function: 
$$
h(x) = \begin{cases}
1 & x\in A\\
-1 & x \notin A
\end{cases}
$$

Now consider $E[h(W)]$:

\begin{align*}
E[h(W)] & = P(W \in A) \times 1 + P(W \notin A) \times -1 \\
&= P(W \in A) - (1-P(W \in A))\\
&= 2P(W \in A) -1 \tag{1.1}
\end{align*}

Similarly,

\begin{align*}
E[h(Z)] &= 2P(Z \in A)-1 \tag{1.2}
\end{align*}
where $A \in Z^{+}$

From (1.1), (1.2)

\begin{align*}
E(h(W))-E(h(Z)) & = 2P(W \in A)-2P(Z \in A)\\
|E(h(W))-E(h(Z))| &= 2|P(W \in A)-P(Z \in A)|\\
max_{h:||h||=1}|E(h(W))-E(h(Z))| &= 2max_{A \in Z^{+}}|P(W \in A)-P(Z \in A)|\\
&= 2|P(W = 0)-P(Z=0) + P(W=1)-P(Z=1) + \cdots|\\
&= \sum_{k \geq 0 } |P(W=k)-P(Z=k)|
\end{align*}

}
\end{homeworkProblem}
\begin{homeworkProblem}[Problem 2]
	\problemAnswer{
		$\boldmath{w} = 11011011$
		
		Periods: $P(\omega) = \{p \in \{1,2 \cdots, w-1\}: w_i = w_{i+p} \}\ \forall i \in \{1,2 \dots w-p\}$
		
		$w_1 = w_4; w_2=w_5;w_3=w_6;\dots w_5=w_8$
		
		$w_1 = w_7$
		
		$w_1 = w_8$
		
		Thus $$P(\omega) = \{3,6,7\}$$
		
		$P(\omega) =6$ can be written as multiple of $P(\omega)=3$ and hence the principal period 
		$$P(\omega') = \{3,7\}$$
		
		The mean $(n-w+1)\mu(w)$ using poisson approximation for the number of clumps is given by: $P(\omega) - \sum_{p \in P(\omega')}P(w^{(p)}w)$
		
		$w=11011011$
		
		$P(\omega) = \{3,6,7\}$
		
		$P(\omega') = \{3,7\}$
		\begin{align*}
		\mu(w) &= P(11011011) - P(110w) -P(1101101w)\\
		&= p^6q^2 - p^2q(p^6q^2) - p^5q^2(p^6q^2)\\
		&= p^6q^2(1-p^2q-p^5q^2)
		\end{align*}
		
		And hence the mean of the poisson approximation is given by $(n-7)p^6q^2(1-p^2q-p^5q^2)$
		
	
		
		where $C_j$ is the even that there are $j$ consecutive occurrences of $w$ starting at that particular location.
		
		$C_j = \{w^{(p_1)}w^{(p_2)} \dots w^{(p_{j-1}w)}\}$ where $p_i$ is  a principal period of $w$.
		and the mean of this process is given by $\mu_j(w) = P(C_j)-2P(C_{j+1})+P(C_{j+2})$
		
		$P(C_j)  $ calculation:
		
		Let there be $kw^{(3)}$ and $j-1-k$ with $w^{(7)}$ among the first $j-1$ occurrences of $w$ starting at $i$. The probability of this is ${{j-1}\choose{k}}(p^2q)^k(p^5q^2)^{j-1-k} $
		
		Thus,
		\begin{align*}
		P(C_j) &= P(w) \sum_{k=0}^{j-1} (P(w^{(3)})^k(P(w^{(7)}))^{j-1-k}\\
		&=p^6q^2(p^2q+p^5q^2)^{j-1}\\
		&=p^{2j+4}q^{j+1}(1+p^3q) ^{j-1}\\
		P(C1) &= p^6q^2\\
		P(C2) &= p^8q^3(1+p^3q)\\
		P(C1)-P(C2) &= p^6q^2(1-p^2q-p^5q^2)\\
		\mu_j(w) &= p^{2j+4}q^{j+1}(1+p^3q) ^{j-1} (1-2p^2q(1+p^3q)+p^4q^2(1+p^3q)^2)
		\end{align*}
		
		Given a clump, 	rhe number of occurrences of $w$ can be approximated by:
		\begin{align*}
		P(L_i=j) &= \frac{u_j(w)}{\sum_{i\geq 1} \mu_j(w)}\\
		&= \frac{p^{2j+4}q^{j+1}(1+p^3q) ^{j-1} (1-2p^2q(1+p^3q)+p^4q^2(1+p^3q)^2)}{ p^6q^2(1-p^2q-p^5q^2)}
		\end{align*}

		

		
	}
\end{homeworkProblem}



\begin{homeworkProblem}[Exercise 6.1]
	\problemAnswer{
		Expected number of apparent islands = $N e^{-c(1-\theta)} = cg e^{-c(1-\theta)}$
		
		where $g = \frac{G}{L}$ (since $cg \rightarrow \infty $) as $g \rightarrow \infty$
		
		\begin{align*}
		f(c) &=cg e^{-c(1-\theta)}\\
			\frac{\partial f(c)}{\partial c} &= g(e^{-c(1-\theta)} + -c(1-\theta)e^{-c(1-\theta)}) = 0\\
			\implies 1-c(1-\theta) &=0\\
			c^{*} &= (1-\theta)^{-1}
			\end{align*}
			
			and the maximum number of apparent islands is $N e^{-c^{*}(1-\theta)} = Ne^{-1} = \frac{G}{L}e^{-1}(1-\theta)^{-1}$
		}
	\end{homeworkProblem}
	

	\begin{homeworkProblem}[Exercise 6.5]
		\problemAnswer{
			From our solution for $Excersice 6.1$ we see that the the maximum number of apparent islands occur
			at $c^{*} = (1-\theta)^{-1}$ Now $c$ was defined to be the expected number of clones covering a point.
			and $c^{*}$ is the maximum number of clones arranged such that no two clones overlap.(overlap if at all is less than $\theta$)
			Then the maximum number of islands at any point would be $ceil((1-\theta)^{-1}) = 1 + max\{k: \text{k integer}, k < (1-\theta)^{-1}\}$ 
		}
	\end{homeworkProblem}
	
	
	
	\begin{homeworkProblem}[Excercise 11.14]
\problemAnswer{
	Stein equation: $(Lf)(x) = \lambda f(x+1) -xf(x)$ and $W=\sum_{i=1}^n X_i$
	
	$E((Lf)(W)) =0 $
	
	And $Z \sim Poisson(E(W))$

Since $X_i$ is iid.
Note: $J_i = \{i\}$ i.e since this is an iid scenarios, the only dependent variable to $X_i$, is $X_i$ itself (this requires $p$ to be $\neq 0$ and $\neq 1$)

$$b_1 = \sum_{i \in I} \sum_{j \in J_i} E(X_i)E(X_j) = \sum_{i=1}p^2 = np^2$$

$$b_2 =  \sum_{i \in I} \sum_{i \neq j \in J_i} E(X_iX_j)	= \sum_i \sum_{i \neq j \in I_i} E(X_iX_j) = 0$$

And hence, by Theorem 11.22:

\begin{align*}
||W-Z|| \leq 2np^2 \frac{1-e^-{\lambda}}{\lambda} \leq 2np^2
\end{align*}
where $EZ=EW=\lambda$

	}
	\end{homeworkProblem}
	
\begin{homeworkProblem}[Exercise 11.10]
	\problemAnswer{
		\begin{align*}
		X_i &= \prod_{j=1}^{i+t-1}D_j\\
		W &= \sum_{i \in I}X_i\\
		\end{align*}
		$I = \{1,2,\dots, n-t+1 \} $ and $J_i= \{ j \in I: |i-j| <t \}$ 
		
		$EW = \lambda  = (n-t+1)p^t$
		
		 Since $p<1$ define $q=1/p$
		 
		 Now, $EW = \frac{n-t+1}{q^t} $ and as $n \rightarrow \infty$ 
		 In order to prevent $EW \rightarrow \infty$, we make $t \rightarrow n$ when $n \rightarrow \infty$
		 
		 $$ i-(t-1) \leq j \leq i+(t-1) $$
		 
		 $$b_1 = \sum_i p^t \times ((2t-2)+1)p^t = (n-t+1)(2t-1)p^{2t}$$
		 
		 $$b_2 = \sum_{i} p^{(2t-2)+1}  = (n-t+1)p^{2t-1}$$
		 
		 Now,
		 Let $q=1/p$
			\begin{align*}
				\lim_{n \rightarrow \infty} b_1 &= \lim_{n \rightarrow \infty}\frac{(n-t+1)(2t-1)}{q^{2t}} \\
				&= \frac{2(n-t+1) +(-1)(2t-1)}{q^{2t}\ln(q)} \ \text{ Using LHospital Rule}\\
				&= \frac{2n-4t+3}{q^{2t}\ln(q)}
			\end{align*}
			
			Thus, for $b_1$  to be finite, $t \rightarrow n/2$ as $n \rightarrow \infty $
			
			and similar is the case for $b_2$. Thus for $t \rightarrow \infty$, $b_1$, $b_2$ are both bounded and infact $b_1,b_2 \rightarrow 0$ as $n \infty$ or as $t \rightarrow n$
		}
\end{homeworkProblem}
\end{document}
