%%%%%%%%%%%%%%%%%%%%%%%%%%%%%%%%%%%%%%%%
% Structured General Purpose Assignment
% LaTeX Template
%
% This template has been downloaded from:
% http://www.latextemplates.com
%
% Original author:
% Ted Pavlic (http://www.tedpavlic.com)
%
% Note:
% The \lipsum[#] commands throughout this template generate dummy text
% to fill the template out. These commands should all be removed when 
% writing assignment content.
%
%%%%%%%%%%%%%%%%%%%%%%%%%%%%%%%%%%%%%%%%%

%----------------------------------------------------------------------------------------
%       PACKAGES AND OTHER DOCUMENT CONFIGURATIONS
%----------------------------------------------------------------------------------------

\documentclass{article}

\usepackage{fancyhdr} % Required for custom headers
\usepackage{lastpage} % Required to determine the last page for the footer
\usepackage{extramarks} % Required for headers and footers
\usepackage{graphicx} % Required to insert images
\usepackage{lipsum} % Used for inserting dummy 'Lorem ipsum' text into the template

\usepackage{amsmath}

% Margins
\topmargin=-0.45in
\evensidemargin=0in
\oddsidemargin=0in
\textwidth=6.5in
\textheight=9.0in
\headsep=0.25in 

\linespread{1.1} % Line spacing

% Set up the header and footer
\pagestyle{fancy}
\lhead{\hmwkAuthorName} % Top left header
\chead{\hmwkClass\ : \hmwkTitle} % Top center header
\rhead{\firstxmark} % Top right header
\lfoot{\lastxmark} % Bottom left footer
\cfoot{} % Bottom center footer
\rfoot{Page\ \thepage\ of\ \pageref{LastPage}} % Bottom right footer
\renewcommand\headrulewidth{0.4pt} % Size of the header rule
\renewcommand\footrulewidth{0.4pt} % Size of the footer rule

\setlength\parindent{0pt} % Removes all indentation from paragraphs

%----------------------------------------------------------------------------------------
%       DOCUMENT STRUCTURE COMMANDS
%       Skip this unless you know what you're doing
%----------------------------------------------------------------------------------------

% Header and footer for when a page split occurs within a problem environment
\newcommand{\enterProblemHeader}[1]{
\nobreak\extramarks{#1}{#1 continued on next page\ldots}\nobreak
\nobreak\extramarks{#1 (continued)}{#1 continued on next page\ldots}\nobreak
}

% Header and footer for when a page split occurs between problem environments
\newcommand{\exitProblemHeader}[1]{
\nobreak\extramarks{#1 (continued)}{#1 continued on next page\ldots}\nobreak
\nobreak\extramarks{#1}{}\nobreak
}

\setcounter{secnumdepth}{0} % Removes default section numbers
\newcounter{homeworkProblemCounter} % Creates a counter to keep track of the number of problems

\newcommand{\homeworkProblemName}{}
\newenvironment{homeworkProblem}[1][Problem \arabic{homeworkProblemCounter}]{ % Makes a new environment called homeworkProblem which takes 1 argument (custom name) but the default is "Problem #"
\stepcounter{homeworkProblemCounter} % Increase counter for number of problems
\renewcommand{\homeworkProblemName}{#1} % Assign \homeworkProblemName the name of the problem
\section{\homeworkProblemName} % Make a section in the document with the custom problem count
\enterProblemHeader{\homeworkProblemName} % Header and footer within the environment
}{
\exitProblemHeader{\homeworkProblemName} % Header and footer after the environment
}

\newcommand{\problemAnswer}[1]{ % Defines the problem answer command with the content as the only argument
\noindent\framebox[\columnwidth][c]{\begin{minipage}{0.98\columnwidth}#1\end{minipage}} % Makes the box around the problem answer and puts the content inside
}

\newcommand{\homeworkSectionName}{}
\newenvironment{homeworkSection}[1]{ % New environment for sections within homework problems, takes 1 argument - the name of the section
\renewcommand{\homeworkSectionName}{#1} % Assign \homeworkSectionName to the name of the section from the environment argument
\subsection{\homeworkSectionName} % Make a subsection with the custom name of the subsection
\enterProblemHeader{\homeworkProblemName\ [\homeworkSectionName]} % Header and footer within the environment
}{
\enterProblemHeader{\homeworkProblemName} % Header and footer after the environment
}
   
%----------------------------------------------------------------------------------------
%       NAME AND CLASS SECTION
%----------------------------------------------------------------------------------------

\newcommand{\hmwkTitle}{Assignment\ \# 3 } % Assignment title
\newcommand{\hmwkDueDate}{Thursday,\ October\ 22,\ 2015} % Due date
\newcommand{\hmwkClass}{MATH-578B} % Course/class
\newcommand{\hmwkClassTime}{} % Class/lecture time
\newcommand{\hmwkAuthorName}{Saket Choudhary} % Your name
\newcommand{\hmwkAuthorID}{2170058637} % Teacher/lecturer
%----------------------------------------------------------------------------------------
%       TITLE PAGE
%----------------------------------------------------------------------------------------

\title{
\vspace{2in}
\textmd{\textbf{\hmwkClass:\ \hmwkTitle}}\\
\normalsize\vspace{0.1in}\small{Due\ on\ \hmwkDueDate}\\
\vspace{0.1in}\large{\textit{\hmwkClassTime}}
\vspace{3in}
}

\author{\textbf{\hmwkAuthorName} \\
        \textbf{\hmwkAuthorID}
        }
\date{} % Insert date here if you want it to appear below your name

%----------------------------------------------------------------------------------------

\begin{document}

\maketitle

%----------------------------------------------------------------------------------------
%       TABLE OF CONTENTS
%----------------------------------------------------------------------------------------

%\setcounter{tocdepth}{1} % Uncomment this line if you don't want subsections listed in the ToC

\newpage
\tableofcontents
\newpage



\begin{homeworkProblem}[Problem 1] % Roman numerals


\problemAnswer{ % Answer

The coverage $c$ depends on the position $x$ as: $c=\frac{NL_x}{G}$ where $L_x$ is the expected length of  of clones covering $x$.

Probability any position $x$ to be covered by atleast one clone = $(1-$ Probability that it is sequenced by atleast one clone).

Probability that position $x$ is not sequenced = Probability of zero clones starting in $(x-L,x]$ = No arrivals in the interval $(x-L,x]$ = $e^{-c(x)}$ 


Probability that it is sequenced = $1-e^{-c(x)}$ where $c(x)$ represents that c is a function of $x$.

$ C \sim \Gamma(\alpha,\beta)$

$$f(c) = \frac{c^{\alpha-1}e^{-c/\beta}}{\beta^\alpha \Gamma(\alpha)}$$
Thus, 
\begin{align*}
P(N_h = k) = \int_0 ^ \infty e^{-ch}\frac{(ch)^k}{k!} \times \frac{c^{\alpha-1}e^{-c/\beta}}{\beta^\alpha \Gamma(\alpha)} dc
\end{align*}



}
\end{homeworkProblem}
\begin{homeworkProblem}[Problem 2]
	\problemAnswer{
		Given: $\lim_{n \rightarrow \infty} (1-F(b\log(n)+x/a)) = G(x)$
		
		\begin{align*}
		\lim_{n \rightarrow \infty} (1-F(b\log(n)+x/a)) &= G(x)\\
		\lim_{n \rightarrow \infty} F(b\log(n)+x/a) &= 1-G(x)/n\\
		\end{align*}
		
		\begin{align*}
		P(a(max_i X_i-b\log(n))\leq x) &= P(max_i X_i \leq x/a+b\log(n))\\
		&= P(X_1 \leq x/a+b\log(n))P(X_2 \leq x/a+b\log(n))\dots P(X_n \leq x/a+b\log(n))\\
		&= (F(x/a+b\log(n)))^n\\
		&= \lim n_infty(1-G(x)/n)^n\\
		&= \lim_{n \rightarrow \infty}e^{n\log(1-G(x)/n)}\\
		&= e^{-G(x)}
		\end{align*}
		
Choosing $a,b$ for $G(x)=e^{-x}$ given $X_i \sim exponential(\lambda)$

$f(x|\lambda) = \lambda e^-{\lambda x} \implies F(x) = 1-e^{-\lambda x}$

Now,
\begin{align*}
\lim {n \rightarrow \infty}1-G(x)/n &= F(b\log(n)+x/a)\\
&= 1-e^{-\lambda (b\log(n)+x/a)}\\
e^{-x}/n &= e^{-\lambda (b\log(n)+x/a)}\\
-x &= \log(n) + -\lambda(b\log(n)+x/a)\\
x(-1+\lambda/a) &= \log(n)-b\lambda\log(n)\\
%1/n - x/n + x^2/2n - x^3/6n \dots &= 1-\lambda (b\log(n)+x/a)+(\lambda (b\log(n)+x/a))^2/2 +
\end{align*}

Thus, $$a=\lambda ; b=\frac{1}{\lambda}$$ 

		
	
}
\end{homeworkProblem}

\begin{homeworkProblem}[Problem 3]
	\problemAnswer{
		Target Distribution in aligned region: $P(R,R)=0.2$ ; $P(Y,Y)=0.7$;  $P(R,Y)=0.1$
		
		\begin{align*}
		\xi_r &= 0.2\\
		\xi_y &= 0.8\\
		\end{align*}
		
		By Theorem 11.7 we have that thte proportion of letter $a$  aligning with letter $b$ in the best matching interval converges to:
		$$
		p(a,b)=\xi_a\xi_bp^{-s(a,b)}
		$$
		
		Equivalently:
		
		$$
		s(a,b) = \log_{1/p}(\frac{p(a,b)}{\xi_a\xi_b})
		$$
		
		$$p = \xi_r \xi_r + \xi_y \xi_y = 0.68$$
		
		
		Thus
		\begin{align*}
		P(RR) & = \xi_r\xi_rp^{-s(r,r)}\\
		s(r,r)&= \log_{1/0.68}(\frac{0.2}{0.04})\\
		&= 4.17		
		\end{align*}		
		\begin{align*}
		s(r,y) &= \log_{1/p}(\frac{p(r,y)}{\xi_r\xi_y})\\
		s(r,y)&= \log_{1/0.68}(\frac{0.1}{0.16})\\		
		&= -1.21		
		\end{align*}		
		\begin{align*}
		s(y,y) &= \log_{1/p}(\frac{p(y,y)}{\xi_y\xi_y})\\
		s(y,y)&= \log_{1/0.68}(\frac{0.7}{0.64})\\		
		&= 0.23		
		\end{align*}
		\begin{center}
		\boxed{s(r,r) = 4.17	}\\
		\boxed{s(y,y) = 0.23} \\
		\boxed{s(y,r) = -1.21}			
		\end{center}
	}
	\problemAnswer{
		To find the value of $\lambda$ such that:
		$$
		\lim_{n \rightarrow \infty, m \rightarrow \infty} P\{\lambda R_{mn} - \log(K_{mn}) <x\} = exp(-exp(-x))
		$$
		
		
		$$\lambda = log(1/p) = 0.38$$
		
		And given that the score for 1000bp alignment is 100,
		the p value is given by;
		
		$$p-value = 1-e^{-e^{-s}} $$
			
		where $s= \lambda R_{mn} - \log(Kmn)$
		
		If the p-value is less than some pre-defined threshold, the hypothesis that alignment is as good as by random chance can be rejected.
		}
\end{homeworkProblem}


\begin{homeworkProblem}[Problem 4]
	\problemAnswer{
		Minimal neighborhood set $J_{i,j}$ such that $\{i',j' \in J_{i',j'}^c\}$ are independent of $Y_{i,j}$ is given by: $\{(i',j'): |i-i'| \leq t or |j-j'|\leq t \}$
		
		
		Now,
		
		\begin{align*}
		b_1 &= \sum_{i \in I}\sum_{j in J_i} E(X_i)E(X_j)\\
		&= p^t \sum_{j \in J_i} E(X_j) + \sum_{i=2}^{n-t+1}(1-p)p^t \sum_{j \in J_i}E(X_j)\\
		&= (n-t+1)p^t (2t+1)p^t \times 2 + (n-t+1)^2(1-p)^2p^{2t}(4t+2) \\
		&= p^{2t}(n-t+1)(4t+2)(1+(n-t+1)(1-p)^2)
		%&= p^{2t}(3t+1 + (n-t)(4t+3)(1-p)^2 )
		\end{align*}
		
		\begin{align*}
			E[NC_n] &=(n-t+1)^2(1-p)p^t\\
			& = \lambda
		\end{align*}
		Thus, $$n^2(1-p)p^t \sim \lambda$$
		
		\begin{align*}
\log(n^2(1-p)) + t\log(p) = \log(\lambda)\\
t = -\log_{1/p}(\lambda) + 2\log_{1/p}(n(1-p))
		\end{align*}
		
		And $b1$ can be approximated as:
		$$
			(n-t+1)^2(4t+2)(1-p)^2p^{2t}
		$$
		If we approximate $t_n = 2\log_{1/p}n +x$
		
		\begin{align*}
		(n-t+1)^2(4t+2)(1-p)^2p^{2t} & = (n-2\log_{1/p} n +1)^2(4(2\log_{1/p} n+x)+2)((1-p)p^{2\log_{1/p}n+x})^2\\
		 &=(n-2\log_{1/p} n +1)^2(4(2\log_{1/p} n+x)+2)((1-p) \times 1/n^2)^2 \rightarrow 0	\\
		 & 	\rightarrow 0	
		\end{align*}
		

				}
\end{homeworkProblem}

\begin{homeworkProblem}[Exercise 11.9]
\problemAnswer{
Part (a):
$\xi_a = 1/|A|$ 

$$ p = \sum_{a\in A} \xi_a\xi_a = |A|/|A|^2 = 1/|A| $$

\begin{align*}
\mu_{a,a} &= \xi_a \xi_a p^{-s(a,a)}\\
&= 1/|A|^2 \times |A|\\
&= 1/|A|
\end{align*}


Part (b): To Derive $s(a,b)$ such that $\mu_{a,a} = 1/|A|$
We have(from problem 3):
$$
	s(a,b) = \log_{1/p}(\frac{p(a,b)}{\xi_a\xi_b})
$$

And hence:


$$
s(a,a) = \log_{|A|}(\frac{1/|A|}{1/|A|^2}) = 1
$$

Thus, $$s(a,a)=1;s(a,b)= -\infty$$




}
\end{homeworkProblem}

\begin{homeworkProblem}[Exercise 11.11]
	\problemAnswer{
%If there are N sequences we have $n^N$ choices for $(i_i,i_2,i_3 \dots i_n)$ where $i_j$ represents the starting position for the $j^{th}$ sequence.
%and hence using naive approach(Largest run being unique and satisfying %$np^{R_n}=1$ and then by allowing shifts there are $n^N$ choices for starting position in each sequence): 

%$H_n$ grows like $$\log_{1/p}n^N = N \log_{1/p} n$$

%and hence $$t(n,N) = N \log_{1/p} n$$

%As $$N \rightarrow \infty \implies t(n,N) \rightarrow \infty$$
$E[NC] = (n-t+1)^Np^t$

The probability of $x$ being the starting position of alignment in the $N$ sequences: $p=(\frac{1}{|A|})^N \times |A| = 1/|A|^{N-1} = 1/n^{N-1}$ (since A,B are iid)

$$|A| = n$$
Thus, assuming the longest run is unique, we have that the expected number of runs of this length be 1:
\begin{align*}
(n-t+1)^Np^t &=1\\
(n-t+1)^N 1/n^{Nt-t} &=1\\
n^{t-Nt+N}&\approx  1\\
t &= \frac{1}{N-1}\log_{1/n} N
\end{align*}

And hence as $$N\rightarrow \infty \implies t \rightarrow 1/N^2 \rightarrow 0$$
}
\end{homeworkProblem}
\end{document}

