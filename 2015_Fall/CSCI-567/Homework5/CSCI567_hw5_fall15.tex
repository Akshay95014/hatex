%%%%%%%%%%%%%%%%%%%%%%%%%%%%%%%%%%%%%%%%
% Structured General Purpose Assignment
% LaTeX Template
%
% This template has been downloaded from:
% http://www.latextemplates.com
%
% Original author:
% Ted Pavlic (http://www.tedpavlic.com)
%
% Note:
% The \lipsum[#] commands throughout this template generate dummy text
% to fill the template out. These commands should all be removed when 
% writing assignment content.
%
%%%%%%%%%%%%%%%%%%%%%%%%%%%%%%%%%%%%%%%%%

%----------------------------------------------------------------------------------------
%       PACKAGES AND OTHER DOCUMENT CONFIGURATIONS
%----------------------------------------------------------------------------------------

\documentclass{article}

\usepackage{fancyhdr} % Required for custom headers
\usepackage{lastpage} % Required to determine the last page for the footer
\usepackage{extramarks} % Required for headers and footers
\usepackage{graphicx} % Required to insert images
\usepackage{lipsum} % Used for inserting dummy 'Lorem ipsum' text into the template
\usepackage{subcaption}


\usepackage{amsmath}

% Margins
\topmargin=-0.45in
\evensidemargin=0in
\oddsidemargin=0in
\textwidth=6.5in
\textheight=9.0in
\headsep=0.25in 

\linespread{1.1} % Line spacing

% Set up the header and footer
\pagestyle{fancy}
\lhead{\hmwkAuthorName} % Top left header
\chead{\hmwkClass\ : \hmwkTitle} % Top center header
\rhead{\firstxmark} % Top right header
\lfoot{\lastxmark} % Bottom left footer
\cfoot{} % Bottom center footer
\rfoot{Page\ \thepage\ of\ \pageref{LastPage}} % Bottom right footer
\renewcommand\headrulewidth{0.4pt} % Size of the header rule
\renewcommand\footrulewidth{0.4pt} % Size of the footer rule

\setlength\parindent{0pt} % Removes all indentation from paragraphs

%----------------------------------------------------------------------------------------
%       DOCUMENT STRUCTURE COMMANDS
%       Skip this unless you know what you're doing
%----------------------------------------------------------------------------------------

% Header and footer for when a page split occurs within a problem environment
\newcommand{\enterProblemHeader}[1]{
\nobreak\extramarks{#1}{#1 continued on next page\ldots}\nobreak
\nobreak\extramarks{#1 (continued)}{#1 continued on next page\ldots}\nobreak
}

% Header and footer for when a page split occurs between problem environments
\newcommand{\exitProblemHeader}[1]{
\nobreak\extramarks{#1 (continued)}{#1 continued on next page\ldots}\nobreak
\nobreak\extramarks{#1}{}\nobreak
}

\setcounter{secnumdepth}{0} % Removes default section numbers
\newcounter{homeworkProblemCounter} % Creates a counter to keep track of the number of problems

\newcommand{\homeworkProblemName}{}
\newenvironment{homeworkProblem}[1][Problem \arabic{homeworkProblemCounter}]{ % Makes a new environment called homeworkProblem which takes 1 argument (custom name) but the default is "Problem #"
\stepcounter{homeworkProblemCounter} % Increase counter for number of problems
\renewcommand{\homeworkProblemName}{#1} % Assign \homeworkProblemName the name of the problem
\section{\homeworkProblemName} % Make a section in the document with the custom problem count
\enterProblemHeader{\homeworkProblemName} % Header and footer within the environment
}{
\exitProblemHeader{\homeworkProblemName} % Header and footer after the environment
}

\newcommand{\problemAnswer}[1]{ % Defines the problem answer command with the content as the only argument
\noindent\framebox[\columnwidth][c]{\begin{minipage}{0.98\columnwidth}#1\end{minipage}} % Makes the box around the problem answer and puts the content inside
}

\newcommand{\homeworkSectionName}{}
\newenvironment{homeworkSection}[1]{ % New environment for sections within homework problems, takes 1 argument - the name of the section
\renewcommand{\homeworkSectionName}{#1} % Assign \homeworkSectionName to the name of the section from the environment argument
\subsection{\homeworkSectionName} % Make a subsection with the custom name of the subsection
\enterProblemHeader{\homeworkProblemName\ [\homeworkSectionName]} % Header and footer within the environment
}{
\enterProblemHeader{\homeworkProblemName} % Header and footer after the environment
}
   
%----------------------------------------------------------------------------------------
%       NAME AND CLASS SECTION
%----------------------------------------------------------------------------------------

\newcommand{\hmwkTitle}{Assignment\ \#5 } % Assignment title
\newcommand{\hmwkDueDate}{Monday,\ November\ 16,\ 2015} % Due date
\newcommand{\hmwkClass}{CSCI-567} % Course/class
\newcommand{\hmwkClassTime}{} % Class/lecture time
\newcommand{\hmwkAuthorName}{Saket Choudhary} % Your name
\newcommand{\hmwkAuthorID}{2170058637} % Teacher/lecturer
%----------------------------------------------------------------------------------------
%       TITLE PAGE
%----------------------------------------------------------------------------------------

\title{
\vspace{2in}
\textmd{\textbf{\hmwkClass:\ \hmwkTitle}}\\
\normalsize\vspace{0.1in}\small{Due\ on\ \hmwkDueDate}\\
\vspace{0.1in}\large{\textit{\hmwkClassTime}}
\vspace{3in}
}

\author{\textbf{\hmwkAuthorName} \\
        \textbf{\hmwkAuthorID}
        }
\date{} % Insert date here if you want it to appear below your name

%----------------------------------------------------------------------------------------

\begin{document}

\maketitle

%----------------------------------------------------------------------------------------
%       TABLE OF CONTENTS
%----------------------------------------------------------------------------------------

%\setcounter{tocdepth}{1} % Uncomment this line if you don't want subsections listed in the ToC

\newpage
\tableofcontents
\newpage


%----------------------------------------------------------------------------------------
%       PROBLEM 2
%----------------------------------------------------------------------------------------





\begin{homeworkProblem}[Problem 1] % Custom section title
\begin{homeworkSection}{\homeworkProblemName: ~(a)}

\problemAnswer{
	To find $\nabla_{y_t} L$:
	
	\begin{align*}
	\nabla_{y_t}L &= \frac{\partial}{\partial y_t} \frac{1}{2}\sum_{i=1}^{N}()
	\end{align*}
}
		\end{homeworkSection}
\end{homeworkProblem}

\begin{homeworkProblem}[Problem 3]
	\problemAnswer{
		Given:
		$$
		p(x_i) = \begin{cases}
		\pi+(1-\pi)e^{-\lambda} & x_i=0\\
		(1-\pi)\frac{\lambda^{x_i}e^{-\lambda}}{x_i!} & x_i>0
		\end{cases}
		$$
		
		Alternatively:
		
		$$
		X_i = \begin{cases}
		0 & \text{probability = }  \pi+(1-\pi)e^{-\lambda}\\
		x_i & \text{probability = } (1-\pi)\frac{\lambda^{x_i}e^{-\lambda}}{x_i!}
		\end{cases}
		$$
		
		We define a $latent$ variable $Z_i$ for all cases where $X_i=0$. It is latent because when we observed $X_i=0$ we do not know if it came out of the 'Poisson' distribution or it came out the 'degenerate' distribution(which has a probability of $1$ at point 0.).  we cannot observe the following. So $X_i$ comes out of a mixture of a degenerate distribution as follows:
		
		$$
		Z_i = \begin{cases}
		1 &  \text{$X_i$ is from the degenerate distrbution }\\
		0 & \text{otherwise}
		\end{cases}
		$$
		
		\begin{eqnarray*}
		p(X_i=0,Z_i=1) = p(Z_i=1) \times p(X_i=0|Z_i=1) = \pi \times 1\\
		P(X_i=0,Z_i=0) = p(Z_i=0) \times p(X_i=0|Z_i=0) = (1-\pi)e^{-\lambda} \times 1
		\end{eqnarray*}
		
		\begin{eqnarray}
		L(Complete) = \prod_{x_i=0} \pi^{Z_i} \times ((1-\pi)e^{-\lambda})^{1-Z_i} \times \prod_{x_i > 0} (1-\pi)e^\frac{\lambda^x_i e^{-\lambda}}{x_i!}\\
		\log L = \sum_{I(x_i=0)} z_i \log(\pi) + (1-z_i) \big( \log(1-\pi) - \lambda \big)  + \sum_{I(x_i>0)} \big( \log(1-\pi) + x_i \log(\lambda_i) - \lambda - \log(x_i!) \big)
		\end{eqnarray}
		
		
		E step:
		
\begin{align*}	
		Q(\theta, \theta_0) &= \sum_{I(x_i=0)} E_{P(Z|X)}[z_i] \log(\pi) + (1-E_{P(Z|X)}[z_i]) \big( \log(1-\pi) - \lambda \big)\\  
		&+ \sum_{I(x_i>0)} \big( \log(1-\pi) + x_i \log(\lambda_i) - \lambda - \log(x_i!) \big)
\end{align*}


\begin{align*}	
E_{P(Z|X_i)}[z_i] &= 0 \times p(Z_i=0|X) + 1 \times p(Z_i=1|X_i=0) \\
&= \frac{p(X_i=0|Z_i=1)p(Z_i=1)}{p(X_i=0|Z_i=0)p(Z_i=0)+p(X_i=0|Z_i=1)p(Z_i=1)}\\
&= \frac{\pi_0}{\pi_0+(1-\pi_0)e^{-\lambda_0}}
\end{align*}

Hence,

\begin{align*}
Q(\theta, \theta_0) &= \sum_{I(x_i=0)} \frac{\pi_0}{\pi_0+(1-\pi_0)e^{-\lambda_0}} \log(\pi) + (\frac{(1-\pi_0)e^{-\lambda_0}}{\pi_0+(1-\pi_0)e^{-\lambda_0}}) \big( \log(1-\pi) - \lambda \big)\\  
&+ \sum_{I(x_i>0)} \big( \log(1-\pi) + x_i \log(\lambda) - \lambda - \log(x_i!) \big)
\end{align*}
}

\problemAnswer{
M step:
\begin{align*}
\frac{\partial Q}{\partial \lambda} &=0\\
&= \sum_{I(x_i=0)} (1-E[z_i])(-1) + \sum_{I(x_i>0)} (\frac{x_i}{\lambda}-1)  =0\\
\implies \hat{\lambda} &= \frac{\sum_{I(x_i>0)}x_i}{n-\sum_{I(x_i=0)}E[z_i]} \\
\hat{\lambda}  &= \frac{\sum_{I(x_i>0)}x_i}{n-\sum_{I(x_i=0)}\hat{z_i}} \\
\text{where } \hat{z} &= \frac{\pi_0}{\pi_0+(1-\pi_0)e^{-\lambda_0}}
\end{align*}
		

\begin{align*}
\frac{\partial Q}{\partial \pi} &=0\\
&=  \sum_{I(x_i=0)} \big(\frac{E[z_i]}{\pi} - \frac{1-E[z_i]}{1-\pi}\big) - \sum_{I(x_i>0)}  \frac{1}{1-\pi} =0 \\
&= \sum_{I(x_i=0)} \big(\frac{E[z_i]}{\pi} + \frac{E[z_i]}{1-\pi}\big) - \frac{n}{1-\pi} = 0\\
\implies \hat{\pi} &= \sum_{I(x_i=0)} \frac{\hat{z_i}}{n}
\end{align*}
		
		
		
		}
\end{homeworkProblem}

\end{document}
