%%%%%%%%%%%%%%%%%%%%%%%%%%%%%%%%%%%%%%%%
% Structured General Purpose Assignment
% LaTeX Template
%
% This template has been downloaded from:
% http://www.latextemplates.com
%
% Original author:
% Ted Pavlic (http://www.tedpavlic.com)
%
% Note:
% The \lipsum[#] commands throughout this template generate dummy text
% to fill the template out. These commands should all be removed when 
% writing assignment content.
%
%%%%%%%%%%%%%%%%%%%%%%%%%%%%%%%%%%%%%%%%%

%----------------------------------------------------------------------------------------
%       PACKAGES AND OTHER DOCUMENT CONFIGURATIONS
%----------------------------------------------------------------------------------------

\documentclass{article}

\usepackage{fancyhdr} % Required for custom headers
\usepackage{lastpage} % Required to determine the last page for the footer
\usepackage{extramarks} % Required for headers and footers
\usepackage{graphicx} % Required to insert images
\usepackage{lipsum} % Used for inserting dummy 'Lorem ipsum' text into the template

\usepackage{amsmath}
    \usepackage{todonotes}

% Margins
\topmargin=-0.45in
\evensidemargin=0in
\oddsidemargin=0in
\textwidth=6.5in
\textheight=9.0in
\headsep=0.25in 

\linespread{1.1} % Line spacing

% Set up the header and footer
\pagestyle{fancy}
\lhead{\hmwkAuthorName} % Top left header
\chead{\hmwkClass\ : \hmwkTitle} % Top center header
\rhead{\firstxmark} % Top right header
\lfoot{\lastxmark} % Bottom left footer
\cfoot{} % Bottom center footer
\rfoot{Page\ \thepage\ of\ \pageref{LastPage}} % Bottom right footer
\renewcommand\headrulewidth{0.4pt} % Size of the header rule
\renewcommand\footrulewidth{0.4pt} % Size of the footer rule

\setlength\parindent{0pt} % Removes all indentation from paragraphs

%----------------------------------------------------------------------------------------
%       DOCUMENT STRUCTURE COMMANDS
%       Skip this unless you know what you're doing
%----------------------------------------------------------------------------------------

% Header and footer for when a page split occurs within a problem environment
\newcommand{\enterProblemHeader}[1]{
\nobreak\extramarks{#1}{#1 continued on next page\ldots}\nobreak
\nobreak\extramarks{#1 (continued)}{#1 continued on next page\ldots}\nobreak
}

% Header and footer for when a page split occurs between problem environments
\newcommand{\exitProblemHeader}[1]{
\nobreak\extramarks{#1 (continued)}{#1 continued on next page\ldots}\nobreak
\nobreak\extramarks{#1}{}\nobreak
}

\setcounter{secnumdepth}{0} % Removes default section numbers
\newcounter{homeworkProblemCounter} % Creates a counter to keep track of the number of problems

\newcommand{\homeworkProblemName}{}
\newenvironment{homeworkProblem}[1][Problem \arabic{homeworkProblemCounter}]{ % Makes a new environment called homeworkProblem which takes 1 argument (custom name) but the default is "Problem #"
\stepcounter{homeworkProblemCounter} % Increase counter for number of problems
\renewcommand{\homeworkProblemName}{#1} % Assign \homeworkProblemName the name of the problem
\section{\homeworkProblemName} % Make a section in the document with the custom problem count
\enterProblemHeader{\homeworkProblemName} % Header and footer within the environment
}{
\exitProblemHeader{\homeworkProblemName} % Header and footer after the environment
}

\newcommand{\problemAnswer}[1]{ % Defines the problem answer command with the content as the only argument
\noindent\framebox[\columnwidth][c]{\begin{minipage}{0.98\columnwidth}#1\end{minipage}} % Makes the box around the problem answer and puts the content inside
}

\newcommand{\homeworkSectionName}{}
\newenvironment{homeworkSection}[1]{ % New environment for sections within homework problems, takes 1 argument - the name of the section
\renewcommand{\homeworkSectionName}{#1} % Assign \homeworkSectionName to the name of the section from the environment argument
\subsection{\homeworkSectionName} % Make a subsection with the custom name of the subsection
\enterProblemHeader{\homeworkProblemName\ [\homeworkSectionName]} % Header and footer within the environment
}{
\enterProblemHeader{\homeworkProblemName} % Header and footer after the environment
}
   
%----------------------------------------------------------------------------------------
%       NAME AND CLASS SECTION
%----------------------------------------------------------------------------------------

\newcommand{\hmwkTitle}{Assignment\ \# 2} % Assignment title
\newcommand{\hmwkDueDate}{Friday,\ October\ 2,\ 2015} % Due date
\newcommand{\hmwkClass}{CSCI-567} % Course/class
\newcommand{\hmwkClassTime}{} % Class/lecture time
\newcommand{\hmwkAuthorName}{Saket Choudhary} % Your name
\newcommand{\hmwkAuthorID}{2170058637} % Teacher/lecturer
%----------------------------------------------------------------------------------------
%       TITLE PAGE
%----------------------------------------------------------------------------------------

\title{
\vspace{2in}
\textmd{\textbf{\hmwkClass:\ \hmwkTitle}}\\
\normalsize\vspace{0.1in}\small{Due\ on\ \hmwkDueDate}\\
\vspace{0.1in}\large{\textit{\hmwkClassTime}}
\vspace{3in}
}

\author{\textbf{\hmwkAuthorName} \\
        \textbf{\hmwkAuthorID}
        }
\date{} % Insert date here if you want it to appear below your name

%----------------------------------------------------------------------------------------

\begin{document}

\maketitle

%----------------------------------------------------------------------------------------
%       TABLE OF CONTENTS
%----------------------------------------------------------------------------------------

%\setcounter{tocdepth}{1} % Uncomment this line if you don't want subsections listed in the ToC

\newpage
\tableofcontents
\newpage
 \listoftodos

%----------------------------------------------------------------------------------------
%       PROBLEM 2
%----------------------------------------------------------------------------------------

\begin{homeworkProblem}[Problem 1] % Custom section title


\begin{homeworkSection}{\homeworkProblemName: ~(a)}
\problemAnswer{ % Answer
Linear regression assumes uncertainty in th measurement of dependent variables($X$). $Y$ being an independent variable is assumed to be the 'true' value that can be measured with ultimate precision. Any model that relates the independent variable to dependent will assume some kind of modelling error which in case of linear regression is often taken as Gaussian random variable . Additively the 'noise' and the dependent variable predict the independent. 

\todo{Fix this answeer. it is verbose}

}
\end{homeworkSection}


\begin{homeworkSection}{\homeworkProblemName: ~(b)}
	\problemAnswer{ % Answer
		In order to make linear regression robust to outliers, a n\"{a}ive solution will choose "absolute deviation"($L1$ norm) over "squared error"($L2$ norm) as the criterion for loss function. The reason this might work out in most cases(especially when the outliers belong to a non normal distribution) is that "squared error" will blow up errors when they are large. Thus $L2$ normn will give more weight to large residuals, while the $L1$ norm gives equal weights to all  residuals.
	}
\end{homeworkSection}


\begin{homeworkSection}{\homeworkProblemName: ~(c)}
	\problemAnswer{ % Answer
		A quick way to realise this is to consider the scale. Say one of the dependent variables is 'time'. Rescaling time from hours to seconds will also rescale its coefficient, but the importance remains the same!
	}
\end{homeworkSection}

\begin{homeworkSection}{\homeworkProblemName: ~(d)}
	\problemAnswer{ % Answer
		If the dependent variables are perfect linear combination, the matrix $XX^T$ will be non invertible.
	}
\end{homeworkSection}

\begin{homeworkSection}{\homeworkProblemName: ~(e)}
	\problemAnswer{ % Answer
		A simple solution would be to start from '0000'(length being equal to the number of levels of the categorical variables) for one of the categories.
	}
\end{homeworkSection}

\begin{homeworkSection}{\homeworkProblemName: ~(f)}
	\problemAnswer{ % Answer
		If the independent variables are highly correlated, the coefficients might still be entirely different. So if a feature 
	}
\end{homeworkSection}

\begin{homeworkSection}{\homeworkProblemName: ~(g)}
	\problemAnswer{ % Answer
		Using a posterior probability cutoff of 0.5 in linear regression is not same as 0.5 for logistic. A 0.5 rehsold on logistic guarantees that the point all points lying to the right belong to one class. However for a regression problem, this is not true, because the predicted value of $y$ is an 'intepolated or extraplolated' 
		In any case, logistic regression is a better choice
		\todo{Not so clear on this, the choice for logistic is tied to the first part answer}
	}
\end{homeworkSection}

\begin{homeworkSection}{\homeworkProblemName: ~(h)}
	\problemAnswer{ % Answer
		When the number of variables exceed the number of samples, the system is undetermined. And yes, it can be solved by simply obtaining psuedo-inverse of $X$ which is always defined.
	}
\end{homeworkSection}

\end{homeworkProblem}

\begin{homeworkProblem}[Problem 2]
	\problemAnswer{
		Given
		}
\end{homeworkProblem}

\end{document}
