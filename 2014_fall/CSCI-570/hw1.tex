%%%%%%%%%%%%%%%%%%%%%%%%%%%%%%%%%%%%%%%%%
% Structured General Purpose Assignment
% LaTeX Template
%
% This template has been downloaded from:
% http://www.latextemplates.com
%
% Original author:
% Ted Pavlic (http://www.tedpavlic.com)
%
% Note:
% The \lipsum[#] commands throughout this template generate dummy text
% to fill the template out. These commands should all be removed when 
% writing assignment content.
%
%%%%%%%%%%%%%%%%%%%%%%%%%%%%%%%%%%%%%%%%%

%----------------------------------------------------------------------------------------
%	PACKAGES AND OTHER DOCUMENT CONFIGURATIONS
%----------------------------------------------------------------------------------------

\documentclass{article}

\usepackage{fancyhdr} % Required for custom headers
\usepackage{lastpage} % Required to determine the last page for the footer
\usepackage{extramarks} % Required for headers and footers
\usepackage{graphicx} % Required to insert images
\usepackage{lipsum} % Used for inserting dummy 'Lorem ipsum' text into the template

\usepackage{amsmath}
%\usepackage[]{algorithm2e}
%\usepackage{algpseudocode}
\usepackage{verbatim}
%\usepackage{algorithm}
%\usepackage[noend]{algpseudocode}
\usepackage[]{algorithm2e}


% Margins
\topmargin=-0.45in
\evensidemargin=0in
\oddsidemargin=0in
\textwidth=6.5in
\textheight=9.0in
\headsep=0.25in 

\linespread{1.1} % Line spacing

% Set up the header and footer
\pagestyle{fancy}
\lhead{\hmwkAuthorName} % Top left header
\chead{\hmwkClass\ : \hmwkTitle} % Top center header
\rhead{\firstxmark} % Top right header
\lfoot{\lastxmark} % Bottom left footer
\cfoot{} % Bottom center footer
\rfoot{Page\ \thepage\ of\ \pageref{LastPage}} % Bottom right footer
\renewcommand\headrulewidth{0.4pt} % Size of the header rule
\renewcommand\footrulewidth{0.4pt} % Size of the footer rule

\setlength\parindent{0pt} % Removes all indentation from paragraphs

%----------------------------------------------------------------------------------------
%	DOCUMENT STRUCTURE COMMANDS
%	Skip this unless you know what you're doing
%----------------------------------------------------------------------------------------

% Header and footer for when a page split occurs within a problem environment
\newcommand{\enterProblemHeader}[1]{
\nobreak\extramarks{#1}{#1 continued on next page\ldots}\nobreak
\nobreak\extramarks{#1 (continued)}{#1 continued on next page\ldots}\nobreak
}

% Header and footer for when a page split occurs between problem environments
\newcommand{\exitProblemHeader}[1]{
\nobreak\extramarks{#1 (continued)}{#1 continued on next page\ldots}\nobreak
\nobreak\extramarks{#1}{}\nobreak
}

\setcounter{secnumdepth}{0} % Removes default section numbers
\newcounter{homeworkProblemCounter} % Creates a counter to keep track of the number of problems

\newcommand{\homeworkProblemName}{}
\newenvironment{homeworkProblem}[1][Problem \arabic{homeworkProblemCounter}]{ % Makes a new environment called homeworkProblem which takes 1 argument (custom name) but the default is "Problem #"
\stepcounter{homeworkProblemCounter} % Increase counter for number of problems
\renewcommand{\homeworkProblemName}{#1} % Assign \homeworkProblemName the name of the problem
\section{\homeworkProblemName} % Make a section in the document with the custom problem count
\enterProblemHeader{\homeworkProblemName} % Header and footer within the environment
}{
\exitProblemHeader{\homeworkProblemName} % Header and footer after the environment
}

\newcommand{\problemAnswer}[1]{ % Defines the problem answer command with the content as the only argument
\noindent\framebox[\columnwidth][c]{\begin{minipage}{0.98\columnwidth}#1\end{minipage}} % Makes the box around the problem answer and puts the content inside
}

\newcommand{\homeworkSectionName}{}
\newenvironment{homeworkSection}[1]{ % New environment for sections within homework problems, takes 1 argument - the name of the section
\renewcommand{\homeworkSectionName}{#1} % Assign \homeworkSectionName to the name of the section from the environment argument
\subsection{\homeworkSectionName} % Make a subsection with the custom name of the subsection
\enterProblemHeader{\homeworkProblemName\ [\homeworkSectionName]} % Header and footer within the environment
}{
\enterProblemHeader{\homeworkProblemName} % Header and footer after the environment
}
   
%----------------------------------------------------------------------------------------
%	NAME AND CLASS SECTION
%----------------------------------------------------------------------------------------

\newcommand{\hmwkTitle}{Homework\ \# 1 } % Assignment title
\newcommand{\hmwkDueDate}{Friday,\ September\ 5 ,\ 2014} % Due date
\newcommand{\hmwkClass}{CSCI-570} % Course/class
\newcommand{\hmwkAuthorName}{Saket Choudhary} % Your name
\newcommand{\hmwkAuthorID}{2170058637} % Teacher/lecturer
\newcommand{\hmwkAuthorEmail}{skchoudh@usc.edu} % Teacher/lecturer
%----------------------------------------------------------------------------------------
%	TITLE PAGE
%----------------------------------------------------------------------------------------

\title{
\vspace{2in}
\textmd{\textbf{\hmwkClass:\ \hmwkTitle}}\\
\normalsize\vspace{0.1in}\small{Due\ on\ \hmwkDueDate}\\
%\vspace{0.1in}\large{\textit{\hmwkClassTime}}
\vspace{3in}
}

\author{\textbf{\hmwkAuthorName} \\
	\textbf{\hmwkAuthorEmail}\\
	\textbf{\hmwkAuthorID}
	}
\date{} % Insert date here if you want it to appear below your name

%----------------------------------------------------------------------------------------

\begin{document}

\maketitle

%----------------------------------------------------------------------------------------
%	TABLE OF CONTENTS
%----------------------------------------------------------------------------------------

%\setcounter{tocdepth}{1} % Uncomment this line if you don't want subsections listed in the ToC

\newpage
\tableofcontents
\newpage


%----------------------------------------------------------------------------------------
%	PROBLEM 2
%----------------------------------------------------------------------------------------

\begin{homeworkProblem}[HW1] % Custom section title


\begin{homeworkSection}{(2: Ch\#1 Ex\#1)} 
\problemAnswer{
		\textbf{False}
		Consider the following example: \\
		$m$ ranks women as: $rank_w > rank_{w'}$ \\
		$m'$ ranks women as: $rank_w' > rank_w$ \\
		$w$ ranks men as: $rank_{m'} > rank_m$\\
		$w'$ ranks men as: $rank_m > rank_{m'}$\\
		
		In such a case there is no possible stable matching where the men $m$ or $m'$ can be paired up with their top preferences.  
		 
}
\end{homeworkSection}

\begin{homeworkSection}{(3: Ch\#1 Ex\#2)} %
	\problemAnswer{
			\textbf{True}
			Consider the statement to be false. In that case
			$m$ pairs up with woman $w'$ such that $w'\neq w$ and hence $w$ pairs up with $m'$ such that $m' \neq m$. Now
			preference wise $m$ is ranked at the top for $w$ and vice-versa(in on of the instances, preferences do not change with instances). So given that the pairs in this instance are $(m,w')$ and $(m',w)$ man $m$ would have preferred $w'$ over $w$ and $w$ would have preferred $m'$ over $m$. But this contradicts the fact that $m$ and $w$ are at the top of each other's list. 
			  
		}
	
\end{homeworkSection}	

\begin{homeworkSection}{(4)} %
	\problemAnswer{
		\textbf{False}
		
		Consider this case: \\
		
		$m$ $\longrightarrow w > w'$\\
		$m'\longrightarrow w' > w$\\
		$w \longrightarrow m' > m$\\
		$w' \longrightarrow m > m'$\\
		
		There is one possible pairing: $(m,w); (m',w')$ if men 
		propose. However if women propose $(m',w); (m, w')$ is also
		one possible configuration. So the $G-S$ algorithm still gives
		\textbf{unique} solutions if \textbf{only men} or \textbf{only women} propose. So even though the male and female versions produce two independent outputs, the output from either of them is still unique!
		
		}
\end{homeworkSection}

\begin{homeworkSection}{(5)}
	\problemAnswer{
			A stable matching will not always exist. Consider the case of simple cyclic permutations: \\
			$a \longrightarrow b > c > d$\\
			
			$b \longrightarrow c > d > a$\\
			
			$c \longrightarrow d > a > b$\\
			
			$d \longrightarrow a > b > c$\\
			
			
			They cannot settle down with their first choices since 
			$(a,b); (c,d)$ is unstable as $b$ prefers $c$ to be its room mate and $d$ prefers $a$ followed by $b$ to be its roommate.   
		}
\end{homeworkSection}

\begin{homeworkSection}{(6: Ch\#1, Ex\#3 )} % Section within problem
%\lipsum[4]\vspace{10pt} % Question

\problemAnswer{ 
	Let the n shows of $A$ have a rating given by $\{A_1,A_2,....,A_n\}$ and that of $B$ have $\{B_1, B_2, ...,B_n\}$. 
	Consider a simpler case $A_1,A_2,A_3$, and $B_1,B_2,B_3$ such that $A=\{1,3,5\}$ and $B=\{2,4,6\}$ would result in a configuration as $B,B,A$ and if however B now changes it configuration to $\{4,2,6\}$ the new configuration would be $\{B,A,B\}$ clearly violating the requirements of stability.
	

}

\end{homeworkSection}

%--------------------------------------------

\begin{homeworkSection}{(7: Ch\#1, Ex\#4 )} % Section within problem
\problemAnswer{
	Let $H$ represent the set of hospitals and $S$ represent the set of students
\begin{algorithm}[H]
\While{there is a hospital $h \epsilon H$ with atleast one spot empty that hasn't been offered to $s \epsilon E$}{
	hire the next best valid student $s'$ as in the preference list of $h$
	\eIf{$s'$ free}{
		$s'$ gets hired by $h$, number of open spots reduce by $1$	 
	}{
		\eIf{$s'$ ranks $h$ higher to its current employer $h'$}{
			$s'$ leaves $h'$\;$h$ becomes occupied; $h'$ has 1 less student 
		}{
			$s'$ remaiins with originla $h'$ and $h$ is still free and moves on to the next student on its preference list
		}
	}
}

To prove it's correctness: \\

First type of instability: \\
$s$ is assigned $h$, $s'$ is assigned no hospital and $h$ prefers $s'$: \\

Let as assume First type of staibility exists. Since $h$ prefers $s'$, it must have tried to hire $s'$ before it hired $s$, but $s'$ could have refused since there was another hospital $h''$ which was higher ranked than $s'$ But at the end $s'$ lands up with no hospital which is a CONTRADICTION.

Second Type of instability: \\
$s$ assigned $h$; $s'$ assigned $h'$, $h$ prefers $s'$ to $s$ and $s'$ prefers $h$ to $h'$.

Let us assume such an instability exists. Since $h$ prefers $s$ to $s'$, it must have tried to hire $s'$ at some point to which $s'$ refused leading to hire of $s$ or alternative $s'$ was hired but later on waas asked by another hospital $h''$ which was higher on its prefernce list leading to $s'$ leaving $h$, But now it is with $h'$ which is lower ranked than $h$ which is clearly a CONTRADICTION since $s'$ is supposed to be shifting to higher ranked hospitals(like the women in the G-S problem/solution) 



%\State $i\gets 0$
%\Else
%\If {$i+k\leq maxval$}
%\State $i\gets i+k$
%\EndIf
%\EndIf
\end{algorithm}

}

\end{homeworkSection}

%--------------------------------------------

\end{homeworkProblem}

\begin{homeworkSection}{(8)}
	\problemAnswer{After stable matching terminated man $m1$ changed his mind to marry woman $w2$ though he was already married to $w1$ 
		Original preference list : $w1 > w2$
		New preference list: $w2 > w1$
		Let the original pairing be $(m1,w1)$ and $(m2,w2)$. New pairings : $?$\\
			Case1: If $w2$ prefers $m2$ over $w1$ the change in preference of $m1$ does not matter, as even if he now asks $w2$ he would be refused since she is already engaged with a person ranked higher.\\
			
			Case2: If $w2$ prefers $m1$ over $m2$ and now $m1$ also prefers $w2$ over $m1$, then when it is $m1$'s turn he would ask $w2$ instead of $w1$ and stands a chance to get engaged initially(when $w2$ divorves $m2$) Now $m2$ is free, $w1$ is free and the G-S algorithm starts to run again with $m2$. $m2$ will not stsrt from the top of his list but should make an efficient choice to start from where he was left off by $w2$, because he is going to go further down the ladder(he would be rejected again by all women who were above $w2$ in his list even if he chososes to ask them again) so he 'initiates' the 'G-S' by asking a woman $w'$ who was ranked just below to $w2$. The procedure then otherwise contiues like normal G-S, though $w1$ is free and would get engaged as soon as $m2$ or any other person proposes her. 
		}
\end{homeworkSection}

\end{document}
