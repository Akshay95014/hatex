%%%%%%%%%%%%%%%%%%%%%%%%%%%%%%%%%%%%%%%%%
% Structured General Purpose Assignment
% LaTeX Template
%
% This template has been downloaded from:
% http://www.latextemplates.com
%
% Original author:
% Ted Pavlic (http://www.tedpavlic.com)
%
% Note:
% The \lipsum[#] commands throughout this template generate dummy text
% to fill the template out. These commands should all be removed when 
% writing assignment content.
%
%%%%%%%%%%%%%%%%%%%%%%%%%%%%%%%%%%%%%%%%%

%----------------------------------------------------------------------------------------
%	PACKAGES AND OTHER DOCUMENT CONFIGURATIONS
%----------------------------------------------------------------------------------------

\documentclass{article}

\usepackage{fancyhdr} % Required for custom headers
\usepackage{lastpage} % Required to determine the last page for the footer
\usepackage{extramarks} % Required for headers and footers
\usepackage{graphicx} % Required to insert images
\usepackage{lipsum} % Used for inserting dummy 'Lorem ipsum' text into the template

\usepackage{amsmath}
%\usepackage[]{algorithm2e}
%\usepackage{algpseudocode}
\usepackage{verbatim}
%\usepackage{algorithm}
%\usepackage[noend]{algpseudocode}
\usepackage[]{algorithm2e}


% Margins
\topmargin=-0.45in
\evensidemargin=0in
\oddsidemargin=0in
\textwidth=6.5in
\textheight=9.0in
\headsep=0.25in 

\linespread{1.1} % Line spacing

% Set up the header and footer
\pagestyle{fancy}
\lhead{\hmwkAuthorName} % Top left header
\chead{\hmwkClass\ : \hmwkTitle} % Top center header
\rhead{\firstxmark} % Top right header
\lfoot{\lastxmark} % Bottom left footer
\cfoot{} % Bottom center footer
\rfoot{Page\ \thepage\ of\ \pageref{LastPage}} % Bottom right footer
\renewcommand\headrulewidth{0.4pt} % Size of the header rule
\renewcommand\footrulewidth{0.4pt} % Size of the footer rule

\setlength\parindent{0pt} % Removes all indentation from paragraphs

%----------------------------------------------------------------------------------------
%	DOCUMENT STRUCTURE COMMANDS
%	Skip this unless you know what you're doing
%----------------------------------------------------------------------------------------

% Header and footer for when a page split occurs within a problem environment
\newcommand{\enterProblemHeader}[1]{
\nobreak\extramarks{#1}{#1 continued on next page\ldots}\nobreak
\nobreak\extramarks{#1 (continued)}{#1 continued on next page\ldots}\nobreak
}

% Header and footer for when a page split occurs between problem environments
\newcommand{\exitProblemHeader}[1]{
\nobreak\extramarks{#1 (continued)}{#1 continued on next page\ldots}\nobreak
\nobreak\extramarks{#1}{}\nobreak
}

\setcounter{secnumdepth}{0} % Removes default section numbers
\newcounter{homeworkProblemCounter} % Creates a counter to keep track of the number of problems

\newcommand{\homeworkProblemName}{}
\newenvironment{homeworkProblem}[1][Problem \arabic{homeworkProblemCounter}]{ % Makes a new environment called homeworkProblem which takes 1 argument (custom name) but the default is "Problem #"
\stepcounter{homeworkProblemCounter} % Increase counter for number of problems
\renewcommand{\homeworkProblemName}{#1} % Assign \homeworkProblemName the name of the problem
\section{\homeworkProblemName} % Make a section in the document with the custom problem count
\enterProblemHeader{\homeworkProblemName} % Header and footer within the environment
}{
\exitProblemHeader{\homeworkProblemName} % Header and footer after the environment
}

\newcommand{\problemAnswer}[1]{ % Defines the problem answer command with the content as the only argument
\noindent\framebox[\columnwidth][c]{\begin{minipage}{0.98\columnwidth}#1\end{minipage}} % Makes the box around the problem answer and puts the content inside
}

\newcommand{\homeworkSectionName}{}
\newenvironment{homeworkSection}[1]{ % New environment for sections within homework problems, takes 1 argument - the name of the section
\renewcommand{\homeworkSectionName}{#1} % Assign \homeworkSectionName to the name of the section from the environment argument
\subsection{\homeworkSectionName} % Make a subsection with the custom name of the subsection
\enterProblemHeader{\homeworkProblemName\ [\homeworkSectionName]} % Header and footer within the environment
}{
\enterProblemHeader{\homeworkProblemName} % Header and footer after the environment
}
   
%----------------------------------------------------------------------------------------
%	NAME AND CLASS SECTION
%----------------------------------------------------------------------------------------

\newcommand{\hmwkTitle}{Homework\ \# 2 } % Assignment title
\newcommand{\hmwkDueDate}{Friday,\ September\ 12 ,\ 2014} % Due date
\newcommand{\hmwkClass}{CSCI-570} % Course/class
\newcommand{\hmwkAuthorName}{Saket Choudhary} % Your name
\newcommand{\hmwkAuthorID}{2170058637} % Teacher/lecturer
\newcommand{\hmwkAuthorEmail}{skchoudh@usc.edu} % Teacher/lecturer
%----------------------------------------------------------------------------------------
%	TITLE PAGE
%----------------------------------------------------------------------------------------

\title{
\vspace{2in}
\textmd{\textbf{\hmwkClass:\ \hmwkTitle}}\\
\normalsize\vspace{0.1in}\small{Due\ on\ \hmwkDueDate}\\
%\vspace{0.1in}\large{\textit{\hmwkClassTime}}
\vspace{3in}
}

\author{\textbf{\hmwkAuthorName} \\
	\textbf{\hmwkAuthorEmail}\\
	\textbf{\hmwkAuthorID}
	}
\date{} % Insert date here if you want it to appear below your name

%----------------------------------------------------------------------------------------

\begin{document}

\maketitle

%----------------------------------------------------------------------------------------
%	TABLE OF CONTENTS
%----------------------------------------------------------------------------------------

%\setcounter{tocdepth}{1} % Uncomment this line if you don't want subsections listed in the ToC

\newpage
\tableofcontents
\newpage


%----------------------------------------------------------------------------------------
%	PROBLEM 2
%----------------------------------------------------------------------------------------

\begin{homeworkProblem}[HW1] % Custom section title


\begin{homeworkSection}{(2: Ch\#2 Ex\#3)} 
\problemAnswer{
	\textbf{Part (a)} $n^2$
	
	Doubling the input size make it slower by $\frac{(2n)^2}{n^2} = 4$
	
	Consider increasing input size by 1: $ \frac{(n+1)^2}{n^2} = \frac{n^2+2n+1}{n^2} = 1 + \frac{1}{n} + \frac{1}{n^2}$
	
	For $\lim_{n\to\infty}$, $1 + \frac{1}{n} + \frac{1}{n^2} = 1$, Thus the algorithm with $n+1$ input is as slow as with input size $n$ for $n\to \infty$.
	
	
	\textbf{Part (b): } $n^3$
	
	Doubling the input size: $\frac{(2n)^3}{n^3} = 8$, thus it is 8 times slower.
	
	Increasing the input size by 1: $\frac{(n+1)^3}{n^3} = \frac{n^3+3n^2+3n+1}{n^3} = 1 + \frac{3}{n} + \frac{3}{n^2} + \frac{1}{n^3} $.
	
	For $\lim_{n\to\infty}$, $1 + \frac{3}{n} + \frac{3}{n^2} + \frac{1}{n^3} = 1$.
	
	\textbf{Part (c):} This is similar to Part(a), since the factor of 100 is common. The solution is exactly similar to Part(a)
	
	\textbf{Part (d): } $n log n$
	
	Doubling the input size: $\frac{2n log 2n}{n log n} = \frac{2 log 2n }{log n}$. For  $\lim_{n\to\infty}$, this would blow up.
	
	Increasing the input size by 1: $\frac{n+1 log (n+1)}{n log n} $.
	
	For $\lim_{n\to\infty}$, $\frac{n+1 log (n+1)}{n log n}  = 1$.
	
	Hence input with $n+1$ is as slow as $n$ $\lim_{n\to\infty}$
	
	\textbf{Part (e):} $2^n$
	
	Doubling the input size: $\frac{2^{2n}}{2^n} = 2^n$, which blows up as $\lim_{n\to\infty}$.
	
	Increasing the input size by 1 : $\frac{2^(n+1)}{2^n} = 2$. 
	
	Thus increasing the input size by 1 causes it to be 2 times slower.
	
	
	 
}
\end{homeworkSection}

\begin{homeworkSection}{(3: Ch\#2 Ex\#4)} %
	\problemAnswer{
			  \textbf{Given:} Operating speed = $10^10$ operations per second. 
			  
			  \textbf{To Find: }Maximum possible n, for 3600s operation
			  
			  \textbf{Part (a): } $n^2$
			  
			  $n^2 = 36 * 10^12$ $\implies$ $n = 6 * 10^6$
			  
			  \textbf{Part (b):} $n^3$
			  
			  $n^3 = 36*10^12$ $\implies$ $ n= (36)^0.333 * 10^4 $
			  
			  \textbf{Part (c):} $100n^2$
			  
			  $100n^2 = 36 * 10^12 \implies n = 6*10^5$
			  
			  \textbf{Part (d):} $n log n$
			  
			  $log (n^n) = 36*10^12$ %TODO
			  
			  \textbf{Part (e):} $2^{2^n}$
			  
			  $2^{2^n} = 36 * 10^12 \implies n = log_2(log_2(36*10^12))$
			  
			   
		}
	
\end{homeworkSection}	

\begin{homeworkSection}{(4: Ch\#2 Ex\#5)} %
	\problemAnswer{
		$f1=n^2.5, f2=\sqrt{2n}, f3=n+10, f4=10^n, f5=100^n, f6=n^2logn$
		
		Consider the square of $f_i$: \\
		
		$f1'= n^5; f2 = 2n; f3 = (n+10)^2; f4=10^{2n}; f5=100^{2n};
		f6 = n^4 (log n)^2$
		
		Since exponentials always grow faster than polynomials, in the order of running time complexity higher to lower:
		
		$100^2n > 10^2n > n^5 > n^4 (log n)^2 > (n+2)^2 > n2n$

		Thus:
		
		$ f5 > f4 > f1 > f6 > f3 > f2 $
		}
\end{homeworkSection}

\begin{homeworkSection}{(5: Ch\#2 Ex\#6)}
	\problemAnswer{
		
		}
\end{homeworkSection}

\begin{homeworkSection}{(6: Ch\#3, Ex\#2 )} % Section within problem
%\lipsum[4]\vspace{10pt} % Question

\problemAnswer{ 

	

}

\end{homeworkSection}

%--------------------------------------------

\begin{homeworkSection}{(7: Ch\#3, Ex\#6 )} % Section within problem
\problemAnswer{


}

\end{homeworkSection}

%--------------------------------------------

\end{homeworkProblem}



\end{document}
