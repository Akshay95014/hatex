%%%%%%%%%%%%%%%%%%%%%%%%%%%%%%%%%%%%%%%%%
% Structured General Purpose Assignment
% LaTeX Template
%
% This template has been downloaded from:
% http://www.latextemplates.com
%
% Original author:
% Ted Pavlic (http://www.tedpavlic.com)
%
% Note:
% The \lipsum[#] commands throughout this template generate dummy text
% to fill the template out. These commands should all be removed when 
% writing assignment content.
%
%%%%%%%%%%%%%%%%%%%%%%%%%%%%%%%%%%%%%%%%%

%----------------------------------------------------------------------------------------
%	PACKAGES AND OTHER DOCUMENT CONFIGURATIONS
%----------------------------------------------------------------------------------------

\documentclass{article}

\usepackage{fancyhdr} % Required for custom headers
\usepackage{lastpage} % Required to determine the last page for the footer
\usepackage{extramarks} % Required for headers and footers
\usepackage{graphicx} % Required to insert images
\usepackage{lipsum} % Used for inserting dummy 'Lorem ipsum' text into the template

\usepackage{amsmath}
%\usepackage[]{algorithm2e}
%\usepackage{algpseudocode}
\usepackage{verbatim}
%\usepackage{algorithm}
%\usepackage[noend]{algpseudocode}
\usepackage[]{algorithm2e}


% Margins
\topmargin=-0.45in
\evensidemargin=0in
\oddsidemargin=0in
\textwidth=6.5in
\textheight=9.0in
\headsep=0.25in 

\linespread{1.1} % Line spacing

% Set up the header and footer
\pagestyle{fancy}
\lhead{\hmwkAuthorName} % Top left header
\chead{\hmwkClass\ : \hmwkTitle} % Top center header
\rhead{\firstxmark} % Top right header
\lfoot{\lastxmark} % Bottom left footer
\cfoot{} % Bottom center footer
\rfoot{Page\ \thepage\ of\ \pageref{LastPage}} % Bottom right footer
\renewcommand\headrulewidth{0.4pt} % Size of the header rule
\renewcommand\footrulewidth{0.4pt} % Size of the footer rule

\setlength\parindent{0pt} % Removes all indentation from paragraphs

%----------------------------------------------------------------------------------------
%	DOCUMENT STRUCTURE COMMANDS
%	Skip this unless you know what you're doing
%----------------------------------------------------------------------------------------

% Header and footer for when a page split occurs within a problem environment
\newcommand{\enterProblemHeader}[1]{
\nobreak\extramarks{#1}{#1 continued on next page\ldots}\nobreak
\nobreak\extramarks{#1 (continued)}{#1 continued on next page\ldots}\nobreak
}

% Header and footer for when a page split occurs between problem environments
\newcommand{\exitProblemHeader}[1]{
\nobreak\extramarks{#1 (continued)}{#1 continued on next page\ldots}\nobreak
\nobreak\extramarks{#1}{}\nobreak
}

\setcounter{secnumdepth}{0} % Removes default section numbers
\newcounter{homeworkProblemCounter} % Creates a counter to keep track of the number of problems

\newcommand{\homeworkProblemName}{}
\newenvironment{homeworkProblem}[1][Problem \arabic{homeworkProblemCounter}]{ % Makes a new environment called homeworkProblem which takes 1 argument (custom name) but the default is "Problem #"
\stepcounter{homeworkProblemCounter} % Increase counter for number of problems
\renewcommand{\homeworkProblemName}{#1} % Assign \homeworkProblemName the name of the problem
\section{\homeworkProblemName} % Make a section in the document with the custom problem count
\enterProblemHeader{\homeworkProblemName} % Header and footer within the environment
}{
\exitProblemHeader{\homeworkProblemName} % Header and footer after the environment
}

\newcommand{\problemAnswer}[1]{ % Defines the problem answer command with the content as the only argument
\noindent\framebox[\columnwidth][c]{\begin{minipage}{0.98\columnwidth}#1\end{minipage}} % Makes the box around the problem answer and puts the content inside
}

\newcommand{\homeworkSectionName}{}
\newenvironment{homeworkSection}[1]{ % New environment for sections within homework problems, takes 1 argument - the name of the section
\renewcommand{\homeworkSectionName}{#1} % Assign \homeworkSectionName to the name of the section from the environment argument
\subsection{\homeworkSectionName} % Make a subsection with the custom name of the subsection
\enterProblemHeader{\homeworkProblemName\ [\homeworkSectionName]} % Header and footer within the environment
}{
\enterProblemHeader{\homeworkProblemName} % Header and footer after the environment
}
   
%----------------------------------------------------------------------------------------
%	NAME AND CLASS SECTION
%----------------------------------------------------------------------------------------

\newcommand{\hmwkTitle}{Homework\ \# 3 } % Assignment title
\newcommand{\hmwkDueDate}{Friday,\ September\ 19 ,\ 2014} % Due date
\newcommand{\hmwkClass}{CSCI-570} % Course/class
\newcommand{\hmwkAuthorName}{Saket Choudhary} % Your name
\newcommand{\hmwkAuthorID}{2170058637} % Teacher/lecturer
\newcommand{\hmwkAuthorEmail}{skchoudh@usc.edu} % Teacher/lecturer
%----------------------------------------------------------------------------------------
%	TITLE PAGE
%----------------------------------------------------------------------------------------

\title{
\vspace{2in}
\textmd{\textbf{\hmwkClass:\ \hmwkTitle}}\\
\normalsize\vspace{0.1in}\small{Due\ on\ \hmwkDueDate}\\
%\vspace{0.1in}\large{\textit{\hmwkClassTime}}
\vspace{3in}
}

\author{\textbf{\hmwkAuthorName} \\
	\textbf{\hmwkAuthorEmail}\\
	\textbf{\hmwkAuthorID}
	}
\date{} % Insert date here if you want it to appear below your name

%----------------------------------------------------------------------------------------

\begin{document}

\maketitle

%----------------------------------------------------------------------------------------
%	TABLE OF CONTENTS
%----------------------------------------------------------------------------------------

%\setcounter{tocdepth}{1} % Uncomment this line if you don't want subsections listed in the ToC

\newpage
\tableofcontents
\newpage


%----------------------------------------------------------------------------------------
%	PROBLEM 2
%----------------------------------------------------------------------------------------

\begin{homeworkProblem}[HW3] % Custom section title


\begin{homeworkSection}{(2)} %
	\problemAnswer{
		A) The intersections can be viewed as the nodes of a directed graph. An intersection $I_i$ can be reached from $I_j$ given that there is an edge incident from $I_j$ to $I_i$ that is $ (I_j, I_i) \in E$ where E= set of edges of Graph $G(V,E)$
		
		If such a directed graph allows to reach from any point to any other point, it needs to be strongly connected implying there is a path from $I_i$ to $I_j$ and from $I_j$ to $I_i$. Checking if a path from $I_j$ exists to $I_i$ will involve reversing the edge directions in the directed graph and  checking if $I_i$ can be reached from $I_j$. So if the mayor is right, it should be possible to traverse from $I_j$ to $I_i$ with the edges inverted.
		
		Such a strongly connected directed graph can be traversed in linear time using DFS with a run time of O(n+m)	
		
		B) Since the mayor's original claim is false $\implies$ G is not strongly connected. However the focus shifts to the town hall being "strongly connected" with the rest of the nodes. In order for this to happen the node for town hall say T must be a sink of a strongly connected component. The approach would involve determining all such components containing T such that they are strongly connected. Next run a DFS in $O(n+m)$ to determine all nodes reachable from T if all these nodes belong to to same strongly connected component, it should be possible to travel from T to these and back. 
		}
\end{homeworkSection}

\begin{homeworkSection}{(3: Ch\#3 Ex\#3)}
	\problemAnswer{
	
		
		 For outputting a cycle G(if any) in G, we perform a BFS and keep track of nodes visited in an array. Initially $visitied[i] = 0 \forall i \in V$ and change the status of this array as we traverse the nodes checking $if\ visited[i] = 1;\ then\ cycle\ exists$ else we follow the following algorithm for creating a topological ordering.
		 
		 BSF traversal takes O(n+m) and so does topological ordering
		 % we make use of two pointers starting from any two connected nodes and run a BFS using these two nodes, At each step pointer 1 moves via 1 edge and pointer 2 by two edges. We maintain two separate arrays to track if any node has been visited by  If any of these pointers ever 
		
		}
\end{homeworkSection}

\begin{homeworkSection}{(4)} %
	\problemAnswer{
		Distance between neighboring gas stations is $p$ miles. Let's stop at stations $\{s_1,s_2,....s_k\} $. To minimise the number of stops, we need to stop only if the distance to the next station(s) is larger than what can be covered with the petrol in the car at present. Hence we should avoid stopping at $I_i$ if distance to be travelled till $I_{i+1}, I_{i+2}...$ is less than p. 
		
		Consider $\{s'_1, s'_2, s'_3,....s'_k\} $ to be some other optimal solution. 
		If $s_1$ comes before $s'_1$ that means greedy worked at step 1. If it does not then $s'_1$ can be avoided and replaced with $s_1$. Now since $s_1$ and $s'_1$ both were first steps they are each at least as far from starting point as the other, so this is an allowed solution. Now travelling from $s_1$ to last point and $s'_1$ to last point can be greedily solved and thus this would be an idnducitve    process. Thus $\{s_1,s_2,....s_k\} $ is an optimal solution.
		
		The running time is O(n) as we search for all vertices till the distance does not exceed $p$ and as such there can be at worse $n-1$ stations coming up.
		\
	}
\end{homeworkSection}

\begin{homeworkSection}{(5: Ch\#4 Ex\#3)} %
	\problemAnswer{
		Consider that the greedy algorithm in use makes use of $k$ trucks  loading boxes $b_1, b_2, ...b_i$ and the optimal algorithm uses $k$ trucks to load $b_1, b_2, ... b_j$ . If the greedy algorithm stays ahead it should ensure $i \geq j$
		
		
		Consider k = 1, then the greedy algorithm fits as many boxes as the optimal solution  
		
		Then consider the case that it holds true for k = r-1, so the greedy algorithm fits $i'$ and the optimal algorithm  fits $j'$ such that $i' \geq j'$. Then for the $k=r$ truck, greedy algorithm can pack $b_{i'+1}, b_{i'+2}...b_{i}$ and the other algo packs $b_{j'+1}, b_{j'+2}, ...b_{j}$ but $ i' > j'$ and $ i > j$  so essentially greedy algo is able to pack all boxes from $b_{i'+1}...b_j$ and the other algo packs $b_{j'+1}...b_j$ but greedy can pack more since it covers box till $b_j$ as ( $ i >j$)  but can go upto $i$
		
		 

	 
	}
\end{homeworkSection}

\begin{homeworkSection}{(6)} %
	\problemAnswer{
	 Given two sets A and B, each containing n positive integers. Perform reordering maximising $\Pi_{i=1}^{n}a_i^{b_i}$.
	 
	 There are 4 ways to proceed. \\
%	 Consider: $A=1,2,3; B = 4,5,6$ \\
	 1. Larger values of $a$ raised to larger values of $b$ \\ 
	 2. Smaller values of $a$ raised to larger values of $b$ \\
	 3. Larger values of $a$ raised to smaller values of $b$\\
	 4. Smaller values of $a$ raised to smaller values of $b$ \\
	 
	 It is easy to rule out Case 4, since it would the smallest possible product.
	 The case maximising the payoff is Case 1  since the product is maximised when the individual terms are masimised which is possible if the larges number has the largest exponent.
	 So $a_1 > a_2 $ and $b_1 > b_2$ then we consider $P$=$a_1^{b_1}*a_2^{b_2}$
	 
	 Consider an alternate optimal arrangement where $a_1$ is paired with $b_2 \neq b_1$
	 Then $P' = a_1^{b_2}*a_2*{b_1}$
	 Then $\frac{P'}{P} = (\frac{a_1}{a_2})^{-b_1+ b_2} < 1$
	 This can be extended to be true for n term product and hence greedy algorithm yields optimal solution.
	 
	 The sorting is possible in $O(n log n)$ and  the assuming multiplication and exponentiation to be elementary operations, the complexity would be $O(n log n)$
	  
	 % if $a > b$ $ \implies$ $log a > log b$ $ \forall a,b >1$
	 %Consider $a^b > b^a$ $\implies$ Take  
	}
\end{homeworkSection}

\begin{homeworkSection}{(7: Ch\#4 Ex\#4)} %
	\problemAnswer{
	Given two sequence $S'$ of size m and $S$ of size n, to determine if $ S' \subset S$
	
	$S'$ can be  visualised as a DAG or more specifically as a topological ordering.
	Also treat S as a DAG. We keep two pointers, one for $S'$ and one for $S$ and perform a DFS on $S$ till we hit a element from $S'$ starting with $S'[0]$, once $S'[0]$ is found in $S$ we start a DFS again after deleting all preceding elements of $S$ now continuing till we find $S[1]$ and deleting the intermediate hits if any.
	
	}
\end{homeworkSection}

\begin{homeworkSection}{(8)} %
	\problemAnswer{
		Choose the largest denomination of (25,10,1) such that $(n-max_denomination)$ is positive
		and then keep doing this until n = 0. This is also equivalent to adding $n / max_denomination$ as the coin count and the new amount being $n mod max_denomination$
		
		
	%	If the first choice is 1 pence then $n$ can be atmost 4, since other 5 cents would habe been a viable option.
		
	%	If the first choice is 5 pence then $n$ can be between  10 and 24.
		
	%	If the first choice is 25c then the $n$ is atleast 25		
	
	Let the first choice of greedy be $n1$ value of coins of denomination $d1$. If a set $G$ represents this greedy solution for $n$ coins then $G-\{d1\}$ is a greedy solution for $n-n1$ Let another optimal solutiuon for this be  $O$, then $U \cup \{d1\}$ should contains fewer coins than $G$, that would mean a total of $n$ is possible in fewer than $d1$ (max possible) value coins which is a contradiction. Hence $G$ is the optimal solution $O$
	
	\textbf{Part b)} Consider $n=6$ and denominations = 1,3,4
	Then greedy gives : $1,1,4$ while the optimal is $3,3$
	
	
	}
\end{homeworkSection}



%--------------------------------------------

\end{homeworkProblem}



\end{document}
