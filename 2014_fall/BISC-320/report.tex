%%%%%%%%%%%%%%%%%%%%%%%%%%%%%%%%%%%%%%%%%
% Short Sectioned Assignment
% LaTeX Template
% Version 1.0 (5/5/12)
%
% This template has been downloaded from:
% http://www.LaTeXTemplates.com
%
% Original author:
% Frits Wenneker (http://www.howtotex.com)
%
% License:
% CC BY-NC-SA 3.0 (http://creativecommons.org/licenses/by-nc-sa/3.0/)
%
%%%%%%%%%%%%%%%%%%%%%%%%%%%%%%%%%%%%%%%%%

%----------------------------------------------------------------------------------------
%	PACKAGES AND OTHER DOCUMENT CONFIGURATIONS
%----------------------------------------------------------------------------------------

\documentclass[paper=a4, fontsize=11pt]{scrartcl} % A4 paper and 11pt font size

\usepackage[T1]{fontenc} % Use 8-bit encoding that has 256 glyphs
\usepackage{fourier} % Use the Adobe Utopia font for the document - comment this line to return to the LaTeX default
\usepackage[english]{babel} % English language/hyphenation
\usepackage{amsmath,amsfonts,amsthm} % Math packages

\usepackage{lipsum} % Used for inserting dummy 'Lorem ipsum' text into the template

\usepackage{sectsty} % Allows customizing section commands
\allsectionsfont{\centering \normalfont\scshape} % Make all sections centered, the default font and small caps

\usepackage{fancyhdr} % Custom headers and footers
\pagestyle{fancyplain} % Makes all pages in the document conform to the custom headers and footers
\fancyhead{} % No page header - if you want one, create it in the same way as the footers below
\fancyfoot[L]{} % Empty left footer
\fancyfoot[C]{} % Empty center footer
\fancyfoot[R]{\thepage} % Page numbering for right footer
\renewcommand{\headrulewidth}{0pt} % Remove header underlines
\renewcommand{\footrulewidth}{0pt} % Remove footer underlines
\setlength{\headheight}{13.6pt} % Customize the height of the header

\numberwithin{equation}{section} % Number equations within sections (i.e. 1.1, 1.2, 2.1, 2.2 instead of 1, 2, 3, 4)
\numberwithin{figure}{section} % Number figures within sections (i.e. 1.1, 1.2, 2.1, 2.2 instead of 1, 2, 3, 4)
\numberwithin{table}{section} % Number tables within sections (i.e. 1.1, 1.2, 2.1, 2.2 instead of 1, 2, 3, 4)

\setlength\parindent{0pt} % Removes all indentation from paragraphs - comment this line for an assignment with lots of text

%----------------------------------------------------------------------------------------
%	TITLE SECTION
%----------------------------------------------------------------------------------------

\newcommand{\horrule}[1]{\rule{\linewidth}{#1}} % Create horizontal rule command with 1 argument of height

\title{	
\normalfont \normalsize 
\textsc{University Of Southern California} \\ [25pt] % Your university, school and/or department name(s)
\horrule{0.5pt} \\[0.4cm] % Thin top horizontal rule
\huge BISC320L: Formal Lab Report  \\ % The assignment title
\horrule{2pt} \\[0.5cm] % Thick bottom horizontal rule
}

\author{Saket Choudhary} % Your name

\date{\normalsize\today} % Today's date or a custom date
\linespread{2.0}

\begin{document}

\maketitle % Print the title

%----------------------------------------------------------------------------------------
%	PROBLEM 1
%----------------------------------------------------------------------------------------

\section{Introduction}

DNA is present in all living organisms and encodes genetic information. It is made up of four bases \textbf{A}denine, \textbf{T}hymine, \textbf{G}uanine and\textbf{ C}ytosine. These bases join each other through giving rise to an alternating sugar-phosphate backbone. C and T are pyrimidines while G and A are purines. DNA is a double stranded entity where A base pairs with T through a double bonded structure while G always pairs up with C through a triple bond.

Any study  involving the DNA requires the DNA to be extracted from the cells. In this particular experiment we relied on cheek cells for extracting DNA. DNA is present in every cell. In eukaryotes it is present inside the nucleus surrounded by a nuclear membrane.  The cell itself is enclosed by a cellular membrane. The membranes are generally made of phospholipids. 

\subsection{Isolation of DNA from cheek cells}
A good source for extracting DNA in humnans is the cheek where  the cells from the lining can be easily scraped given that they are loose. In order to extract the DNA these cells, it needs to be separated from the rest of the molecules contained in the cell. The basic principle underlyting such a separation technique involves washing the cells with detergent that throws open the cell membrane and the nuclear membrane. 
Since these membranes are primarily lipids, they can easily be \textit{digested} by lysis buffers. In order to separate the DNA from the rest of the cellular components we use protease stock solution that breaks the proteins into smaller pieces. This also causes the wound DNA to unwind and separate from the proteins. The last step involved adding alcohol(Ethanol). Alcohol reduces the 	solubility of DNA causing the DNA to precipitate while the rest of the proteins and lipids become part of the solution. This becomes evident as the DNA forms a white precipitate at the bottom of the test tube.
%\url{http://www.sfponline.org/uploads/dnaextractlab.pdf}

\subsection{PCR}
The amount of DNA extracted from cells is often limited. In order to perform further experiments, this sequence needs to be enriched/amplified. P
Polymerase Chain Reaction or PCR is a technology used to make multiple copies of a sequence of DNA. It was invented by Kary Mullis in 1983. \cite{bartlett2003short}
PCR consists of 3 main steps:
\begin{enumerate}
\item Denaturing: The DNA strands are denatured by heating causing them to separate out as single strands
\item Annealing: The temperature is lowered to allow the primers to hybridize

\item Elongation: Enzymes help the hybrid strands to extend further
\end{enumerate}

\subsection{\textit{Alu} inserts}
\textit{Alu} refers to the short stretch of DNA classified as short interspersed elements (SINEs). They do not code for proteins. Though Alu insertions are known to bne associated with hereditary diseases in humans such as hemophilia and breast cancer\cite{batzer2002alu}
\subsection{PV92 locus}
PV92 is a specific Alu insertion on Chromosome 16. The locus can exhibit presence(+) or absence(-) of alleles. Thus a total of 3 genotypes are possible:
\begin{enumerate}
\item Homozygous(+/+)
\item Heterozygous(+/-)
\item Homozygous(-/-)
\end{enumerate} 
\subsection{Hardy-Weinberg Equilbrium}
According to Hardy-Weinberg principle allele and genotype frequencies remain constant over generations given the other evolutionary influences are absent. 
\subsection{Population genetics}
\subsection{Agarose gel electrophoresis}

%------------------------------------------------

\section{Materials \& Methods}

\subsection{Isolation of DNA from Cheek cells}
\subsection{Polymerase Chain Reaction Procedure}
\subsection{PCR Conditions}
\subsection{Agarose gel electrophoresis procedure}
\subsection{Hardy-Weinberg analysis of class data}

%------------------------------------------------




\section{Data \& Results}

\subsection{Image of DNA agarose gel showing results of PCR}
\subsection{My Genotype for Alu insertion}
\subsection{Observed Class Genotypic Frequencies}
\newpage
\begin{tabular}{|c|c|c|}
	
	\hline Category  & Number of Genotypes & Frequencies(\# genotypes/Total)  \\ 
	\hline  Homozygous(+/+)&  &  \\ 
	\hline  Heterozygous(+/-)&  &  \\ 
	\hline  Homozygous(-/-)&  &  \\ 
	\hline  & Total= & =1  \\ 
	\hline 
\end{tabular} 

\begin{tabular}{|c|c|c|}
	
	\hline Category  & Number & Class Allelic Frequencies  \\ 
	\hline  total (+) alleles = $p$&  &  \\ 
	\hline  Total (-) alleles = $q$ &  &  \\ 

	\hline  & Total Alleles= & =1.00  \\ 
	\hline 
\end{tabular} 
%----------------------------------------------------------------------------------------
%	PROBLEM 2
%----------------------------------------------------------------------------------------

\section{Discussion}
\subsection{What important components are in the PCR Master mix?}
\subsection{Explain why precise target DNA sequence does not get amplified until the third cycle is completed. Make Diagram :-/}
%------------------------------------------------


\section{Conclusion}

%----------------------------------------------------------------------------------------


\end{document}