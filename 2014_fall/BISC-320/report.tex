%%%%%%%%%%%%%%%%%%%%%%%%%%%%%%%%%%%%%%%%%
% Short Sectioned Assignment
% LaTeX Template
% Version 1.0 (5/5/12)
%
% This template has been downloaded from:
% http://www.LaTeXTemplates.com
%
% Original author:
% Frits Wenneker (http://www.howtotex.com)
%
% License:
% CC BY-NC-SA 3.0 (http://creativecommons.org/licenses/by-nc-sa/3.0/)
%
%%%%%%%%%%%%%%%%%%%%%%%%%%%%%%%%%%%%%%%%%

%----------------------------------------------------------------------------------------
%	PACKAGES AND OTHER DOCUMENT CONFIGURATIONS
%----------------------------------------------------------------------------------------

\documentclass[paper=a4, fontsize=11pt]{scrartcl} % A4 paper and 11pt font size

\usepackage[T1]{fontenc} % Use 8-bit encoding that has 256 glyphs
\usepackage{fourier} % Use the Adobe Utopia font for the document - comment this line to return to the LaTeX default
\usepackage[english]{babel} % English language/hyphenation
\usepackage{amsmath,amsfonts,amsthm} % Math packages

\usepackage{lipsum} % Used for inserting dummy 'Lorem ipsum' text into the template

\usepackage{sectsty} % Allows customizing section commands
\allsectionsfont{\centering \normalfont\scshape} % Make all sections centered, the default font and small caps

\usepackage{fancyhdr} % Custom headers and footers
\pagestyle{fancyplain} % Makes all pages in the document conform to the custom headers and footers
\fancyhead{} % No page header - if you want one, create it in the same way as the footers below
\fancyfoot[L]{} % Empty left footer
\fancyfoot[C]{} % Empty center footer
\fancyfoot[R]{\thepage} % Page numbering for right footer
\renewcommand{\headrulewidth}{0pt} % Remove header underlines
\renewcommand{\footrulewidth}{0pt} % Remove footer underlines
\setlength{\headheight}{13.6pt} % Customize the height of the header

\numberwithin{equation}{section} % Number equations within sections (i.e. 1.1, 1.2, 2.1, 2.2 instead of 1, 2, 3, 4)
\numberwithin{figure}{section} % Number figures within sections (i.e. 1.1, 1.2, 2.1, 2.2 instead of 1, 2, 3, 4)
\numberwithin{table}{section} % Number tables within sections (i.e. 1.1, 1.2, 2.1, 2.2 instead of 1, 2, 3, 4)

\setlength\parindent{0pt} % Removes all indentation from paragraphs - comment this line for an assignment with lots of text

%----------------------------------------------------------------------------------------
%	TITLE SECTION
%----------------------------------------------------------------------------------------

\newcommand{\horrule}[1]{\rule{\linewidth}{#1}} % Create horizontal rule command with 1 argument of height

\title{	
\normalfont \normalsize 
\textsc{University Of Southern California} \\ [25pt] % Your university, school and/or department name(s)
\horrule{0.5pt} \\[0.4cm] % Thin top horizontal rule
\huge BISC320L: Formal Lab Report  \\ % The assignment title
\horrule{2pt} \\[0.5cm] % Thick bottom horizontal rule
}

\author{Saket Choudhary} % Your name

\date{\normalsize\today} % Today's date or a custom date
\linespread{2.0}

\begin{document}

\maketitle % Print the title

%----------------------------------------------------------------------------------------
%	PROBLEM 1
%----------------------------------------------------------------------------------------

\section{Introduction}
\subsection{Isolation of DNA from cheek cells}
\subsection{PCR}
\subsection{\textit{Alu} inserts}
\subsection{PV92 locus}
\subsection{Hardy-Weinberg Equilbrium}
\subsection{Population genetics}
\subsection{Agarose gel electrophoresis}

%------------------------------------------------

\section{Materials \& Methods}

\subsection{Isolation of DNA from Cheek cells}
\subsection{Polymerase Chain Reaction Procedure}
\subsection{PCR Conditions}
\subsection{Agarose gel electrophoresis procedure}
\subsection{Hardy-Weinberg analysis of class data}

%------------------------------------------------




\section{Data \& Results}

\subsection{Image of DNA agarose gel showing results of PCR}
\subsection{My Genotype for Alu insertion}
\subsection{Observed Class Genotypic Frequencies}
\newpage
\begin{tabular}{|c|c|c|}
	
	\hline Category  & Number of Genotypes & Frequencies(\# genotypes/Total)  \\ 
	\hline  Homozygous(+/+)&  &  \\ 
	\hline  Heterozygous(+/-)&  &  \\ 
	\hline  Homozygous(-/-)&  &  \\ 
	\hline  & Total= & =1  \\ 
	\hline 
\end{tabular} 

\begin{tabular}{|c|c|c|}
	
	\hline Category  & Number & Class Allelic Frequencies  \\ 
	\hline  total (+) alleles = $p$&  &  \\ 
	\hline  Total (-) alleles = $q$ &  &  \\ 

	\hline  & Total Alleles= & =1.00  \\ 
	\hline 
\end{tabular} 
%----------------------------------------------------------------------------------------
%	PROBLEM 2
%----------------------------------------------------------------------------------------

\section{Discussion}
\subsection{What important components are in the PCR Master mix?}
\subsection{Explain why precise target DNA sequence does not get amplified until the third cycle is completed. Make Diagram :-/}
%------------------------------------------------


\section{Conclusion}

%----------------------------------------------------------------------------------------

\end{document}