%%%%%%%%%%%%%%%%%%%%%%%%%%%%%%%%%%%%%%%%%
% Structured General Purpose Assignment
% LaTeX Template
%
% This template has been downloaded from:
% http://www.latextemplates.com
%
% Original author:
% Ted Pavlic (http://www.tedpavlic.com)
%
% Note:
% The \lipsum[#] commands throughout this template generate dummy text
% to fill the template out. These commands should all be removed when 
% writing assignment content.
%
%%%%%%%%%%%%%%%%%%%%%%%%%%%%%%%%%%%%%%%%%

%----------------------------------------------------------------------------------------
%	PACKAGES AND OTHER DOCUMENT CONFIGURATIONS
%----------------------------------------------------------------------------------------

\documentclass{article}

\usepackage{fancyhdr} % Required for custom headers
\usepackage{lastpage} % Required to determine the last page for the footer
\usepackage{extramarks} % Required for headers and footers
\usepackage{graphicx} % Required to insert images
\usepackage{lipsum} % Used for inserting dummy 'Lorem ipsum' text into the template

\usepackage{amsmath}
%\usepackage{multline}

% Margins
\topmargin=-0.45in
\evensidemargin=0in
\oddsidemargin=0in
\textwidth=6.5in
\textheight=9.0in
\headsep=0.25in 

\linespread{1.1} % Line spacing

% Set up the header and footer
\pagestyle{fancy}
\lhead{\hmwkAuthorName} % Top left header
\chead{\hmwkClass\ : \hmwkTitle} % Top center header
\rhead{\firstxmark} % Top right header
\lfoot{\lastxmark} % Bottom left footer
\cfoot{} % Bottom center footer
\rfoot{Page\ \thepage\ of\ \pageref{LastPage}} % Bottom right footer
\renewcommand\headrulewidth{0.4pt} % Size of the header rule
\renewcommand\footrulewidth{0.4pt} % Size of the footer rule

\setlength\parindent{0pt} % Removes all indentation from paragraphs

%----------------------------------------------------------------------------------------
%	DOCUMENT STRUCTURE COMMANDS
%	Skip this unless you know what you're doing
%----------------------------------------------------------------------------------------

% Header and footer for when a page split occurs within a problem environment
\newcommand{\enterProblemHeader}[1]{
\nobreak\extramarks{#1}{#1 continued on next page\ldots}\nobreak
\nobreak\extramarks{#1 (continued)}{#1 continued on next page\ldots}\nobreak
}

% Header and footer for when a page split occurs between problem environments
\newcommand{\exitProblemHeader}[1]{
\nobreak\extramarks{#1 (continued)}{#1 continued on next page\ldots}\nobreak
\nobreak\extramarks{#1}{}\nobreak
}

\setcounter{secnumdepth}{0} % Removes default section numbers
\newcounter{homeworkProblemCounter} % Creates a counter to keep track of the number of problems

\newcommand{\homeworkProblemName}{}
\newenvironment{homeworkProblem}[1][Problem \arabic{homeworkProblemCounter}]{ % Makes a new environment called homeworkProblem which takes 1 argument (custom name) but the default is "Problem #"
\stepcounter{homeworkProblemCounter} % Increase counter for number of problems
\renewcommand{\homeworkProblemName}{#1} % Assign \homeworkProblemName the name of the problem
\section{\homeworkProblemName} % Make a section in the document with the custom problem count
\enterProblemHeader{\homeworkProblemName} % Header and footer within the environment
}{
\exitProblemHeader{\homeworkProblemName} % Header and footer after the environment
}

\newcommand{\problemAnswer}[1]{ % Defines the problem answer command with the content as the only argument
\noindent\framebox[\columnwidth][c]{\begin{minipage}{0.98\columnwidth}#1\end{minipage}} % Makes the box around the problem answer and puts the content inside
}

\newcommand{\homeworkSectionName}{}
\newenvironment{homeworkSection}[1]{ % New environment for sections within homework problems, takes 1 argument - the name of the section
\renewcommand{\homeworkSectionName}{#1} % Assign \homeworkSectionName to the name of the section from the environment argument
\subsection{\homeworkSectionName} % Make a subsection with the custom name of the subsection
\enterProblemHeader{\homeworkProblemName\ [\homeworkSectionName]} % Header and footer within the environment
}{
\enterProblemHeader{\homeworkProblemName} % Header and footer after the environment
}
   
%----------------------------------------------------------------------------------------
%	NAME AND CLASS SECTION
%----------------------------------------------------------------------------------------

\newcommand{\hmwkTitle}{Homework\ \# 3 } % Assignment title
\newcommand{\hmwkDueDate}{Friday,\ September \ 12,\ 2014} % Due date
\newcommand{\hmwkClass}{MATH-505A} % Course/class
\newcommand{\hmwkClassTime}{10:30am} % Class/lecture time
\newcommand{\hmwkAuthorName}{Saket Choudhary} % Your name
\newcommand{\hmwkAuthorID}{2170058637} % Teacher/lecturer
%----------------------------------------------------------------------------------------
%	TITLE PAGE
%----------------------------------------------------------------------------------------

\title{
\vspace{2in}
\textmd{\textbf{\hmwkClass:\ \hmwkTitle}}\\
\normalsize\vspace{0.1in}\small{Due\ on\ \hmwkDueDate}\\
%\vspace{0.1in}\large{\textit{\hmwkClassTime}}
\vspace{3in}
}

\author{\textbf{\hmwkAuthorName} \\
	\textbf{\hmwkAuthorID}
	}
\date{} % Insert date here if you want it to appear below your name

%----------------------------------------------------------------------------------------

\begin{document}

\maketitle

%----------------------------------------------------------------------------------------
%	TABLE OF CONTENTS
%----------------------------------------------------------------------------------------

%\setcounter{tocdepth}{1} % Uncomment this line if you don't want subsections listed in the ToC

\newpage
\tableofcontents
\newpage

\begin{homeworkProblem}[Exercise \# 1.7] % Custom section title
	\begin{homeworkSection}{(1)} % Section within problem
		\problemAnswer{ % Answer
			\textbf{Given:} Two roads $r1_{AB}$, $r2_{AB}$ connectinng points A and B and $s1_{BC}$, $s2_{BC}$ connecting B and C. 
			%$p(X)$ denotes the probability that the road X is blocked. Then, $p(r1_{AB}) = p(r2_{AB}) = p(s1_{AB}) = p(s2_{AB}) = p$.
			
			Let $p(AB)$ denote the probability that path between A$\longrightarrow$B is open and let $p({AB}^c)$ denote the probability that there is no open road b/w A and B.
			Alternativels $p(AB)$ denotes that road(s) between A and B are open.
			\textbf{To find:}$Y=P(AB | AC^c)$.
			
		Y is equal to the probability that road between A and B is open AND still the path between A and C is closed $\implies$ Path between B and C is closed AND between A and B is open
		
		p(AB) $=$ Path b/w A,B is open $=$ 1 - Path b/w A,B is closed $=$  $1 - p*p$
		
		Thus \begin{equation}
		\label{1c1}
		p(AB) = 1 - p^2
		\end{equation}
		Also, 
		\begin{equation}
		\label{1c2}
			p(AB) = p(BC)
		\end{equation}
		
		
		
		$p(AC^C) = $ 1 - Probability A,C is open = 1 - Probability AB is open AND BC is open. % = $1-p(AB)*p(BC)$
		Thus,
			\begin{equation}
			\label{1c3}
			p(AC^c) = 1- p(AB)p(AC) = 1-(1-p^2)^2
			\end{equation}
			
			
		\begin{equation}
		\label{1c4}
		p(AB \cap AC^C) = p(AC^C | AB)p(AB) = p(BC^C)p(AB) = p^2(1-p^2)
		\end{equation}
			
		\begin{equation}
		\label{1c5}
		p(AB|AC^c) = \frac{P(AB \cap AC^c)}{p(AC^c)} = \frac{p(AC^C | AB)p(AB)}{p(AC^c)} = \frac{p^2(1-p^2)}{1-(1-p^2)^2} 
		\end{equation}
		
		\textbf{Part 2:} Additional direct road from A to C. Find $p(AB | AC^c)$: \\
		$p(AC^c|AB) = $ Probability that A,C is closed given A,B are open = Probability A,C(direct) are closed AND B,C are closed
		
		\begin{equation}
		\label{1c6}
		p(AC^C|AB) = p*p(BC^c)p(AB)
		\end{equation}
		
		where the extra p in \ref{1c6} as compared to \ref{1c4} is because the direct path A,C should be blocked too.
		
		\begin{equation}
		\label{1c7}
				p(AC^c) =  1 - (1-p^2)^2(1-p)
		\end{equation}
		
		where the extra $(1-p)$ factor in \ref{1c7} as compared to \ref{1c3} accounts for the fact that direct path AC is open.
		
		Thus, for part 2:
		
		\begin{equation}
		\label{1c8}
		p(AB|AC^c) = \frac{p^3(1-p^2)}{1-(1-p^2)^2(1-p)}
		\end{equation}

		
		 
		}
	\end{homeworkSection}
		\begin{homeworkSection}{(2)} % Section within problem
			\problemAnswer{ % Answer
				\begin{equation}
					p(2K \cap 1A ) = \frac{\binom{4}{2} * \binom{4}{1} * \binom{52-4-4}{10}}{\binom{52}{13}}
					= \frac{24 * 44! * 13!}{10! * 52!} = 1.357*10^{-9}
			\end{equation}
				
				$p(1A | 2K) = \frac{p(1A \cap 2K)}{p(2K)}$
				
				\begin{equation}
					\label{2c2}
						p(2K ) = \frac{\binom{4}{2}*\binom{52-4}{11}}{\binom{52}{13}}
				\end{equation}

				
			}
		\end{homeworkSection}	
		
			\begin{homeworkSection}{(4)} % Section within problem
				\problemAnswer{ % Answer
				}
			\end{homeworkSection}	
			
			\begin{homeworkSection}{(5)} % Section within problem
				\problemAnswer{ % Answer
				}
			\end{homeworkSection}				
\end{homeworkProblem}

\begin{homeworkProblem}[Exercise \# 1.8] % Custom section title
	\begin{homeworkSection}{()} % Section within problem
		\problemAnswer{ % Answer
		}
	\end{homeworkSection}			
\end{homeworkProblem}

\end{document}
