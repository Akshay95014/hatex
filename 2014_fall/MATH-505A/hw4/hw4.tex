%%%%%%%%%%%%%%%%%%%%%%%%%%%%%%%%%%%%%%%%%
% Structured General Purpose Assignment
% LaTeX Template
%
% This template has been downloaded from:
% http://www.latextemplates.com
%
% Original author:
% Ted Pavlic (http://www.tedpavlic.com)
%
% Note:
% The \lipsum[#] commands throughout this template generate dummy text
% to fill the template out. These commands should all be removed when 
% writing assignment content.
%
%%%%%%%%%%%%%%%%%%%%%%%%%%%%%%%%%%%%%%%%%

%----------------------------------------------------------------------------------------
%	PACKAGES AND OTHER DOCUMENT CONFIGURATIONS
%----------------------------------------------------------------------------------------

\documentclass{article}

\usepackage{fancyhdr} % Required for custom headers
\usepackage{lastpage} % Required to determine the last page for the footer
\usepackage{extramarks} % Required for headers and footers
\usepackage{graphicx} % Required to insert images
\usepackage{latexsym}

\usepackage{lipsum} % Used for inserting dummy 'Lorem ipsum' text into the template

\usepackage{amsmath}
%\usepackage{multline}

% Margins
\topmargin=-0.45in
\evensidemargin=0in
\oddsidemargin=0in
\textwidth=6.5in
\textheight=9.0in
\headsep=0.25in 

\linespread{1.1} % Line spacing

% Set up the header and footer
\pagestyle{fancy}
\lhead{\hmwkAuthorName} % Top left header
\chead{\hmwkClass\ : \hmwkTitle} % Top center header
\rhead{\firstxmark} % Top right header
\lfoot{\lastxmark} % Bottom left footer
\cfoot{} % Bottom center footer
\rfoot{Page\ \thepage\ of\ \pageref{LastPage}} % Bottom right footer
\renewcommand\headrulewidth{0.4pt} % Size of the header rule
\renewcommand\footrulewidth{0.4pt} % Size of the footer rule

\setlength\parindent{0pt} % Removes all indentation from paragraphs

%----------------------------------------------------------------------------------------
%	DOCUMENT STRUCTURE COMMANDS
%	Skip this unless you know what you're doing
%----------------------------------------------------------------------------------------

% Header and footer for when a page split occurs within a problem environment
\newcommand{\enterProblemHeader}[1]{
\nobreak\extramarks{#1}{#1 continued on next page\ldots}\nobreak
\nobreak\extramarks{#1 (continued)}{#1 continued on next page\ldots}\nobreak
}

% Header and footer for when a page split occurs between problem environments
\newcommand{\exitProblemHeader}[1]{
\nobreak\extramarks{#1 (continued)}{#1 continued on next page\ldots}\nobreak
\nobreak\extramarks{#1}{}\nobreak
}

\setcounter{secnumdepth}{0} % Removes default section numbers
\newcounter{homeworkProblemCounter} % Creates a counter to keep track of the number of problems

\newcommand{\homeworkProblemName}{}
\newenvironment{homeworkProblem}[1][Problem \arabic{homeworkProblemCounter}]{ % Makes a new environment called homeworkProblem which takes 1 argument (custom name) but the default is "Problem #"
\stepcounter{homeworkProblemCounter} % Increase counter for number of problems
\renewcommand{\homeworkProblemName}{#1} % Assign \homeworkProblemName the name of the problem
\section{\homeworkProblemName} % Make a section in the document with the custom problem count
\enterProblemHeader{\homeworkProblemName} % Header and footer within the environment
}{
\exitProblemHeader{\homeworkProblemName} % Header and footer after the environment
}

\newcommand{\problemAnswer}[1]{ % Defines the problem answer command with the content as the only argument
\noindent\framebox[\columnwidth][c]{\begin{minipage}{0.98\columnwidth}#1\end{minipage}} % Makes the box around the problem answer and puts the content inside
}

\newcommand{\homeworkSectionName}{}
\newenvironment{homeworkSection}[1]{ % New environment for sections within homework problems, takes 1 argument - the name of the section
\renewcommand{\homeworkSectionName}{#1} % Assign \homeworkSectionName to the name of the section from the environment argument
\subsection{\homeworkSectionName} % Make a subsection with the custom name of the subsection
\enterProblemHeader{\homeworkProblemName\ [\homeworkSectionName]} % Header and footer within the environment
}{
\enterProblemHeader{\homeworkProblemName} % Header and footer after the environment
}
   
%----------------------------------------------------------------------------------------
%	NAME AND CLASS SECTION
%----------------------------------------------------------------------------------------

\newcommand{\hmwkTitle}{Homework\ \# 4 } % Assignment title
\newcommand{\hmwkDueDate}{Friday,\ September \ 19,\ 2014} % Due date
\newcommand{\hmwkClass}{MATH-505A} % Course/class
\newcommand{\hmwkClassTime}{10:30am} % Class/lecture time
\newcommand{\hmwkAuthorName}{Saket Choudhary} % Your name
\newcommand{\hmwkAuthorID}{2170058637} % Teacher/lecturer
%----------------------------------------------------------------------------------------
%	TITLE PAGE
%----------------------------------------------------------------------------------------

\title{
\vspace{2in}
\textmd{\textbf{\hmwkClass:\ \hmwkTitle}}\\
\normalsize\vspace{0.1in}\small{Due\ on\ \hmwkDueDate}\\
%\vspace{0.1in}\large{\textit{\hmwkClassTime}}
\vspace{3in}
}

\author{\textbf{\hmwkAuthorName} \\
	\textbf{\hmwkAuthorID}
	}
\date{} % Insert date here if you want it to appear below your name

%----------------------------------------------------------------------------------------

\begin{document}

\maketitle

%----------------------------------------------------------------------------------------
%	TABLE OF CONTENTS
%----------------------------------------------------------------------------------------

%\setcounter{tocdepth}{1} % Uncomment this line if you don't want subsections listed in the ToC

\newpage
\tableofcontents
\newpage

\begin{homeworkProblem}[Exercise \# 2.1] % Custom section title


	\begin{homeworkSection}{(1)} % Section within problem
		
		\problemAnswer{ % Answer
			\textbf{Given:} $X$ is a random variable $\implies$
			\begin{equation}
			\label{1c1}
				\{\omega \in \Omega: X(\omega) \leq x\} \in \mathcal{F}\  \forall x \in R
			\end{equation}
			 
			
			\textbf{Part A)} To Prove: $aX$ is a random variable
			
			Consider $Y = aX$, then since equation \ref{1c1} holds:
			
			\textbf{Case1: $a \geq 0$}
			
			Then  $\{\omega \in \Omega: aX(\omega) \leq x'\} \in \mathcal{F}\  \forall x' \in R$ where $x' = ax$
			
			\textbf{Case2: $a \leq 0$} 
			
			Then $\{\omega \in \Omega: aX(\omega) \geq x'\} \forall x' \in R$ where $x' = ax$ $\implies$ $\cup\{\{\omega \in \Omega: aX(\omega) \leq x''\}\}^c \in \mathcal{F}$ where $x'' = x'$ 
			
			\textbf{Case3: $a\ is\ 0$}
			
			Then, $aX=0$
			
			\textbf{Case i: $ x < 0$ }
			
			$\{\omega \in \Omega: aX(\omega) = \phi \} \in \mathcal{F}$ 
			
			\textbf{Case ii: $ x \geq 0$}
			
			$\{\omega \in \Omega: aX(\omega) = \Omega \} \in \mathcal{F}$ 
			
			Thus from all the above cases.
			
			\textbf{Part (b))}:
			
			Consider $Y = X - X$, Then:
			
			$Y = X(\omega) - X(\omega) \forall \omega \in R$ 
			$\implies$ $ Y = 0$
			
			Consider $Y = X + X$, Then $Y= X(\omega) + X(\omega)  \forall \omega in \Omega \implies Y = 2X(\omega) \forall \omega in \Omega$ Thus $Y = 2X$.
			
			
						
		}
		
	\end{homeworkSection}
	\begin{homeworkSection}{(2)} % Section within problem
		
		\problemAnswer{ % Answer
			
			For part 1, $Y' = aX$ is also a random variable:
			
			\textbf{To Find:}  Distribution fucntion of $Y = aX + b$
			
			Consider $P(Y \leq y) = P(aX + b \leq y) = P(X \leq \frac{y-b}{a})$ $\implies P(Y\leq y) = P(X \leq \frac{x-b}{a})$ 
			%\begin{comment}
			%$Y = Y' + b$ is a ranodm variable where $Y'$ is a random variable and $b$ is a constant.
			
			%Since $Y'$ is a random variable: $\{\omega \in \Omega: Y(\omega) \leq y\} \in \mathcal{F} \ \forall y \in R$ and so,
			%$\{\omega \in \Omega: Y(\omega) + b \leq y'\} \in \mathcal{F}\  \forall y' \in R$ where $y' = y + b$
			
			%Since $\{\omega \in \Omega: Y(\omega) + b \leq y'\} \in \mathcal{F}\  \forall y' \in R$, $Y' + b$ is a random variable $\implies$ $aX + b$ is a random variable
			%\end{comment}
		}
		
	\end{homeworkSection}
	\begin{homeworkSection}{(3)} % Section within problem
		
		\problemAnswer{ % Answer
			
			$p(H) = p$; $p(T) = 1-p$
			
			Tossing a coin $n$ times is a binomial process(each individual toss is a bernoulli process) and let A be the event such that $k$ out of $n$ tosses are heads and this can occur in $\binom{n}{k}$ ways with probability $p^k$. There would also be $n-k$ tails  and the probability for that is $(1-p)^{n-k}$. Thus,: \\ 
			$p(A) = \binom{n}{k} p^{k}*(1-p)^{n-k}$ 
			
			For a fair coin, $p = \frac{1}{2}$ and hence p(A) = $\binom{n}{k} (\frac{1}{2})^k (\frac{1}{2})^{n-k}$ = $\binom{n}{k} (\frac{1}{2})^n$ 
		}
		
	\end{homeworkSection}
	\begin{homeworkSection}{(4)} % Section within problem
		
		\problemAnswer{ % Answer
				A distribution function satisfies the following set of properties: \\
				
				a) $\lim_{x\rightarrow- \infty}\ F(x)=0$, $\lim_{x\rightarrow\infty}\ F(x)=1$
				
				b) if $x<y$ then $F(x) \leq F(y)$,
				
				c) F is right continuous, $ c < x < c + \delta$ then
				$|F(x)-F(c)| < \epsilon$ for $\epsilon>0, \delta > 0$
				
				Consider $Y = \lambda F + (1-\lambda)G$, Both G,F satisfy $a,b,c$
				Then $\lim_{x \rightarrow -\infty}Y(x) = \lambda \lim_{x \rightarrow -\infty} F(x) + (1-\lambda) \lim_{x \rightarrow -\infty} G(x)$ $\implies$ 
				$\lim_{x \rightarrow -\infty}Y(x) = 0$
				
				Similarly considering limit as $x \rightarrow \infty$:
				Then $\lim_{x \rightarrow \infty}Y(x) = \lambda \lim_{x \rightarrow \infty} F(x) + (1-\lambda) \lim_{x \rightarrow \infty} G(x)$ $\implies$ 
				$\lim_{x \rightarrow -\infty}Y(x) = \lambda * 1 + (1-\lambda)*1 = 1$
				
				Since for $x < y $, then $F(x) < F(y)$; $G(x) < G(y)$
				$\implies$ $\lambda F(x) < \lambda F(y)$; $(1-\lambda)G(x) < (1-\lambda)G(y)$ since $ 0 \leq \lambda \leq 1$
				
				Adding the two inequalities we get: \\
				
				$\lambda F(x) + (1-\lambda)G(x) < \lambda F(y) + (1-\lambda)G(y)$ $\implies$ $Y(x) < Y(y)$.
				
				Since F,G are right continuous, any linear combination of these would be right continuous too.
				
				
				Hence $Y=\lambda F + (1-\lambda)G$ satisfies all the 3 required properties and is a distribution function.
				
		}
		
	\end{homeworkSection}
	\begin{homeworkSection}{(5)} % Section within problem
		
		\problemAnswer{ % Answer
		
		Since F is a distribution function: \\
		(i) $\lim_{x \rightarrow - \infty}F(x) = 0; \lim_{x \rightarrow \infty} = 1$
		
		(ii) If $x<y$ then, $F(x) < F(y)$
		
		(iii) F is right continuous
		
		\textbf{Part a)} $F(x)^r$
		(i) $\lim_{x \rightarrow - \infty}F(x)^r = 0$ since $\lim_{x \rightarrow -\infty}F(x) = 0 $ and $ r > 0$
		
		(ii) If $ x< y$ as $F(x) < F(y) and r > 0$ $\implies$ $F(x)^r < F(y)^r $
		
		(iii) Since $r > 0$ and $F(x)$ is right-continuous $F(x)^r$ is right continuous. (One possible case ehere $F(x)^r$ would not have been right continuous is for $r<0$ say r = -1 where $F(x)^-1$ is not right continuous at all $x_0$ such that $F(x_0)=0$.
		
		\textbf{Part b)} $1 - (1-F(x))^r$
		
		(i) $; \lim_{x \rightarrow -\infty}(1-(1-F(x))^r)) = 1 - \lim_{x \rightarrow - \infty}(1-F(x))^r = 1 - (1-0)^r = 0$
		
		Similarly for $; \lim_{x \rightarrow \infty} (1-(1-F(x))^r) = 1- (1-1)^r = 1$
		
		(ii) If $ x < y$, $F(x) < F(y) \implies -F(x) > -F(y) \implies 1 - F(x) > 1 - F(y) \implies (1-F(x))^r > (1-F(y))^r \forall r > 0$
		Thus, $1 - (1-F(x))^r < 1 - (1-F(y))^r$ 
		
		(iii) Since F(x) is right continuous, $1-F(x)$ is right continuous $\implies$ $(1-F(x))^r$ is right continuous(since ($r>0$) $implies$ $1 - (1-F(x))^r$ is right continuous
		
		\textbf{Part c} $F(x) + (1-F(x))log(1-F(x))$
		
		(i) $\lim_{x \rightarrow -\infty} (F(x) + (1-F(x))log(1-F(x))) =  \lim_{x \rightarrow -\infty} F(x) + \lim_{x \rightarrow -\infty} (1-F(x))log(1-F(x)) = 0 + (1-0)log(1-0) = 0 $
		
		Consider $\lim_{x \rightarrow \infty} (F(x) + (1-F(x))log(1-F(x))) = \lim_{x \rightarrow \infty} F(x) + \lim_{x \rightarrow \infty} (1-F(x))log(1-F(x)) = 1 + (1-1) log(1-1) = 0 $
		
		(ii) If $x < y$ then $F(x) < F(y) \implies 1 - F(x) > 1- F(y)$ . Since log is a monotonic non-increasing function in $[0, 1]$ and is in fact negative definite: 
		$log(1-F(x)) < log (1-F(y))$ and $1-F(x) > 1-F(y)$ $\implies$ $-F(x)log(1-F(x)) < -F(y)log(1-F(y))$ (This holds only because log is negative definite in $[0,1]$)	
		
		Thus, $F(x) -F(x)log(1-F(x)) < F(y)-F(y)log(1-F(y))$
		
		(iii) $F(x) -F(x)log(1-F(x))$ is right continuous as log(1-F(x)) is right continuous
		
		
		\textbf{Part d)} $(F(x)-1)e + exp(1-F(x))$
		
		(i) $\lim_{x \rightarrow -\infty} (F(x)-1)e + exp(1-F(x)) = \lim_{x \rightarrow - \infty} (F(x)-1)e + exp(\lim_{x \rightarrow - \infty} (1-F(x))) = (0-1)e + exp(1-0) = -e + e = 0$
		
		$\lim_{x \rightarrow \infty} (F(x)-1)e + exp(1-F(x)) = \lim_{x \rightarrow \infty} (F(x)-1)e + exp(\lim_{x \rightarrow  \infty} (1-F(x))) = (1-1)e + exp(1-1) = 0 + 1 =1 $
		
		(ii) if $ x < y$ , $F(x) < F(y)$ $\implies$ $F(x)-1 < F(y)-1$ $\implies$ $(F(x) -1)e < (F(y)-1)e$
		Also, $ 1-F(x) > 1-F(y)$
		
		Since $exp$ is a non-increasing function in $[0,1]$
		$exp(1-F(x)) < exp(1-F(y))$ 
		
		Thus,
		
		$(F(x)-1)e + exp(1-F(x)) < (F(y)-1)e + exp(1-F(y))$   
		
		(iii) $(F(x)-1)e + exp(1-F(x))$ is right continuous as $exp$ is right continuous.
		
		
		
		FG is also a  density  function since it satisfies: \\
		(i) $\lim_{x \rightarrow -\infty} F(x)G(x) = \lim_{x \rightarrow - \infty} F(x) * \lim_{x \rightarrow - \infty}G(x) = 0$
		
		And  $\lim_{x \rightarrow \infty} F(x)G(x) = \lim_{x \rightarrow  \infty} F(x) * \lim_{x \rightarrow \infty}G(x) = 1$ 
		
		(ii) If $x<y$, $F(x)<F(y)$ and $G(x)<G(y)$ $\implies$ $F(x)G(x)<F(y)G(y)$
		
		(iii) Since $F(x), G(x)$ are right continuous
		
		
		
		 
		}
		

	\end{homeworkSection}


\end{homeworkProblem}

\begin{homeworkProblem}[Exercise \# 2.3] % Custom section title
	
	
	\begin{homeworkSection}{(1)} % Section within problem
		
		\problemAnswer{ % Answer
			\textbf{Given:} $\lim_{m \rightarrow -\infty} a_{m} \rightarrow -\infty$ and $\lim_{m \rightarrow \infty} a_{m} \rightarrow \infty$; $G(x) = P(X \leq a_m)$ when $a_{m-1} \leq x \leq a_{m}$; $a_m$ is a strictly increasing sequence.
			
			Sequence $a$ is chosen so that $sup_m | a_m - a_{m-1}|$ becomes smaller and smaller so even though the sequence is increasing the successive difference between the terms keep on decreasing essentially indicating $a_m$ saturates as $m \rightarrow \infty $
			
			
			
		
		}
		
	\end{homeworkSection}
	\begin{homeworkSection}{(2)} % Section within problem
		
		\problemAnswer{ % Answer
			\textbf{Given:} $g(x)$ is continuous and strictly increasing, X is a random variable: \\
			
			Since g(X) is continuous and strictly increasing $\implies g^{-1}$ exists. 
			
			Consider $\{Y \leq y\}$ $\implies \{g(X) \leq y\} $. Since $g^{-1}$ exists, such a set is equivalent to: $\{X \leq g^{-1}(y)\}$ which belongs to $\mathcal{F}$ as $g : R \rightarrow R$
		
				
		}
		
	\end{homeworkSection}
	\begin{homeworkSection}{(3)} % Section within problem
		
		\problemAnswer{ % Answer
			\begin{math}
				F(x) = P(X \leq x) = 
				\begin{cases}
					0 & if x \leq 0,\\
					x & if 0 < x \leq 1, \\
					1 & if x>1,
					 
				\end{cases}
			\end{math}
			\textbf{To Prove:} $Y = F^{-1}(x)$ is a random variable:
			Consider $\{Y \leq y\} = \{F^{-1}(x) \leq y\}$ . Since F is continuous and strictly increasing $\implies$ $F^{-1}(x)$ exists in $R$ so $\{Y \leq y\} = \{F^{-1}(x) \leq y\} = \{(x) \leq F(y)\} = \{x \leq P(X\leq y)\} \in \mathcal{F}$

	
			$F$ should necessarily be continuous and monotonic for the inverse to exist! 
		}
		
	\end{homeworkSection}
	\begin{homeworkSection}{(4)} % Section within problem
		
		\problemAnswer{ % Answer
			$f,g$ are density functions: \\
			$\int_{-\infty}^{\infty}f(x) = 1$ and $\int_{-\infty}^{\infty}g(x) = 1$	
			
			Consider $y(x) = \lambda f(x) + (1-\lambda)g(x)$  
			Thus, $\int_{-\infty}^{\infty} y(x)dx =  \int_{-\infty}^{\infty} \lambda f(x)dx + \int_{-\infty}^{\infty}(1-\lambda)g(x)dx =  \lambda * \int_{-\infty}^{\infty}  f(x)dx + (1-\lambda) * \int_{-\infty}^{\infty}g(x)dx  = \lambda*1 + (1-\lambda)*1 = 1$
			
			Thus $\lambda f(x) + (1-\lambda)g(x)$ is a density function too.
			
			
			Now consider $y(x) = f(x)g(x)$, then : \\
			$\int_{-\infty}^{\infty} y(x)dx =\int_{-\infty}^{\infty}f(x)g(x)dx $
			Clearly this is nor neceesarily equal to 1 so $fg$ is not a density function!
		}
		
	\end{homeworkSection}
	\begin{homeworkSection}{(5)} % Section within problem
		
		\problemAnswer{ % Answer
			\textbf{Part a)}
			\begin{math}
				f(x) =
				\begin{cases}
				cx^{-d} & x>1,\\
				0 & otherwise,
				\end{cases}
			\end{math}
			
			$F(x) = \int_{-\infty}^{\infty}f(c)dx = \int_{1}^{\infty}cx^{-d} = 1$
			$\implies$ $\frac{-c}{-d+1} =1$ $\implies$ $c = d-1$ and $-d+1 <1$ i.e $d>0$ else the integral blows up to $\infty$
			
			\textbf{Part b)}
			\begin{math}
				f(x) = ce^x(1+e^x)^{-2} x \in R
			\end{math}  
			
			$F(x) = \int_{-\infty}^{infty} ce^x(1+e^x)^{-2}$ 
			Let $t = e^x+1$ then $e^xdx = dt$
			$F(x) = \int_{1}^{\infty}ct^{-2} dt $
			$F(x=1) = -(c*0 - c) = 1$
			
			Thus $c=1$. 
			
		}
		
	\end{homeworkSection}
	
	
	
	
\end{homeworkProblem}

\end{document}
