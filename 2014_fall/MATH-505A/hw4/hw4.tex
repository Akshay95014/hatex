%%%%%%%%%%%%%%%%%%%%%%%%%%%%%%%%%%%%%%%%%
% Structured General Purpose Assignment
% LaTeX Template
%
% This template has been downloaded from:
% http://www.latextemplates.com
%
% Original author:
% Ted Pavlic (http://www.tedpavlic.com)
%
% Note:
% The \lipsum[#] commands throughout this template generate dummy text
% to fill the template out. These commands should all be removed when 
% writing assignment content.
%
%%%%%%%%%%%%%%%%%%%%%%%%%%%%%%%%%%%%%%%%%

%----------------------------------------------------------------------------------------
%	PACKAGES AND OTHER DOCUMENT CONFIGURATIONS
%----------------------------------------------------------------------------------------

\documentclass{article}

\usepackage{fancyhdr} % Required for custom headers
\usepackage{lastpage} % Required to determine the last page for the footer
\usepackage{extramarks} % Required for headers and footers
\usepackage{graphicx} % Required to insert images
\usepackage{latexsym}

\usepackage{lipsum} % Used for inserting dummy 'Lorem ipsum' text into the template

\usepackage{amsmath}
%\usepackage{multline}

% Margins
\topmargin=-0.45in
\evensidemargin=0in
\oddsidemargin=0in
\textwidth=6.5in
\textheight=9.0in
\headsep=0.25in 

\linespread{1.1} % Line spacing

% Set up the header and footer
\pagestyle{fancy}
\lhead{\hmwkAuthorName} % Top left header
\chead{\hmwkClass\ : \hmwkTitle} % Top center header
\rhead{\firstxmark} % Top right header
\lfoot{\lastxmark} % Bottom left footer
\cfoot{} % Bottom center footer
\rfoot{Page\ \thepage\ of\ \pageref{LastPage}} % Bottom right footer
\renewcommand\headrulewidth{0.4pt} % Size of the header rule
\renewcommand\footrulewidth{0.4pt} % Size of the footer rule

\setlength\parindent{0pt} % Removes all indentation from paragraphs

%----------------------------------------------------------------------------------------
%	DOCUMENT STRUCTURE COMMANDS
%	Skip this unless you know what you're doing
%----------------------------------------------------------------------------------------

% Header and footer for when a page split occurs within a problem environment
\newcommand{\enterProblemHeader}[1]{
\nobreak\extramarks{#1}{#1 continued on next page\ldots}\nobreak
\nobreak\extramarks{#1 (continued)}{#1 continued on next page\ldots}\nobreak
}

% Header and footer for when a page split occurs between problem environments
\newcommand{\exitProblemHeader}[1]{
\nobreak\extramarks{#1 (continued)}{#1 continued on next page\ldots}\nobreak
\nobreak\extramarks{#1}{}\nobreak
}

\setcounter{secnumdepth}{0} % Removes default section numbers
\newcounter{homeworkProblemCounter} % Creates a counter to keep track of the number of problems

\newcommand{\homeworkProblemName}{}
\newenvironment{homeworkProblem}[1][Problem \arabic{homeworkProblemCounter}]{ % Makes a new environment called homeworkProblem which takes 1 argument (custom name) but the default is "Problem #"
\stepcounter{homeworkProblemCounter} % Increase counter for number of problems
\renewcommand{\homeworkProblemName}{#1} % Assign \homeworkProblemName the name of the problem
\section{\homeworkProblemName} % Make a section in the document with the custom problem count
\enterProblemHeader{\homeworkProblemName} % Header and footer within the environment
}{
\exitProblemHeader{\homeworkProblemName} % Header and footer after the environment
}

\newcommand{\problemAnswer}[1]{ % Defines the problem answer command with the content as the only argument
\noindent\framebox[\columnwidth][c]{\begin{minipage}{0.98\columnwidth}#1\end{minipage}} % Makes the box around the problem answer and puts the content inside
}

\newcommand{\homeworkSectionName}{}
\newenvironment{homeworkSection}[1]{ % New environment for sections within homework problems, takes 1 argument - the name of the section
\renewcommand{\homeworkSectionName}{#1} % Assign \homeworkSectionName to the name of the section from the environment argument
\subsection{\homeworkSectionName} % Make a subsection with the custom name of the subsection
\enterProblemHeader{\homeworkProblemName\ [\homeworkSectionName]} % Header and footer within the environment
}{
\enterProblemHeader{\homeworkProblemName} % Header and footer after the environment
}
   
%----------------------------------------------------------------------------------------
%	NAME AND CLASS SECTION
%----------------------------------------------------------------------------------------

\newcommand{\hmwkTitle}{Homework\ \# 4 } % Assignment title
\newcommand{\hmwkDueDate}{Friday,\ September \ 19,\ 2014} % Due date
\newcommand{\hmwkClass}{MATH-505A} % Course/class
\newcommand{\hmwkClassTime}{10:30am} % Class/lecture time
\newcommand{\hmwkAuthorName}{Saket Choudhary} % Your name
\newcommand{\hmwkAuthorID}{2170058637} % Teacher/lecturer
%----------------------------------------------------------------------------------------
%	TITLE PAGE
%----------------------------------------------------------------------------------------

\title{
\vspace{2in}
\textmd{\textbf{\hmwkClass:\ \hmwkTitle}}\\
\normalsize\vspace{0.1in}\small{Due\ on\ \hmwkDueDate}\\
%\vspace{0.1in}\large{\textit{\hmwkClassTime}}
\vspace{3in}
}

\author{\textbf{\hmwkAuthorName} \\
	\textbf{\hmwkAuthorID}
	}
\date{} % Insert date here if you want it to appear below your name

%----------------------------------------------------------------------------------------

\begin{document}

\maketitle

%----------------------------------------------------------------------------------------
%	TABLE OF CONTENTS
%----------------------------------------------------------------------------------------

%\setcounter{tocdepth}{1} % Uncomment this line if you don't want subsections listed in the ToC

\newpage
\tableofcontents
\newpage

\begin{homeworkProblem}[Exercise \# 2.1] % Custom section title


	\begin{homeworkSection}{(1)} % Section within problem
		
		\problemAnswer{ % Answer
			\textbf{Given:} $X$ is a random variable $\implies$
			\begin{equation}
			\label{1c1}
				\{
				\omega \in \Omega: X(\omega) \leq x\} \forall x \in R
			\end{equation}
			 
			
			\textbf{Part A)} To Prove: $aX$ is a random variable
			
			Consider $Y = aX$, then since \ref{1c1} holds:
			
			\textbf{Case1: $a \geq 0$}
			
			Then  $\{\omega \in \Omega: aX(\omega) \leq x'\} \forall x' \in R$ where $x' = ax$
			
			\textbf{Case2: $a \leq 0$} 
			
			Then $\{\omega \in \Omega: aX(\omega) \geq x'\} \forall x' \in R$ where $x' = ax$ $\implies$ $\cup\{\{\omega \in \Omega: aX(\omega) \leq x''\}\}^c$ where $x'' = x'$ 
			
			\textbf{Case3: $a is\ 0$}
			
			Then, $aX=0$
			
			\textbf{Case 1: $ x < 0$ }
			
			$\{\omega \in \Omega: aX(\omega) = \phi \}$ 
			
			\textbf{Case 2: $ x \geq 0$}
			
			$\{\omega \in \Omega: aX(\omega) = \Omega \}$ 
			
			
						
		}
		
	\end{homeworkSection}
	\begin{homeworkSection}{(2)} % Section within problem
		
		\problemAnswer{ % Answer
			
			For part 1, $Y' = aX$ is also a random variable:
			\textbf{To Prove:} $Y = Y' + b$ is a ranodm variable where $Y'$ is a random variable and $b$ is a constant.
			
			Since $Y'$ is a random variable: $\{\omega \in \Omega: Y(\omega) \leq y\} \forall y \in R$ and so,
			$\{\omega \in \Omega: Y(\omega) + b \leq y'\} \forall y' \in R$ where $y' = y + b$
			
			Hence $Y'+b$ is a random variable $\implies$ $aX + b$ is a random variable
			
		}
		
	\end{homeworkSection}
	\begin{homeworkSection}{(3)} % Section within problem
		
		\problemAnswer{ % Answer
			
			$p(H) = p$; $p(T) = 1-p$
			
			Tossing a coin $n$ times is a binomial process(each individual toss is a bernoulli process) and let A be the event such that $k$ out of $n$ tosses are heads: \\
			
			$p(A) = \binom{n}{k} p^{k}*(1-p)
		}
		
	\end{homeworkSection}
	\begin{homeworkSection}{(4)} % Section within problem
		
		\problemAnswer{ % Answer
			
		}
		
	\end{homeworkSection}
	\begin{homeworkSection}{(5)} % Section within problem
		
		\problemAnswer{ % Answer
			
		}
		

	\end{homeworkSection}


\end{homeworkProblem}

\begin{homeworkProblem}[Exercise \# 2.3] % Custom section title
	
	
	\begin{homeworkSection}{(1)} % Section within problem
		
		\problemAnswer{ % Answer
			
		}
		
	\end{homeworkSection}
	\begin{homeworkSection}{(2)} % Section within problem
		
		\problemAnswer{ % Answer
			
		}
		
	\end{homeworkSection}
	\begin{homeworkSection}{(3)} % Section within problem
		
		\problemAnswer{ % Answer
			
		}
		
	\end{homeworkSection}
	\begin{homeworkSection}{(4)} % Section within problem
		
		\problemAnswer{ % Answer
			
		}
		
	\end{homeworkSection}
	\begin{homeworkSection}{(5)} % Section within problem
		
		\problemAnswer{ % Answer
			
		}
		
	\end{homeworkSection}
	
	
	
	
\end{homeworkProblem}

\end{document}
