%%%%%%%%%%%%%%%%%%%%%%%%%%%%%%%%%%%%%%%%%
% Structured General Purpose Assignment
% LaTeX Template
%
% This template has been downloaded from:
% http://www.latextemplates.com
%
% Original author:
% Ted Pavlic (http://www.tedpavlic.com)
%
% Note:
% The \lipsum[#] commands throughout this template generate dummy text
% to fill the template out. These commands should all be removed when 
% writing assignment content.
%
%%%%%%%%%%%%%%%%%%%%%%%%%%%%%%%%%%%%%%%%%

%----------------------------------------------------------------------------------------
%	PACKAGES AND OTHER DOCUMENT CONFIGURATIONS
%----------------------------------------------------------------------------------------

\documentclass{article}

\usepackage{fancyhdr} % Required for custom headers
\usepackage{lastpage} % Required to determine the last page for the footer
\usepackage{extramarks} % Required for headers and footers
\usepackage{graphicx} % Required to insert images
\usepackage{latexsym}

\usepackage{lipsum} % Used for inserting dummy 'Lorem ipsum' text into the template

\usepackage{amsmath}

%\usepackage{multline}

% Margins
\topmargin=-0.45in
\evensidemargin=0in
\oddsidemargin=0in
\textwidth=6.5in
\textheight=9.0in
\headsep=0.25in 

\linespread{1.1} % Line spacing

% Set up the header and footer
\pagestyle{fancy}
\lhead{\hmwkAuthorName} % Top left header
\chead{\hmwkClass\ : \hmwkTitle} % Top center header
\rhead{\firstxmark} % Top right header
\lfoot{\lastxmark} % Bottom left footer
\cfoot{} % Bottom center footer
\rfoot{Page\ \thepage\ of\ \pageref{LastPage}} % Bottom right footer
\renewcommand\headrulewidth{0.4pt} % Size of the header rule
\renewcommand\footrulewidth{0.4pt} % Size of the footer rule

\setlength\parindent{0pt} % Removes all indentation from paragraphs

%----------------------------------------------------------------------------------------
%	DOCUMENT STRUCTURE COMMANDS
%	Skip this unless you know what you're doing
%----------------------------------------------------------------------------------------

% Header and footer for when a page split occurs within a problem environment
\newcommand{\enterProblemHeader}[1]{
\nobreak\extramarks{#1}{#1 continued on next page\ldots}\nobreak
\nobreak\extramarks{#1 (continued)}{#1 continued on next page\ldots}\nobreak
}

% Header and footer for when a page split occurs between problem environments
\newcommand{\exitProblemHeader}[1]{
\nobreak\extramarks{#1 (continued)}{#1 continued on next page\ldots}\nobreak
\nobreak\extramarks{#1}{}\nobreak
}

\setcounter{secnumdepth}{0} % Removes default section numbers
\newcounter{homeworkProblemCounter} % Creates a counter to keep track of the number of problems

\newcommand{\homeworkProblemName}{}
\newenvironment{homeworkProblem}[1][Problem \arabic{homeworkProblemCounter}]{ % Makes a new environment called homeworkProblem which takes 1 argument (custom name) but the default is "Problem #"
\stepcounter{homeworkProblemCounter} % Increase counter for number of problems
\renewcommand{\homeworkProblemName}{#1} % Assign \homeworkProblemName the name of the problem
\section{\homeworkProblemName} % Make a section in the document with the custom problem count
\enterProblemHeader{\homeworkProblemName} % Header and footer within the environment
}{
\exitProblemHeader{\homeworkProblemName} % Header and footer after the environment
}

\newcommand{\problemAnswer}[1]{ % Defines the problem answer command with the content as the only argument
\noindent\framebox[\columnwidth][c]{\begin{minipage}{0.98\columnwidth}#1\end{minipage}} % Makes the box around the problem answer and puts the content inside
}

\newcommand{\homeworkSectionName}{}
\newenvironment{homeworkSection}[1]{ % New environment for sections within homework problems, takes 1 argument - the name of the section
\renewcommand{\homeworkSectionName}{#1} % Assign \homeworkSectionName to the name of the section from the environment argument
\subsection{\homeworkSectionName} % Make a subsection with the custom name of the subsection
\enterProblemHeader{\homeworkProblemName\ [\homeworkSectionName]} % Header and footer within the environment
}{
\enterProblemHeader{\homeworkProblemName} % Header and footer after the environment
}
   
%----------------------------------------------------------------------------------------
%	NAME AND CLASS SECTION
%----------------------------------------------------------------------------------------

\newcommand{\hmwkTitle}{Homework\ \# 6 } % Assignment title
\newcommand{\hmwkDueDate}{Friday,\ October \ 3,\ 2014} % Due date
\newcommand{\hmwkClass}{MATH-505A} % Course/class
\newcommand{\hmwkClassTime}{10:30am} % Class/lecture time
\newcommand{\hmwkAuthorName}{Saket Choudhary} % Your name
\newcommand{\hmwkAuthorID}{2170058637} % Teacher/lecturer
%----------------------------------------------------------------------------------------
%	TITLE PAGE
%----------------------------------------------------------------------------------------

\title{
\vspace{2in}
\textmd{\textbf{\hmwkClass:\ \hmwkTitle}}\\
\normalsize\vspace{0.1in}\small{Due\ on\ \hmwkDueDate}\\
%\vspace{0.1in}\large{\textit{\hmwkClassTime}}
\vspace{3in}
}

\author{\textbf{\hmwkAuthorName} \\
	\textbf{\hmwkAuthorID}
	}
\date{} % Insert date here if you want it to appear below your name

%----------------------------------------------------------------------------------------

\begin{document}

\maketitle

%----------------------------------------------------------------------------------------
%	TABLE OF CONTENTS
%----------------------------------------------------------------------------------------

%\setcounter{tocdepth}{1} % Uncomment this line if you don't want subsections listed in the ToC

\newpage
\tableofcontents
\newpage

\begin{homeworkProblem}[Exercise \# 3.1] % Custom section title


	
	\begin{homeworkSection}{1}
	\problemAnswer{
	 \textbf{Part a: $f(x) = C2^{-x}$}
	 
	 For f(x) to be a mass function $\sum_{1}^{\infty} C2^-i=1$
	 $C(\frac{1}{2} + \frac{1}{2^2} + \frac{1}{2^3} +....) = C \frac{1}{2} * \frac{1}{1-\frac{1}{2}} = 1$ $\implies$ $C=1$
	 
	 \textbf{Part b: $f(x) = \frac{C2^{-x}}{x}$}
	 	 $\sum_{1}^{\infty} \frac{C}{2^ii}=1$ 
	 	 %\begin{comment}
	 	 %Approximating summation by a definite integral: \\
	 	 %$\int_{1}^{\infty} C\frac{2^{-x}}{x}=1 \implies C\int_{1}^{\infty}\frac{2^{-x}}{x}dx = 1$
	 	 %Let $t =2^{-x} \implies dt = -2^{-x}\ ln\ 2\ dx ; x = -\frac{ln\ t}{ln\ 2} $
	 	 %Thus $\int_{1}^{\infty}\frac{2^{-x}}{x}dx = \int_{\frac{1}{2}}^{0} \frac{-dt}{-ln\ t} $
	 	 %\end{comment}
	 	 
	 	 %$\sum_{1}^{\infty} \frac{C}{2^ii} = -C\sum_{1}^{\infty} \frac{2^{-i}}{-i} = -C e^-2$ Thus  
	 	 
	 	 
	 	 Notice $ln(1+x) =x + \frac{x^2}{2} + \frac{x^3}{3} +...$.
	 	 
	 	 Hence $C\sum_{1}^{\infty} \frac{(\frac{1}{2})^i}{i}=C\ ln(1+1/2)=1 \implies C=\frac{1}{ln 1.5}$
	 	  
	 	 
	 	 
	 	 
	 	 \textbf{ Part c: $f(x) = Cx^-2$}
	 	 	 	 $\sum_{1}^{\infty} \frac{C}{x^2} = 1$
	 	 	 	 
	 	 	 	 Besel sum: \\
	 	 	 	 $\sum_{1}^{n}\frac{1}{i^2} = $
	 	 	 	 
	 	 	 	 Using taylor expansion of $sin x$ and the fact that $\frac{sin x}{x}$ has roots at $x=\pi, 2\pi, 3\pi.....$  : \\
	 	 	 	 
	 	 	 	 $\frac{sin x}{x} = 1-\frac{x^2}{3!}+\frac{x^4}{5!}+...... = (1-\frac{x}{\pi})(1-\frac{x}{2\pi})(1-\frac{x}{3\pi})....(1-\frac{x}{\pi})$
	 	 	 	 
	 	 	 	 
	 	 	 	 The product of productions of $\frac{sin x}{x}$ is given the the coefficient of $x^2$ in the original and hence $-\frac{1}{3!} = -\pi^2\sum_{1}^{\infty} i^2$
	 	 	 	 
	 	 	 	 
	 	 	 	 Thus 	 $\sum_{1}^{\infty} \frac{C}{x^2} = C*\frac{\pi^2}{6}$.
	 	 	 	 Thus C=$\frac{6}{\pi^2}$
	 	 	 	 
	 	 	 
	 	 \textbf{Part d: $C2^x/x!$}
	 	 $\sum_{1}^{\infty}C2^i/i!= C\sum_{1}^{\infty}2^i/i! = Ce^2 \implies C = \frac{1}{e^2}$
	 	 
	 	 
	 
	 
	 	

	}
	\end{homeworkSection}
	
	\begin{homeworkSection}{2(i)}
	\problemAnswer{
	\textbf{Part a} $P(X>1) = \sum_{2}^{\infty}{2^{-i}} = \frac{1}{4}*2 = \frac{1}{2}$
	
	\textbf{Part b} $P(X>1) = \frac{1}{1.5}\sum_{2}^{\infty}\frac{(\frac{1}{2})^i}{i} = 1 - ln(1.5)/2$
	
	\textbf{Part c $P(X>1)$} = $1-\frac{6}{\pi^2}1^{-2} = 1-\frac{6}{\pi^2}$
	
	\textbf{Part d $P(X>1)$} = $1-\frac{1}{e^2}2 = 1 - \frac{2}{e^2} $
		 	
	}
	\end{homeworkSection}
	
	\begin{homeworkSection}{2 (iii)}
	\problemAnswer{
	Probability that X is even = $P(X=2k)\ for\ k=1,2,3...$
	
	\textbf{ Part a}
	$P(X=2k) = 2^{-2k}$
	Summing up over all k: $P=\frac{1}{2^2} + \frac{1}{2^4} + \frac{1}{2^6} + ....$ = $\frac{1}{4}\frac{1}{1-\frac{1}{4}} = \frac{1}{3}$
	
	\textbf{Part b}
	$ P(X=2k) = ln(3.5)\frac{\frac{1}{2}^{2k}}{2k} = ln(3.5)\frac{1}{2}\frac{(\frac{1}{4})^k}{k} = \frac{ln(3.5)}{2}e^{1.25}$
	
	\textbf{Part c}
	
	$P(X=2k) = \frac{6}{\pi^2}{4k^2} = \frac{3}{2\pi^2}*\frac{\pi^2}{6}=\frac{1}{4}$

	\textbf{Part d}
	
	$P(X=2k) = \frac{1}{e^2}\frac{2^{2k}}{(2k)!} $
		
	
	}
	\end{homeworkSection}
	
	
	\begin{homeworkSection}{3}
	\problemAnswer{
	Since the coin tosses are independent, the choice can be represented by two successive coin tosses with probability of heads being $p*p$.
	Thus $P(X=k) = \binom{n}{k}p^{2k}(1-p^2)^{n-k}$
	}
	\end{homeworkSection}
	

	\begin{homeworkSection}{5a}
	 \problemAnswer{
	 	\textbf{For Binomial: $f(k) = \binom{n}{k}p^k(1-p)^{n-k}$}
	 	Consider $LHS=f(k-1)*f(k+1) $
	 	$LHS=\binom{n}{k-1}p^{k-1}(1-p)^{n-k-1}+\binom{n}{k+1}p^{k+1}(1-p)^{n-k-1} = 
	 	\binom{n}{k-1}\binom{n}{k+1}(p^{k}(1-p)^{n-k})^2 $
	 	$RHS = \binom{n}{k}^2$
	 	We now focus on $\binom{n}{k-1}\binom{n}{k+1}$
	 	Let $y=	 	\frac{\binom{n}{k-1}\binom{n}{k+1}}{\binom{n}{k}^2}$
	 	Expanding: $y=\frac{n!n!(n-k)!(n-k)!k!k!}{(k-1)!(k+1)!(n-k+1)!(n-k-1)!} = \frac{k(n-k)}{(k+1)(n-k+1)}$
	 	$\frac{k}{k+1}\leq1$ and $\frac{n-k}{n-k+1}\leq1$ $\forall k$
	 	
	 	Hence $y \leq 1  $

		Thus 	 	$LHS \leq RHS$
		
		
		\textbf{For Poisson}
				$f(k) = e^{-\lambda}\frac{\lambda^k}{k!}$
				$LHS = f(k-1)f(k+1) = \frac{ e^{-2\lambda} \lambda^{2k} }{ (k+1)! (k-1)!}$
				%$RHS = f(k)^2 = \frac{e^{2k}\lambda^{2k}}{k!k!}$
				
				%$y=LHS/RHS = \frac{k!k!}{(k+1)!(k-1)!} = \frac{k}{k+1} \leq 1$
				Thus $LHS \leq RHS$
		
		
		}
	\end{homeworkSection}
	
	\begin{homeworkSection}{5b}
	\problemAnswer{
	$f(k)=\frac{90}{(\pi k)^4} $
	$LHS=f(k+1)f(k-1) = \frac{90^2}{\pi^8(k+1)^4(k-1)^4}$
	$RHS=f(k+1)^2 = \frac{90^2}{\pi^8(k)^8} $
	
	$y=LHS/RHS = \frac{k^8}{(k+1)^4(k-1)^4} = (\frac{k}{(k+1)}\frac{k}{(k-1)})^4 = (\frac{k^2}{k^2-1})^4 \geq 1$
	

	Thus $LHS \geq RHS$
	
	}
	
	\begin{homeworkSection}{5c}
	\problemAnswer{
		Any function of the form $P(x=k) = \frac{1}{n}$ satisfies $f(k)f(k-1) = f(k)^2$
		Note: We aren't explicitly talking about countably many case.
	}
	\end{homeworkSection}
	
	\end{homeworkSection}
	
	
		
	
	
\end{homeworkProblem}

\begin{homeworkProblem}[Exercise \# 3.2]
	\begin{homeworkSection}{2}
	\problemAnswer{
		\textbf{Part a $P(min(X,Y) \leq x )$}
		= $1-P(X>x \cup Y>y)=1-P(X>x)P(Y>y)=1-4^{-x}$
		
		\textbf{Part b $P(Y > x )$}
		=$\frac{1}{3}$ by symmetery of $P(X>Y)=P(X<Y)=P(X=Y)$
		\textbf{Part c $P(X=Y)$}
		=$\frac{1}{3}$ as in (b)
		
		\textbf{Part d $P(X \geq y )$}
		=$\frac{1}{3}$ as in (b)
		
		\textbf{Part e $P(X\ divides Y )$}
		=$P(Y=kX)=P(Y=kx,X=x)=P(Y=kx)P(X=x)=\sum2^{-kx}2^{-x}=\sum_{k=1}{k=\infty}\frac{1}{2^{k+1}-1}=$
		
		
		\textbf{Part f $P(X = rY )$}
		=$P(X=ry, Y=y)=\sum {2^{-ry}}2^{-y}=2^{-r-1}(2)=2^{-r}$
				
		
	}
		\begin{homeworkSection}{4}
		\problemAnswer{
		Consider three possibilities:
		1. A rolls a 6, B,C do not\\
		2. A and B roll a 6\\
		3. No one rolls 6\\
		
		$p=\frac{1}{6}(\frac{5}{6})^2P(B<C)+\frac{1}{6}\frac{1}{6} + (\frac{5}{6})^3 p$
		
		$P(B<C)=\frac{5}{6}\frac{5}{6}P(B<C)+\frac{1}{6} \implies P(B<C)=\frac{6}{11}$
		
		$p(1-\frac{125}{216})=\frac{25}{216}\frac{6}{11}+\frac{6}{216}$
		
		$p(\frac{91}{216})=\frac{216}{216*11}=\frac{216}{1001}$
		%$ $
		}
		\end{homeworkSection}
	
	\end{homeworkSection}
\end{homeworkProblem}


\begin{homeworkProblem}[Exercise \# 3.3]
	\begin{homeworkSection}{1}
	\problemAnswer{

	$E(\frac{1}{X}) = \sum p(x)(\frac{1}{x})$\\
	$\frac{1}{E(X)}) = \sum p(x)(x)$ \\
	For $E(X) = E(\frac{1}{X})	$: $\sum(p(x)(x-\frac{1}{x})=0$
	The above equation might not true be in general. However one possible case where this is true is  for this distribution: \\
	\begin{math}
	p(x) = \begin{cases}
	1/2 & x=1 or x=-1\\
	0 & otherwise	
	\end{cases}
	\end{math}
	}
	\end{homeworkSection}
	\begin{homeworkSection}{2}
	\problemAnswer{
	$a)$ Given there are $c$ objects with $j$ chosen. The probability to select a new 'distinct' component given j are already selected is $\frac{c-j}{c}$ . Thus the distribution  is geometric with parameter $p=\frac{c-j}{c}$ and the mean being $\frac{1}{p}=\frac{c}{c-j}$
	
	$b)$ Mean time required = $\sum1$
	$\sum_{j=0}^{c-1}\frac{c}{c-j}$ 
	}
	\end{homeworkSection}
	\begin{homeworkSection}{5}
	\problemAnswer{
	\begin{math}
	f(x) = \begin{cases}
	\frac{1}{x(x+1)} & x=1,2,...\\
	0 & otherwise

	\end{cases}
	\end{math}
	
	$E(X^\alpha) = \sum \frac{x^{\alpha-1}}{x+1}$
	For $E(X^\alpha) < \infty$, the above sequence should not be diverging.
	and hence:
	$E(X^\alpha) = \sum \frac{1}{x^{2-\alpha}+x^(1-\alpha)}$
	and hence ${x^{2-\alpha}+x^(1-\alpha)}$ should be converging $\implies$
	$\alpha \leq 1$
		
	}
	\end{homeworkSection}
\end{homeworkProblem}
\end{document}
