\documentclass{article}
\usepackage{times,amsmath,amsthm,amsfonts,eucal,graphicx, amssymb}
\setlength{\oddsidemargin}{0.25 in}
\setlength{\evensidemargin}{-0.25 in}
\setlength{\topmargin}{-0.6 in}
\setlength{\textwidth}{6.5 in}
\setlength{\textheight}{8.5 in}
\setlength{\headsep}{0.75 in}
\setlength{\parindent}{0 in}
\setlength{\parskip}{0.1 in}

\newcounter{lecnum}
\renewcommand{\thepage}{\thelecnum-\arabic{page}}
\renewcommand{\thesection}{\thelecnum.\arabic{section}}
\renewcommand{\theequation}{\thelecnum.\arabic{equation}}
\renewcommand{\thefigure}{\thelecnum.\arabic{figure}}
\renewcommand{\thetable}{\thelecnum.\arabic{table}}

\newcommand{\indep}{{\bot\negthickspace\negthickspace\bot}}
\newcommand{\notindep}{{\not\negthickspace\negthinspace{\bot\negthickspace\negthickspace\bot}}}
\newcommand{\definedtobe}{\stackrel{\Delta}{=}}
\renewcommand{\choose}[2]{{{#1}\atopwithdelims(){#2}}}
\newcommand{\argmax}[1]{{\hbox{$\underset{#1}{\mbox{argmax}}\;$}}}
\newcommand{\argmin}[1]{{\hbox{$\underset{#1}{\mbox{argmin}}\;$}}}

%
% The following macro is used to generate the header.
%
\newcommand{\lecture}[4]{
   \pagestyle{myheadings}
   \thispagestyle{plain}
   \newpage
   \setcounter{lecnum}{#1}
   \setcounter{page}{1}
   \noindent
   \begin{center}
   \framebox{
      \vbox{\vspace{2mm}
    \hbox to 6.58in { {\bf EE512~Stochastic Processes
                        \hfill University of Southern California} }
    \hbox to 6.58in { {\bf Fall 2016
                        \hfill Dept. of Electrical Engineering} }
       \vspace{4mm}
       \hbox to 6.28in { {\Large \hfill Lecture #1: #2  \hfill} }
       \vspace{2mm}
       \hbox to 6.28in { {\it Lecturer: {\it Prof: Nayyar {\tt <>}} \hfill Scribe: #3} }
      \vspace{2mm}}
   }
   \end{center}
   \markboth{Lecture #1: #2}{Lecture #1: #2}
   \vspace*{4mm}
}

%
% Convention for citations is authors' initials followed by the year.
% For example, to cite a paper by Leighton and Maggs you would type
% \cite{LM89}, and to cite a paper by Strassen you would type \cite{S69}.
% (To avoid bibliography problems, for now we redefine the \cite command.)
% Also commands that create a suitable format for the reference list.
\renewcommand{\cite}[1]{[#1]}
\def\beginrefs{\begin{list}%
        {[\arabic{equation}]}{\usecounter{equation}
         \setlength{\leftmargin}{2.0truecm}\setlength{\labelsep}{0.4truecm}%
         \setlength{\labelwidth}{1.6truecm}}}
\def\endrefs{\end{list}}
\def\bibentry#1{\item[\hbox{[#1]}]}

%Use this command for a figure; it puts a figure in wherever you want it.
%usage: \fig{NUMBER}{CAPTION}{.eps FILE TO INCLUDE}{WIDTH-IN-INCHES}
\newcommand{\fig}[4]{
			\begin{center}
	                \includegraphics[width=#4,clip=true]{#3} \\
			Figure \thelecnum.#1:~#2
			\end{center}
	}
% Use these for theorems, lemmas, proofs, etc.
\newtheorem{theorem}{Theorem}[lecnum]
\newtheorem{lemma}[theorem]{Lemma}
\newtheorem{proposition}[theorem]{Proposition}
\newtheorem{claim}[theorem]{Claim}
\newtheorem{corollary}[theorem]{Corollary}
\newtheorem{definition}[theorem]{Definition}
% \newenvironment{proof}{{\bf Proof:}}{\hfill\rule{2mm}{2mm}}

% **** IF YOU WANT TO DEFINE ADDITIONAL MACROS FOR YOURSELF, PUT THEM HERE:

\begin{document}
%FILL IN THE RIGHT INFO.
%\lecture{**LECTURE-NUMBER**}{**DATE**}{**LECTURER**}{**SCRIBE**}
\lecture{2}{Aug 25, 2016}{Saket Choudhary}
%\footnotetext{These notes are partially based on those of Nigel Mansell.}

% **** YOUR NOTES GO HERE:

% Some general latex examples and examples making use of the
% macros follow.  
%**** IN GENERAL, BE BRIEF AND COMPLETE. 
% **** THIS ENDS THE EXAMPLES. DON'T DELETE THE FOLLOWING LINE:

%\section{}
Independent RV $P(X=x,Y=y) = P(X=x)P(Y=y)$i or $P(X \in A, Y \in B) = P(X \in A)P(Y \in B)$

$E[X] = \sum_x xP(X=x)$ or $E[X] = \int xf(x)dx$

$Variance = E[(X-\mu)^2] = E[X^2]-(E[X])^2$

$Cov(X,Y) = E[(X-\mu_X)(Y-\mu_Y)] = E[XY]-E[X]E[Y]$

Independence $\implies$ Cov = 0

Cov =0 does not $\implies$ Indepennce.
Example: $X= \{-1,0,1\}$ and $Y=X^2$


Indicator variable: $E[I_x] = P(I_x=1)$

Moment generating function: $\psi(t) = E[e^{tX}]$

$\frac{d\psi(t)}{dt} = \frac{d}{dt} \int  e^{tx} f(x) dx =  \int xe^{tx} f(x) dx$
Thus, $\psi'(0) = E[X]$

$\psi''(t) = E[X^2]$

$\psi^n(t) = E[X^n]$


Example: Two gaussian Rv. $X,Y$, $Z=X+Y$
$M_Z = E[e^{tX+tY}] = E[e^{tX}]E[e^{tY}] = \psi_X(t) \psi_Y(t) = e^{\mu_xt + \frac{1}{2} \sigma_x^2} \times e^{\mu_yt+\frac{1}{2}\sigma_y^2} 
= e^{\mu_xt+\mu_yt + \frac{1}{2}(\sigma_x^2+\sigma_y^2) }$


Joint MGF: $\psi(t_1, t_2, \dots, t_n) = E[e^{\sum_i t_i X_i}]$


\section{Conditional distrbituion and conditional expectation}

$E[X_1+X_2|Y=y] = E[X_1|Y=y] + E[X_2|Y=y]$

\section{Conditional expectation as r.v}

Y - discrete r.v.

X - discerte r.v

$E[X|Y=y_1] = \sum x P(X|Y=y_1)$

$h(y_i) = E[X|Y=y_i]$ so h(Y) is like a r.v.

$E[h(y)] = \sum_i h(y_i)P(Y=y_i) = \sum E[X|Y=y_i]P(Y=y_i)  = \sum_i \sum_j x_j P(X=x_j|Y=y_i)P(Y=y_i) = \sum_i \sum_j x_j P(X=x_j, Y=y_j)
= \sum_j \sum_i x_j P(X=x_j)$

Smoothing property: $E[E[X|Y]] = EX$


$E[E[X|Y]] = \int E[X|Y=y] f(y) dy$
   
  /---\  1
 A --- B 2
  \---/  3

$P(Z=i)=p_i$
$E[D|Z=i] = d_i$ $\implies$ $E[D] = E[E[D|Z]] = \sum p_id_i$


\section{Probability inequalities}

Markov's inequality: 
$X$ is non negative RV $X \geq 0$ then for $a > 0$, $P(X \geq a) \leq E[X]/a$

Proof. $E[X] = \int_0^{\infty} xf(x)dx = \int_{0}^a xf(x)dx + \int_a^{\infty} xf(x)dx \geq \int_{a}^{\infty} xf(x) dx \geq aP(X\geq a) $
Alter. $X \geq a I_A$


Chebychev's Inequality:
$X$ is r.v. with mean $\mu$ and $var =\sigma^2$

$P(|X-\mu| \geq a) \leq \sigma^2 a^2$

Proof: $Y=(X-\mu)^2$ $P(Y \geq a^2) \leq EY/a^2$

Jensen's Ineuqlity:

Convex function: $f(\lambda x + (1-\lambda)y) \leq \lambda f(x) + (1-\lambda) f(y)$

If $f(x)$ is convex then $E[f(x)] \geq f(E[X])$

 


\end{document}


