%%%%%%%%%%%%%%%%%%%%%%%%%%%%%%%%%%%%%%%%%
% Structured General Purpose Assignment
% LaTeX Template
%
% This template has been downloaded from:
% http://www.latextemplates.com
%
% Original author:
% Ted Pavlic (http://www.tedpavlic.com)
%
% Note:
% The \lipsum[#] commands throughout this template generate dummy text
% to fill the template out. These commands should all be removed when 
% writing assignment content.
%
%%%%%%%%%%%%%%%%%%%%%%%%%%%%%%%%%%%%%%%%%

%----------------------------------------------------------------------------------------
%	PACKAGES AND OTHER DOCUMENT CONFIGURATIONS
%----------------------------------------------------------------------------------------

\documentclass{article}

\usepackage{fancyhdr} % Required for custom headers
\usepackage{lastpage} % Required to determine the last page for the footer
\usepackage{extramarks} % Required for headers and footers
\usepackage{graphicx} % Required to insert images
\usepackage{latexsym}
\usepackage{mathtools}

\usepackage{lipsum} % Used for inserting dummy 'Lorem ipsum' text into the template
%\usepackage[]{algorithm2e}
%\usepackage{algorithmicx}
%\usepackage{algorithm}
%\usepackage{algorithm}
%\usepackage{algorithmic}
%\usepackage{algpseudocode}
%\usepackage{algcompatible}


\usepackage{algorithm}
\usepackage{algorithmic}
%\usepackage{algorithmicx}


%\usepackage{algpseudocode}

%\usepackage{algpseudocode}

%\usepackage[noend]{algpseudocode}
\renewcommand{\algorithmicrequire}{\textbf{Input:}}
\renewcommand{\algorithmicensure}{\textbf{Output:}}
\newcommand{\algorithmicbreak}{\textbf{break}}
\newcommand{\algorithmicgiven}{\textbf{Given:}}
\newcommand{\BREAK}{\STATE \algorithmicbreak}
\newcommand{\GIVEN}{\STATEx \algorithmicgiven}
%\def\NoNumber#1{{\def\alglinenumber##1{}\State #1}\addtocounter{ALG@line}{-1}}

\usepackage{amsmath}
%\usepackage{multline}

% Margins
\topmargin=-0.45in
\evensidemargin=0in
\oddsidemargin=0in
\textwidth=6.5in
\textheight=9.0in
\headsep=0.25in 

\linespread{1.1} % Line spacing

% Set up the header and footer
\pagestyle{fancy}
\lhead{\hmwkAuthorName} % Top left header
\chead{\hmwkClass\ : \hmwkTitle} % Top center header
\rhead{\firstxmark} % Top right header
\lfoot{\lastxmark} % Bottom left footer
\cfoot{} % Bottom center footer
\rfoot{Page\ \thepage\ of\ \pageref{LastPage}} % Bottom right footer
\renewcommand\headrulewidth{0.4pt} % Size of the header rule
\renewcommand\footrulewidth{0.4pt} % Size of the footer rule

\setlength\parindent{0pt} % Removes all indentation from paragraphs

%----------------------------------------------------------------------------------------
%	DOCUMENT STRUCTURE COMMANDS
%	Skip this unless you know what you're doing
%----------------------------------------------------------------------------------------

% Header and footer for when a page split occurs within a problem environment
\newcommand{\enterProblemHeader}[1]{
\nobreak\extramarks{#1}{#1 continued on next page\ldots}\nobreak
\nobreak\extramarks{#1 (continued)}{#1 continued on next page\ldots}\nobreak
}

% Header and footer for when a page split occurs between problem environments
\newcommand{\exitProblemHeader}[1]{
\nobreak\extramarks{#1 (continued)}{#1 continued on next page\ldots}\nobreak
\nobreak\extramarks{#1}{}\nobreak
}

\setcounter{secnumdepth}{0} % Removes default section numbers
\newcounter{homeworkProblemCounter} % Creates a counter to keep track of the number of problems

\newcommand{\homeworkProblemName}{}
\newenvironment{homeworkProblem}[1][Problem \arabic{homeworkProblemCounter}]{ % Makes a new environment called homeworkProblem which takes 1 argument (custom name) but the default is "Problem #"
\stepcounter{homeworkProblemCounter} % Increase counter for number of problems
\renewcommand{\homeworkProblemName}{#1} % Assign \homeworkProblemName the name of the problem
\section{\homeworkProblemName} % Make a section in the document with the custom problem count
\enterProblemHeader{\homeworkProblemName} % Header and footer within the environment
}{
\exitProblemHeader{\homeworkProblemName} % Header and footer after the environment
}

\newcommand{\problemAnswer}[1]{ % Defines the problem answer command with the content as the only argument
\noindent\framebox[\columnwidth][c]{\begin{minipage}{0.98\columnwidth}#1\end{minipage}} % Makes the box around the problem answer and puts the content inside
}

\newcommand{\homeworkSectionName}{}
\newenvironment{homeworkSection}[1]{ % New environment for sections within homework problems, takes 1 argument - the name of the section
\renewcommand{\homeworkSectionName}{#1} % Assign \homeworkSectionName to the name of the section from the environment argument
\subsection{\homeworkSectionName} % Make a subsection with the custom name of the subsection
\enterProblemHeader{\homeworkProblemName\ [\homeworkSectionName]} % Header and footer within the environment
}{
\enterProblemHeader{\homeworkProblemName} % Header and footer after the environment
}
   
%----------------------------------------------------------------------------------------
%	NAME AND CLASS SECTION
%----------------------------------------------------------------------------------------
\DeclarePairedDelimiter\ceil{\lceil}{\rceil}
\DeclarePairedDelimiter\floor{\lfloor}{\rfloor}
\newcommand{\hmwkTitle}{Homework\ \# 1 } % Assignment title
\newcommand{\hmwkDueDate}{Tuesday,\ March \ 10,\ 2015} % Due date
\newcommand{\hmwkClass}{MATH-578A} % Course/class
\newcommand{\hmwkClassTime}{11:00am} % Class/lecture time
\newcommand{\hmwkAuthorName}{Saket Choudhary} % Your name
\newcommand{\hmwkAuthorID}{2170058637} % Teacher/lecturer
%----------------------------------------------------------------------------------------
%	TITLE PAGE
%----------------------------------------------------------------------------------------

\title{
\vspace{2in}
\textmd{\textbf{\hmwkClass:\ \hmwkTitle}}\\
\normalsize\vspace{0.1in}\small{Due\ on\ \hmwkDueDate}\\
%\vspace{0.1in}\large{\textit{\hmwkClassTime}}
\vspace{3in}
}

\author{\textbf{\hmwkAuthorName} \\
	\textbf{\hmwkAuthorID}
	}
\date{} % Insert date here if you want it to appear below your name

%----------------------------------------------------------------------------------------

\begin{document}

\maketitle

%----------------------------------------------------------------------------------------
%	TABLE OF CONTENTS
%----------------------------------------------------------------------------------------

%\setcounter{tocdepth}{1} % Uncomment this line if you don't want subsections listed in the ToC

\newpage
\tableofcontents
\newpage




\begin{homeworkSection}{Question \# 1} % Section within problem

\problemAnswer{
	Definition: $SP(i)$ = max $k < i$ such that $P[1..k] = P[i-k+1..i]$\\
	
	String: CACGCAACGA
	
	NOTE: Iteration indexed at 0. So SP[0] = 0(By Definition) and hence the loop iterations start from 1 and go till n-1=9 ;
	
	
	\centering

		\begin{tabular}{|c|c|c|c|}
			\hline Iteration  & $SP[i]$  & All other SP values examined  & \# of times inner while loop executed \\ 
			\hline 1 & 0 & -  & 0  \\ 
			\hline 2 & 1 & - & 0  \\ 
			\hline 3 & 0 & $SP[0]$  & 1  \\ 
			\hline 4 & 1 & -  & 0  \\ 
			\hline 5 & 2  & -  & 0  \\ 
			\hline 6 & 1 & $SP[0]$ & 1  \\ 
			\hline 7 & 1 & - & 0  \\ 
			\hline 8 & 1 & $SP[0]$ & 1  \\ 
			\hline 9 & 0 & - & 0  \\ 
			\hline 
		\end{tabular} 

	
}

\end{homeworkSection}

\begin{homeworkSection}{Question \#2}
	\problemAnswer{
		$S = CACGGCACGG$
		
		NOTE: Indexing starts from 0. By definition $Z[0]=|S|=10 $
		
		The 'cases' are choosen out of:
		
		Case 1. $k>r$. The index for which $Z$ value is being calculated is greater than the right most ending of all the previous(till $k-1$) $Z$ boxes calculated. Since this is as good as having no pre-calculated $Z$ scores, this case leads to explicit character comparison(starting at $k$) till a mismatch occurs.
		
		Case 2.	$k\leq r$ The current position $k$ is inside one of the previoulsy calculated $Z$ boxes. Hence there exists a correpsonding position $k'=k-l+1$ where $l$ is the left ending of the $Z$ box with it's right ending at $r$, such that $S[k`]=S[k]$. There is a corresponding one to one match for $S[k'..r-l+1]$ with $S[k..r]$ and we define this to be another box $\beta$ with $\beta=r-k+1$ and hence  $Z[k]$ can be caculated utilising this information. 
		
		The following three cases arise:
			
			Case 2a. $Z_k' < |\beta|$ 
				So starting at $k'$ the length of largest substring that matches the prefix of S is less than size of that $\beta$ box starting at $k'$. Since this $\beta$ box appears starting from $k$ too and $Z_k'<|\beta|$ implies $Z_k = Z_k'$. Total comparisons:
				\begin{enumerate}
					\item Comparison: $k\leq r$
					\item Assignment/Calculation: $k'=k-l+1$
					\item Lookup: $Z_k'$
					\item Assignment/Caculation: $|\beta| = r-k+1$
					\item Comparison: $Z_k' < |\beta|$
					\item Assigment: $Z_k = Z_k'$
				\end{enumerate}
				No character comparisons are involved.
				
			Case 2b. $Z_k' > |\beta|$ 
				So the substring starting at $k'$ matches a prefix of S and has length equal to the $\beta$ box. If we call the box with it's leftmost end=$l$ and rightmost end=$r$ as $\alpha$, then we know that 
				$S[r+1]\neq S[|\alpha|+1]$ otherwise $\alpha$ would not have been the largest such box. Thus, $Z_k = \beta$
				Thus no character comparisons involved in this case too.
				
				The comparisons involved: 
				\begin{enumerate}
					\item Comparison: $k\leq r$
					\item Assignment/Calculation: $k'=k-l+1$
					\item Lookup: $Z_k'$
					\item Assignment/Caculation: $|\beta| = r-k+1$
					\item Comparison: $Z_k' > |\beta|$
					\item Assigment: $Z_k = Z_k'$
				\end{enumerate}
			
				
				 
			Case 2c. $Z_k' = |\beta|$
			
				The substring starting at $k$ might have a matching prefix in S, and hence explicit character 
				comparions are required from $r+1$ to $q\geq r+1$ till the first mismatch occurs. These iterations
				are bound by  $O(|S|)$ since the maximum possible mismatches are $O(|S|)$. 
				
				The comparisons involved: 
				\begin{enumerate}
					\item Comparison: $k\leq r$
					\item Assignment/Calculation: $k'=k-l+1$
					\item Lookup: $Z_k'$
					\item Assignment/Caculation: $|\beta| = r-k+1$
					\item Comparison: $Z_k' == |\beta|$
					\item Iteration for explicit character comparison: $while(Z_[r+1]==Z[\alpha+1])...$, bounded by O(|S|)
				\end{enumerate}
				
		
		\begin{tabular}{|c|c|c|c|c|}
			\hline $i$ & $Z[i]$  & $l_i$  &  $r_i$ & Case  \\ 
			\hline 1 & 2 & 1 & 0 & 1 \\ 
			\hline 2 & 3 & 3 & 1 & 1 \\ 
			\hline 3 & 4 & 3 & 0 & 1 \\ 
			\hline 4 & 5 & 4 & 0 & 1 \\ 
			\hline 5 & 6 & 10 & 5 & 1 \\ 
			\hline 6 & 6 & 10 & 0 & 2a \\ 
			\hline 7 & 6 & 10 & 1 & 2a \\ 
			\hline 8 & 6 & 10 & 0 & 2a \\ 
			\hline 9 & 6 & 10 & 0 & 2a \\ 

			\hline 
		\end{tabular} 
		}
\end{homeworkSection}

\begin{homeworkSection}{Question \# 3}
	%\problemAnswer{
		
	
		

	
	%\begin{algorithm}
	%	\caption{My algorithm}\label{euclid}

	%	\begin{algorithmic}[1]
	%	   \Statex {}
		  %\STATEx \textbf{Input:}
		  %\STATEx \textbf{Output:} Determine if $\alpha$ is a circular rotation of $\beta$
		%\end{algorithmic}
	%\end{algorithm}
	
	\begin{algorithm}                      % enter the algorithm environment
		\caption{Find circular rotation}          % give the algorithm a caption
		\label{alg1}                           % and a label for \ref{} commands later in the document
		\begin{algorithmic}                    % enter the algorithmic environment
			%\STATE
			\REQUIRE {Two string $\alpha$, $\beta$ and a linear time algorithm say $Z$ algorithm to solve exact string matching problem in linear time} \\ 
			\ENSURE Determine if $\alpha$ is a circular rotation of $\beta$
			\STATE $S \Leftarrow \alpha\$\beta\beta$
			\STATE $Z_{values} \Leftarrow  Z(S)$
			\STATE $N \Leftarrow |S|$
			
			\WHILE{$N \neq 3|S|+1$}
				\IF{$Z_{values}[i]\geq |\alpha|$}
					\STATE {return $true$}
				\ENDIF
			\ENDWHILE
			\STATE return $false$
		\end{algorithmic}
	\end{algorithm}
	
		%	}
	
\end{homeworkSection}

\begin{homeworkSection}{Question \# 4}
	\problemAnswer{
		Case 2b of $Z$ algorithm can be split into following sub cases:
			Case 2b $Z_k' > |\beta|$ \\
			Case 2c $Z_k' = |\beta|$\\
		Let $r$ denote the right most edge of the $Z$ box(call it $\alpha$)
		such that $k\leq r$. $l$ denotes the left most edge of this $Z$ box.
		When $Z_k' > \beta$, let $S[r+1]=X$ Let $k'=k-l+1$ denote the cooresponding position in the prefix of S,
		such that $S[1...k']$ matches $S[l...k]$  and also $S[1...r-l+1]$ matches $S[l..r]$
		
		Consider $r'=r-l+1$ let $S[r'+1]=Y$, then $X\neq Y$, else the $Z$ box would have been longer than $|\alpha|$,
		 contrary to the definition. 
		 
		 Now consider $Z_k'>|\beta|$ $\implies$ there exists a matching prefix of S for substring starting at $k'$ which also implies
		 that $S[Z_k'+1]=S[r'+1] = Y$ because $Z_k'$ will be at least $|\beta| +1$ in size.
		
		Since $X \neq Y$, $Z_k = |\beta|$, because $|\beta|$ is the length of longest matching prefix given $S[|\beta|+1] = S[r'+1] \neq S[r+1]$ 
			
			Question 7.
			
			No. there is no extra speedup if we take into consideration all comparisons.
			
			Case 2a, 2b approach: Comparison required: 1 character comparison on failore of conditional check $Z_k < |\beta|$
			Case 2a,2b,2c appraich: Comparison required: 1 integer comparison $Z_k == |\beta|$
			 
		}
\end{homeworkSection}

\begin{homeworkSection}{Question \# 5}
	%\problemAnswer{
		Solution: 
		1. The first occurence of parameters is very flexible, since they can be made to match to any other parameter.
		2. Any parameter appearing more than once arises a constraint
		
		Approach:
		
		\begin{algorithm}                      % enter the algorithm environment
			\caption{Find multisets}          % give the algorithm a caption
			\label{alg2}                           % and a label for \ref{} commands later in the document
			\begin{algorithmic}  
				\REQUIRE {String P, T} 
				\ENSURE {Find all p-matches of P in T in $O(P+T)$}
				\STATE{$m \Leftarrow |P|$}
				\STATE{$n \Leftarrow |T|$}
				\STATE{$lastParameterMap$}
				\STATE{$parametersTotal$}
				%\STATE{$numSymbols \Leftarrow []$}
				\STATE{$P' \Leftarrow null$}
				\STATE{$S=P\$T$}
				
				\FOR{$i \Leftarrow 1\ to\ m+n$}
					\IF{$isParameter(S[i])$}
						\IF{$P[i]\ in\ lastParameterMap$}
							\STATE{$parametersTotal[i] \Leftarrow parametersTotal[i-1]+1$}
							\STATE {$lastOccurenceAt \Leftarrow lastParameterMap[P[i]]$}
							\STATE {$numParamsFromLastOccurnce \Leftarrow parametersTotal[i]-parametersTotal[lastOccuredAt]$}
							\STATE {$P' \Leftarrow concat(P', numFromLastOccurence)$}
						\ELSE
							\STATE $lastParameterMap[P[i]]=i$
						\ENDIF
					\ELSE
						\STATE $P' \Leftarrow concat(P', S[i])$
					\ENDIF
				\ENDFOR
				
				
				$z_values \Leftarrow ZAlgorithm(P')$
				\STATE return all positions where $z_values \geq m$ 
				%$return [z_values[i] for z_values[i]$
			\end{algorithmic}
		\end{algorithm}
				
		Example: XYabCaCXZddbW
		%}
		
		
\end{homeworkSection}

\begin{homeworkSection}{Question \# 6}
\begin{algorithm}                      % enter the algorithm environment
	\caption{Find multisets}          % give the algorithm a caption
	\label{alg2}                           % and a label for \ref{} commands later in the document
	\begin{algorithmic}  
		\REQUIRE {String S, T} 
		\ENSURE {Find all substrings of T that are formed by characters of S}
		\STATE $patternMap \Leftarrow CreateFrequencyOfCharacters(S)$
		\STATE $longestSubstringPossible \Leftarrow []$
		\STATE $m \Leftarrow |S|$
		\STATE $n \Leftarrow |T|$
		\STATE $i \Leftarrow 2$
		\STATE $sum \Leftarrow 0$
		%\STATE $sequenceMap \LeftArrow CreateFrequencyOFCharaceters(S[1..m])$
		%\STATE $previousChar \LeftArrow  S[1]$
		\WHILE{$i\leq m$}
			\IF{$S[i] in sequenceMap.keys()$}
				\STATE{$sequenceMap[S[i]] \Leftarrow sequenceMap[S[i]]+1$}
			\ENDIF
				
			\IF{$patternMap[S[i]]>=1 and sequenceMap[S[i]] < sequenceMap[S[i]] $}
				\STATE{$sum \Leftarrow sum+1$}
				\ELSE
					\STATE{$sum \Leftarrow sum-1$}
			\ENDIF
			
			\STATE{$i \Leftarrow i+1$}
			
			\FOR{$i \Leftarrow 2\ to\ n-m$}
			\STATE {$next \Leftarrow S[i+m]$}
			\IF{$sequenceMap[previous] > patternMap[previous] $}
			\STATE{$sum \Leftarrow sum+1$}
			\ELSE
			\STATE{$sum \Leftarrow sum-1$}
			\ENDIF
			
			\IF{$patternMap[next]>=1 and sequenceMap[next]] < sequenceMap[S[i]] $}
			\STATE{$sum \Leftarrow sum+1$}
			\ELSE
			\STATE{$sum \Leftarrow sum-1$}
			\ENDIF
			
			\STATE{$longestSubstringPossible[i]=sum$}
			\STATE{$previous \Leftarrow S[i]$}
			
			\ENDFOR
			
			
		\ENDWHILE
		
		\STATE{$longestSubstringPossible[1]=sum$}
		\STATE{$previous \Leftarrow S[1]$}
	
	\end{algorithmic}
\end{algorithm}
\end{homeworkSection}


\begin{homeworkSection}{Question \# 7}
	%\problemAnswer{
			\begin{algorithm}                      % enter the algorithm environment
				\caption{Find occurence of P in T in linear time using sp values}          % give the algorithm a caption
				\label{alg3}                           % and a label for \ref{} commands later in the document
				\begin{algorithmic}                    % enter the algorithmic environment
					%\STATE
					\REQUIRE {Strings P and T} \\ 
					\ENSURE Find all occurences of P in T in linear time using $sp$ values
					\STATE $S \Leftarrow PT$
					\STATE $sp_{values} \Leftarrow  SPCalculator(S)$
					\STATE $N \Leftarrow |S|$
					\STATE $P_{occurences} = []$
					\WHILE{$N \geq |P|+1$}
					\IF{$sp_{values}[i]\geq |P|$}
						\IF{$S[N]==P[|P|] and S[N-|P|]==P[1]$}
							\STATE $P_{occurences}.push(i)$
							\STATE $ N \Leftarrow N-|P|$
						\ELSE
							\STATE $N \Leftrightarrow N-1$
						\ENDIF
					\ELSE
						\STATE $N \Leftrightarrow N-1$
					\ENDIF
					\ENDWHILE
					\STATE return $P_{occurences}$
				\end{algorithmic}
			\end{algorithm}
	%	}
	
\end{homeworkSection}


\end{document}
