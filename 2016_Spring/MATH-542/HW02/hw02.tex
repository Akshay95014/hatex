\documentclass[a4paper]{article}

\usepackage[english]{babel}
\usepackage[utf8x]{inputenc}
\usepackage{amsmath}
\usepackage{graphicx}
\usepackage[colorinlistoftodos]{todonotes}

\title{MATH 542 Homework 2}
\author{Saket Choudhary\\skchoudh@usc.edu}

\begin{document}
\maketitle 
\section*{Problem 1}
$$ A=\begin{pmatrix} A_{11} & A_{12}\\ 
A_{21} & A_{22}
\end{pmatrix} $$Consider $C=A^{-1}$ so that $CA=I$

$$
C = \begin{pmatrix} C_{11} & C_{12}\\ C_{21} & C_{22}\end{pmatrix}
$$
\begin{align*}
CA &= \begin{pmatrix} A_{11}C_{11}+A_{12}C_{21} & A_{11}C_{12}+A_{12}C_{22}\\  
A_{21}C_{11}+A_{22}C_{21} & A_{21}C_{12}+A_{22}C_{22}
\end{pmatrix}\\
&= \begin{pmatrix} I_k &  O_{n-k} \\ O_{n-k} & I_{k} \end{pmatrix}
\end{align*}
Thus,
\begin{align*}
A_{11}C_{11}+A_{12}C_{21} &=  I_k\\
A_{11}C_{12}+A_{12}C_{22} &= O_{n-k}\\
A_{21}C_{11}+A_{22}C_{21} &= O_{n-k}\\
A_{21}C_{12}+A_{22}C_{22}&=I_k 
\end{align*}

Thus,

\begin{align*}
C_{11}&= -A_{11}^{-1}(I-A_{12}C_{21})\\
C_{12} &= -A_{11}^{-1}A_{12}C_{22}\\ 
C_{21} &= -A_{22}^{-1}A_{21}C_{11}\\
C_{22}&=A_{22}^{-1}(I-A_{21}C_{12})
\end{align*}

Substituting for $C_{12}$ and $C_{21}$ :


\begin{align*}
A_{11}C_{11}-A_{12}A_{22}^{-1}A_{21}C_{11} &= I_k\\
\implies C_{11} &= (A_{11}-A_{12}A_{22}^{-1}A_{21})^{-1}
\end{align*}

Similarly,
\begin{align*}
-A_{21}A_{11}^{-1}A_{12}C_{22}+A_{22}C_{22} &= I_k\\
C_{22} &=  (A_{22}-A_{21}A_{11}^{-1}A_{12})^{-1}\\
&=B
\end{align*}
Thus, 
\begin{align*}
B &= (A_{22}-A_{21}A_{11}^{-1}A_{12})\\
C_{11} &=(A_{11}-A_{12}A_{22}^{-1}A_{21})^{-1} \\
C_{21}&= -A_{22}^{-1}A_{21}(A_{11}-A_{12}A_{22}^{-1}A_{21})^{-1}\\
C_{22}&= (A_{22}-A_{21}A_{11}^{-1}A_{12})^{-1} = B^{-1}\\
C_{12} &= -A_{11}^{-1}A_{12}C_{22}\\
\end{align*}

Now we use Woodbury matrix identity $(A+UCV )^{-1}= A^{-1}-A^{-1}U(C^{-1}+VA^{-1}U)^{-1}VA^{-1}$

Using which, \begin{align*}
C_{11} &= (A_{11}-A_{12}A_{22}^{-1}A_{21})^{-1}\\ 
&= A_{11}^{-1}-A_{11}^{-1}(-A_{12})(A_{22}^{-1} + -A_{21}A_{11}^{-1}A_{12} )^{-1}A_{21}A_{11}^{-1}\\  
&= A_{11}+A_{11}^{-1}A_{12}B^{-1}A_{21}A_{11}^{-1}
\end{align*}

\begin{align*}C_{12} &= -A_{11}^{-1}A_{12}C_{22}\\
&= -A_{12}^{-1}A_{12}B^{-1} \end{align*}

\subsection*{Problem 1.b}
For this part we substitute $k=1$  
\begin{align*}
B_{1 \times 1} &= a_{22}-a_{12}'A_{11}^{-1}a_{12}\\
C_{11} &= A_{11}+A_{11}a_{12}B^{-1}a_{21}'A_{11}^{-1}\\
C_{12}&=-A_{11}^{-1}a_{12}B^{-1}\\ 
C_{21}&= -B^{-1}a_{12}'A_{11}^{-1}\\
C_{22}&=B^{-1}\\
\end{align*}


\section*{Problem 2}

\subsection*{Problem 2.a}

$X'X = \begin{pmatrix}2J & J  & J \\ J & J & 0 \\ J & 0 & J  \end{pmatrix} = J \begin{pmatrix} 2 & 1 &1\\ 1 & 1 & 0 \\ 1 & 0 & 1 \end{pmatrix}$

\subsection*{Problem 2.b}
$X'X$ is not invertible( Perform $C1-C3$)

$(X'X) \times (X'X)_1^{-} =J \begin{pmatrix} 2 & 1 &1\\ 1 & 1 & 0 \\ 1 & 0 & 1 \end{pmatrix} \times XX_1^- = \begin{pmatrix}1 & 0 & 0\\ 0 & 1 & 0\\ 1& -1 & 0 \end{pmatrix}$ 

Also, $(X'X) \times (X'X)_2^{-} =J \begin{pmatrix} 2 & 1 &1\\ 1 & 1 & 0 \\ 1 & 0 & 1 \end{pmatrix} \times XX_2^- = \begin{pmatrix} 0 & 0 & 1\\ 0 & 1 & 0\\ 0 & 0 & 1 \end{pmatrix}$ 
\subsection*{Problem 2.c}

$P = X(X'X)_1^-X' = \begin{pmatrix}2 & 1 & 1\\ 1 & 1 & 0\\ 1 & 0 & 1 \end{pmatrix}=X(X'X)_2^-X'= X'X$
and thus P is unique

\section*{Problem 3}

$$
X = \begin{pmatrix} 
1 & -1 & 1\\
-1 & 0 & 2\\
1 & 1 & 1\\
\end{pmatrix}
$$Consider $X^T = \begin{pmatrix} 
1 & -1 & 1\\
-1 & 0 & 1\\
1 & 2 & 1\\
\end{pmatrix}$

Finding the inverse gives $X^{-1} =   \begin{pmatrix} 
1 & -1 & 1\\
-1 & 0 & 2\\
1 & 1 & 1\\
\end{pmatrix}$
In fact $X^TX = \begin{pmatrix}3 & 0 & 0\\ 
0 & 2 & 0\\
0 & 0 & 6\\
\end{pmatrix}$and $XX^T = \begin{pmatrix}3 & 1 & 1\\ 1 & 5 & 1\\ 1 & 1 &3
\end{pmatrix}$

Clearly $X^T \neq X^{-1}$ and this leads us to conclude that $X$ is \textbf{not} an orthonormal matrix.

$$
C = \begin{pmatrix} 1/\sqrt{3} & -1/\sqrt{2} & 1/\sqrt{6}\\
-1/\sqrt{3} & 0/\sqrt{2} & 2/\sqrt{6}\\
1/\sqrt{3} & 1/\sqrt{2} & 1/\sqrt{6}
\end{pmatrix}
$$
$$
C^T = \begin{pmatrix} 1/\sqrt{3} & -1/\sqrt{3} & 1/\sqrt{3}\\
-1/\sqrt{2} & 0/\sqrt{2} & 1/\sqrt{2}\\
1/\sqrt{6} & 2/\sqrt{6} & 1/\sqrt{6}
\end{pmatrix}
$$
Thus, $$CC^T = C^TC = I_3 \text{ (Skipped calculations, did in R)}$$ 








\section*{Problem 4}

\subsection*{Problem 4.a}

Using the property $\det(XY) = \det(X) \times \det(Y)$  we have $\det(A^2) =  \det(A)\times \det(A) = (\det(A))^2$

\subsection{Problem 4.b}

$A = P\Lambda P^T$ such that $\Lambda$ is a diagonal matrix with its diagonals as the eigen values and $P$ is an orthonormal matrix thus, $PP^T=I$ and $\det(P) = +1\ or\-1$ 

$\det(A) = \det(P\Lambda P^T)= \det(P) \det(\Lambda) \det(P^T) = \det(\Lambda) \det(P)^2 = \det(\Lambda) = \prod \lambda_i$ 
Using $tr(AB) = tr(BA)$
$tr(A) = tr(P\Lambda P^T) = tr(\lambda P^T P) = tr(\Lambda \times I) = tr(\Lambda) = \sum \lambda_i$ 


\end{document}