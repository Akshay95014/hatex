%%%%%%%%%%%%%%%%%%%%%%%%%%%%%%%%%%%%%%%%%
% Structured General Purpose Assignment
% LaTeX Template
%
% This template has been downloaded from:
% http://www.latextemplates.com
%
% Original author:
% Ted Pavlic (http://www.tedpavlic.com)
%
% Note:
% The \lipsum[#] commands throughout this template generate dummy text
% to fill the template out. These commands should all be removed when 
% writing assignment content.
%
%%%%%%%%%%%%%%%%%%%%%%%%%%%%%%%%%%%%%%%%%

%----------------------------------------------------------------------------------------
%	PACKAGES AND OTHER DOCUMENT CONFIGURATIONS
%----------------------------------------------------------------------------------------

\documentclass{article}

\usepackage{fancyhdr} % Required for custom headers
\usepackage{lastpage} % Required to determine the last page for the footer
\usepackage{extramarks} % Required for headers and footers
\usepackage{graphicx} % Required to insert images
\usepackage{latexsym}

\usepackage{lipsum} % Used for inserting dummy 'Lorem ipsum' text into the template

\usepackage{amsmath}
%\usepackage{multline}

% Margins
\topmargin=-0.45in
\evensidemargin=0in
\oddsidemargin=0in
\textwidth=6.5in
\textheight=9.0in
\headsep=0.25in 

\linespread{1.1} % Line spacing

% Set up the header and footer
\pagestyle{fancy}
\lhead{\hmwkAuthorName} % Top left header
\chead{\hmwkClass\ : \hmwkTitle} % Top center header
\rhead{\firstxmark} % Top right header
\lfoot{\lastxmark} % Bottom left footer
\cfoot{} % Bottom center footer
\rfoot{Page\ \thepage\ of\ \pageref{LastPage}} % Bottom right footer
\renewcommand\headrulewidth{0.4pt} % Size of the header rule
\renewcommand\footrulewidth{0.4pt} % Size of the footer rule

\setlength\parindent{0pt} % Removes all indentation from paragraphs

%----------------------------------------------------------------------------------------
%	DOCUMENT STRUCTURE COMMANDS
%	Skip this unless you know what you're doing
%----------------------------------------------------------------------------------------

% Header and footer for when a page split occurs within a problem environment
\newcommand{\enterProblemHeader}[1]{
\nobreak\extramarks{#1}{#1 continued on next page\ldots}\nobreak
\nobreak\extramarks{#1 (continued)}{#1 continued on next page\ldots}\nobreak
}

% Header and footer for when a page split occurs between problem environments
\newcommand{\exitProblemHeader}[1]{
\nobreak\extramarks{#1 (continued)}{#1 continued on next page\ldots}\nobreak
\nobreak\extramarks{#1}{}\nobreak
}

\setcounter{secnumdepth}{0} % Removes default section numbers
\newcounter{homeworkProblemCounter} % Creates a counter to keep track of the number of problems

\newcommand{\homeworkProblemName}{}
\newenvironment{homeworkProblem}[1][Problem \arabic{homeworkProblemCounter}]{ % Makes a new environment called homeworkProblem which takes 1 argument (custom name) but the default is "Problem #"
\stepcounter{homeworkProblemCounter} % Increase counter for number of problems
\renewcommand{\homeworkProblemName}{#1} % Assign \homeworkProblemName the name of the problem
\section{\homeworkProblemName} % Make a section in the document with the custom problem count
\enterProblemHeader{\homeworkProblemName} % Header and footer within the environment
}{
\exitProblemHeader{\homeworkProblemName} % Header and footer after the environment
}

\newcommand{\problemAnswer}[1]{ % Defines the problem answer command with the content as the only argument
\noindent\framebox[\columnwidth][c]{\begin{minipage}{0.98\columnwidth}#1\end{minipage}} % Makes the box around the problem answer and puts the content inside
}

\newcommand{\homeworkSectionName}{}
\newenvironment{homeworkSection}[1]{ % New environment for sections within homework problems, takes 1 argument - the name of the section
\renewcommand{\homeworkSectionName}{#1} % Assign \homeworkSectionName to the name of the section from the environment argument
\subsection{\homeworkSectionName} % Make a subsection with the custom name of the subsection
\enterProblemHeader{\homeworkProblemName\ [\homeworkSectionName]} % Header and footer within the environment
}{
\enterProblemHeader{\homeworkProblemName} % Header and footer after the environment
}
   
%----------------------------------------------------------------------------------------
%	NAME AND CLASS SECTION
%----------------------------------------------------------------------------------------

\newcommand{\hmwkTitle}{Homework\ \# 5 } % Assignment title
\newcommand{\hmwkDueDate}{Friday,\ September \ 26,\ 2014} % Due date
\newcommand{\hmwkClass}{MATH-505A} % Course/class
\newcommand{\hmwkClassTime}{10:30am} % Class/lecture time
\newcommand{\hmwkAuthorName}{Saket Choudhary} % Your name
\newcommand{\hmwkAuthorID}{2170058637} % Teacher/lecturer
%----------------------------------------------------------------------------------------
%	TITLE PAGE
%----------------------------------------------------------------------------------------

\title{
\vspace{2in}
\textmd{\textbf{\hmwkClass:\ \hmwkTitle}}\\
\normalsize\vspace{0.1in}\small{Due\ on\ \hmwkDueDate}\\
%\vspace{0.1in}\large{\textit{\hmwkClassTime}}
\vspace{3in}
}

\author{\textbf{\hmwkAuthorName} \\
	\textbf{\hmwkAuthorID}
	}
\date{} % Insert date here if you want it to appear below your name

%----------------------------------------------------------------------------------------

\begin{document}

\maketitle

%----------------------------------------------------------------------------------------
%	TABLE OF CONTENTS
%----------------------------------------------------------------------------------------

%\setcounter{tocdepth}{1} % Uncomment this line if you don't want subsections listed in the ToC

\newpage
\tableofcontents
\newpage

\begin{homeworkProblem}[Exercise \# 2.7] % Custom section title


	\begin{homeworkSection}{(1)} % Section within problem
		
		\problemAnswer{ % Answer
		Coin toss shows head with probability = $p$
		\textbf{To Find:} $P(X > m)$
		
		By definition $P(X>m) = 1 - P(X<m)$
		
		$P(X<m)$ = $P(head\ comes\ on\ toss\ 1) + P(head\ comes\ on\ toss\ 2) + .. + P(head\ comes\ on\ toss\ m-1) =  p + (1-p)p + (1-p)^2p + ... + (1-p)^(m-1)p = p(\frac{1-(1-p)^{m-1+1}}{1-(1-p)}) = 1-(1-p)^m$
		
		Thus $P(X>m) = 1 - 1-(1-p)^m = (1-p)^m$
		
		The distribution function of $X$ is given by: \\
		\begin{math}
			F(x) = 	
			\begin{cases}
	  			  1 - (1-p)^x & x \leq 0 \\
				  0 & otherwise	 
			\end{cases}
		\end{math}
		
		given that $x$ takes only integer values$[$$X$ is a discrete random variable$]$
}
	\end{homeworkSection}
	
	\begin{homeworkSection}{(2 a)}
	\problemAnswer{
	$X$ is a  random variable.
	Let $\Omega$= $\{x1, x2, ...., x_n\}$
	Define a indicator random variables $I_j(x_j)$ such that: \\
	\begin{math}
	I_j(x) = 
	\begin{cases}
		1 & if x = x_j\\
		0 & otherwise
	
	\end{cases}
	\end{math}
	
	Thus $X$ can be now expressed as : \\
	\begin{math}
	X = \sum_{j=1}^{n}x_j I_j(x_j)
	\end{math}
	}
	\end{homeworkSection}
	
	\begin{homeworkSection}{(2 c)}
	\problemAnswer{
		Consider the set of events $X=\{X_1, X_2, X_3, ... , X_n\}$
		such that $X_1(\omega) < X_2(\omega) < ... < X_m(\omega) \forall \omega in \Omega$
		
		In order to prove if $X$ is a random variable  we consider $\{X(\omega) \leq x \}$
		which is equivalent to $\{X_i(\omega) \leq x \} \forall i \in [1, n]]$
		which is equivalent to $\{min X(\omega) \leq x\}$ where $min X$ refers to $min(X_i)$
		
	}
	\end{homeworkSection}
	
	\begin{homeworkSection}{(4)}
	\problemAnswer{
	\begin{math}
		F(x) = \begin{cases}
		0 & if x < 0 \\
		\frac{1}{2}x & if 0 < x < 2, \\
		1 & if x > 2
		\end{cases}
	\end{math}
	
	Thus  $f(x) = F'(x)$ : \\
	\begin{math}
			f(x) = \begin{cases}
			0 & if x < 0 \\
			\frac{1}{2} & if 0 < x < 2, \\
			0 & if x > 2
			\end{cases}
		\end{math}
		
	
	\textbf{Part a: $P(\frac{1}{2} \leq X \leq \frac{3}{2})$ }
	$P(\frac{1}{2} \leq X \leq \frac{3}{2}) = \int_{\frac{1}{2}}^{\frac{3}{2}}f(x)dx = \int_{\frac{1}{2}}^{\frac{3}{2}}\frac{1}{2}dx = \frac{1}{2} $\\
	
	\textbf{Part b: $P(1 \leq X \leq 2)$}  
	$P(1 \leq X \leq 2) = \int_{1}{2}\frac{1}{2}dx = \frac{1}{2}$\\
	
	\textbf{Part c: $P(Y \leq X)$}
	$P(Y \leq X) = P(X^2 \leq X) = P(X^2-X \leq 0) = P(X^2-X \leq 0) = P(X(X-1) \leq 0 ) $
	Since $X(X-1) \leq 0 \implies 0 \leq X \leq 1$\\
	
	Thus, 	$P(Y \leq X) = P(0 \leq X \leq 1 ) = \int_{0}^{1}\frac{1}{2} dx = \frac{1}{2} $\\
	
	\textbf{Part d: $P(Y \leq 2X)$}
	$P(Y \leq 2X) = p(X \leq 2X^2) = P(X(1-2X) \leq 0) = P(X(2X-1) \geq 0) = P(X \geq \frac{1}{2}) \cup P(X \leq 0 ) = P(X \geq \frac{1}{2}) + 0 = \int_{\frac{1}{2}}^{2} \frac{1}{2} dx + \int_{2}^{\infty} 0 dx = \frac{3}{4}$
	
	\textbf{Part e: $P(X+X^2 \leq \frac{3}{4})$}
	$P(X+X^2 \leq \frac{3}{4}) = P((X+0.5)^2 \leq 1) = P(X \leq 0.5) = \int_{0.5}^{1}\frac{1}{2} = \frac{1}{4}$
	
	\textbf{Part f: $P(\sqrt X \leq z)$}
	 $P(\sqrt X \leq z) = P(X\leq z^2)$ = $\frac{1}{2}z^2\ if 0 \leq z \leq \sqrt{(2)}$ $0$ otherwise
	
	}
	\end{homeworkSection}
	
	\begin{homeworkSection}{(5)}
	\problemAnswer{
		\vspace*{200px}

		\begin{math}
			F(x) = \begin{cases}
			0 & if x < -1 \\
			1-p & if -1 \leq x < 0, \\
			1-p+\frac{1}{2}xp & if  0 \leq x < 2\\
			1 & x \geq 2\\
			\end{cases}
		\end{math}
		
		Thus  $f(x) = F'(x)$ : \\
		\begin{math}
				f(x) = \begin{cases}
				0 & if x < -1 \\
						0 & if -1 \leq x < 0, \\
						\frac{1}{2}p & if  0 \leq x < 2\\
						0 & x \geq 2\\
				\end{cases}
			\end{math}		

		\textbf{Part a} $P(X=-1)$
		$P(X=-1) = 1-p$
		
		\textbf{Part b} $P(X=0)$ \\
		$P(X=0)$ = $f(x=0)= 0$ \\ 
		
		\textbf{Part c}$P(X \geq 1)$ \\
		$P(X \geq 1)$ = $\int_{1}^{2}(\frac{1}{2}p)dx + \int_{1}0dx = \frac{1}{2}p$
	}
	
	\end{homeworkSection}
	
	\begin{homeworkSection}{(7)}
	\problemAnswer{
	p(Teeny Weeny is overbooked) = p(All 10 passengers turn up) = $(\frac{9}{10})^10 = 0.34$
	
	p(Blockbuster airways is overbooked) = p(19 or 20 passengers turn up) = 
	$\binom{20}{19}(\frac{1}{10})(\frac{9}{10})^19 + \binom{20}{2}(\frac{1}{10})^2(\frac{9}{10})^18 = .39$
	
	Thus blockbuster ariways is overbooked on average.
	}
	\end{homeworkSection}
	
	\begin{homeworkSection}{(9)}
	\problemAnswer{
		\textbf{Part a}\\
	$X^+ = max(0, X)$: \\
	As $F(x)$ is positive definite: \\
	\begin{math}
		P(X^+ \leq x) = \begin{cases}
		
		0 & x \leq 0 \\
		F(x) & x\geq 0
		
		\end{cases}
	\end{math}
		\textbf{Part b}\\
	For $X^- = -min(0, X)$: \\
	$P(X^- \leq x) = P((0 \leq x) \cup (-X \leq 0)) = P((0 \leq x) \cup (X \geq -x)) = P(0 \leq x) + P(X \geq -x) = P(0 \leq x) + (1-P(X \leq -x)) $
	
	Thus,
	\begin{math}
		P(X^- \leq x) = \begin{cases}
		0 & x \leq 0\\
		1-F(-x) & x \geq 0\\
		\end{cases}
	\end{math}
	
	\textbf{Part c}\\
	
	$|X| = X^+ + X^-$
	
	%Thus $P(|X| \leq x) = P((X^+ \leq x) \cup P(X^- \leq -x)) = P(X^+ \leq x) + P(X^- \leq -x) - P((X^+ \leq x) \cap P(X^- \leq -x))$
	Thus, $P(|X| \leq x) = P(-x \leq X \leq x) = F(x) - F(-x)$
	
	\textbf{Part d}

	$-X$
	
	$P(-X \leq x) = P(-X \geq x) = 1-P(X \leq -x) = 1-F(-x)$
	  
	
	}
	\end{homeworkSection}
	
	\begin{homeworkSection}{11}
	\problemAnswer{
	\textbf{Given:} Median $m$ = $\lim_{y \rightarrow \infty} F(y) \leq \frac{1}{2} \leq F(m)$
	\textbf{To Prove:} Every distribution of F has a median and it is a closed interval
	
	Since $F$ is a continuous function with range $[0,1]$ applying intermediate values theorem: \\
	$F$ will take all values between its extremum.
 so every distribution of F has a median and is closed under $[0,1] $
 
 
	}
	\end{homeworkSection}
	
	
	\begin{homeworkSection}{(12)}
	
	\problemAnswer{
	Consider the outcomes of the dice to be $X1, X2$ and the sum to be $S=X1+X2$.
	Let us assume the outcomes that $S=2$ to $S=12$ are equally likely
	Then let: \\
	$ S=2: A=\{(1,1)\}\\
	  S=9: A=\{(3,6), (4,5), (5,4), (6,3)\}
	$
	
	Thus for $S=2,9$ , $A=1,4$ the number of favourable events are unequal $\implies$ the initial assumption is false!
	
	}
	\end{homeworkSection}
	
	\begin{homeworkSection}{(15)}
	\problemAnswer{

	}
	\end{homeworkSection}
	
	
	\begin{homeworkSection}{(18)}
	\problemAnswer{
	Total ways to arrange 8 pawns on a chessboard = $\binom{64}{8} $
	
	Number of ways of being in straight line = Number of ways they are in same row + Number of ways they are in same column + Number of ways they are in same diagonal
	= $\binom{8}{1} + \binom{8}{1} + \binom{2}{1}$
	p(pawns are in straight line) 	= $\frac{\binom{8}{1}+ \binom{8}{1} + \binom{2}{1}}{\binom{64}{8} } = \frac{18}{\binom{64}{8}}$
	
	Number of ways that pawns are not in the same row or column =
	First pawn has 8 rows/columns to choose from, then the second pawn will have 7 rows/columns to choose from and so on.
	At each step the row taken by the pawn at earlier step is removed from choice of consideration
	
	Thus 	Number of ways that pawns are not in the same row or column = $8*7*6...=8!$
	
	Thus p(	Number of ways that pawns are not in the same row or column ) = $\frac{8!}{\binom{64}{8}}$
	
	
	}
	\end{homeworkSection}
	\begin{homeworkSection}{(19)}
		\problemAnswer{
			\textbf{Part a} 
			
			\begin{math}
				F(x) = \begin{cases}
				1- e^{-x^2} & x \geq 0 \\
				0 & otherwise \\
				\end{cases}
			\end{math}
			$F(\infty) = 1$ and $F(-\infty) = 0$
			
			Differentiating $F(x)$ we get: \\
			\begin{math}
				f(x) = \begin{cases}
					2xe^{-x^2} & x \geq 0 \\
					0 & otherwise \\
				\end{cases}
			\end{math}
				F is a distribution function\\
			
			\textbf{Part b} 
			
			\begin{math}
				F(x) = \begin{cases}
				e^{-\frac{1}{x}} & x > 0 \\
				0 & otherwise \\
				\end{cases}
			\end{math}
			$F(\infty) = 1$ and $F(-\infty) = 0$
			
			Differentiating $F(x)$ we get: \\
			\begin{math}
				f(x) = \begin{cases}
				   \frac{e^{-\frac{1}{x}}}{x^2} & x > 0 \\
					0 & otherwise \\
				\end{cases}
			\end{math}	
			F is a distribution function\\
			
			\textbf{Part c} 
			
			\begin{math}
				F(x) = \begin{cases}
			\frac{e^x}{e^x+ e^{-x}} & x \geq 0 \\
				0 & otherwise \\
				\end{cases}
			\end{math}
			$F(\infty) = 1$ and $F(-\infty) = 0$
			
			Differentiating $F(x)$ we get: \\
			\begin{math}
				f(x) = \begin{cases}
					\frac{2}{(e^x+e^-x)^2}& x \geq 0 \\
					0 & otherwise \\
				\end{cases}
			\end{math}
				F is a distribution function\\
			\textbf{Part d} 
			
			\begin{math}
				F(x) = \begin{cases}
			e^{-x^2} + \frac{e^x}{e^x+ e^{-x}} & x \geq 0 \\
				0 & otherwise \\
				\end{cases}
			\end{math}
			$F(\infty) = 1$ and $F(-\infty) = 0$
		
			
			Differentiating $F(x)$ we get: \\
			\begin{math}
				f(x) = \begin{cases}
					-2xe^{-x^2} + \frac{2}{(e^x+e^-x)^2}  & x \geq 0 \\
					0 & otherwise \\
				\end{cases}
			\end{math}
			F is NOT a distribution function as $f$ can take in negative values.
		}
	\end{homeworkSection}
	
	
\end{homeworkProblem}

\end{document}
