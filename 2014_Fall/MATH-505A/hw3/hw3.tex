%%%%%%%%%%%%%%%%%%%%%%%%%%%%%%%%%%%%%%%%%
% Structured General Purpose Assignment
% LaTeX Template
%
% This template has been downloaded from:
% http://www.latextemplates.com
%
% Original author:
% Ted Pavlic (http://www.tedpavlic.com)
%
% Note:
% The \lipsum[#] commands throughout this template generate dummy text
% to fill the template out. These commands should all be removed when 
% writing assignment content.
%
%%%%%%%%%%%%%%%%%%%%%%%%%%%%%%%%%%%%%%%%%

%----------------------------------------------------------------------------------------
%	PACKAGES AND OTHER DOCUMENT CONFIGURATIONS
%----------------------------------------------------------------------------------------

\documentclass{article}

\usepackage{fancyhdr} % Required for custom headers
\usepackage{lastpage} % Required to determine the last page for the footer
\usepackage{extramarks} % Required for headers and footers
\usepackage{graphicx} % Required to insert images
\usepackage{lipsum} % Used for inserting dummy 'Lorem ipsum' text into the template

\usepackage{amsmath}
%\usepackage{multline}

% Margins
\topmargin=-0.45in
\evensidemargin=0in
\oddsidemargin=0in
\textwidth=6.5in
\textheight=9.0in
\headsep=0.25in 

\linespread{1.1} % Line spacing

% Set up the header and footer
\pagestyle{fancy}
\lhead{\hmwkAuthorName} % Top left header
\chead{\hmwkClass\ : \hmwkTitle} % Top center header
\rhead{\firstxmark} % Top right header
\lfoot{\lastxmark} % Bottom left footer
\cfoot{} % Bottom center footer
\rfoot{Page\ \thepage\ of\ \pageref{LastPage}} % Bottom right footer
\renewcommand\headrulewidth{0.4pt} % Size of the header rule
\renewcommand\footrulewidth{0.4pt} % Size of the footer rule

\setlength\parindent{0pt} % Removes all indentation from paragraphs

%----------------------------------------------------------------------------------------
%	DOCUMENT STRUCTURE COMMANDS
%	Skip this unless you know what you're doing
%----------------------------------------------------------------------------------------

% Header and footer for when a page split occurs within a problem environment
\newcommand{\enterProblemHeader}[1]{
\nobreak\extramarks{#1}{#1 continued on next page\ldots}\nobreak
\nobreak\extramarks{#1 (continued)}{#1 continued on next page\ldots}\nobreak
}

% Header and footer for when a page split occurs between problem environments
\newcommand{\exitProblemHeader}[1]{
\nobreak\extramarks{#1 (continued)}{#1 continued on next page\ldots}\nobreak
\nobreak\extramarks{#1}{}\nobreak
}

\setcounter{secnumdepth}{0} % Removes default section numbers
\newcounter{homeworkProblemCounter} % Creates a counter to keep track of the number of problems

\newcommand{\homeworkProblemName}{}
\newenvironment{homeworkProblem}[1][Problem \arabic{homeworkProblemCounter}]{ % Makes a new environment called homeworkProblem which takes 1 argument (custom name) but the default is "Problem #"
\stepcounter{homeworkProblemCounter} % Increase counter for number of problems
\renewcommand{\homeworkProblemName}{#1} % Assign \homeworkProblemName the name of the problem
\section{\homeworkProblemName} % Make a section in the document with the custom problem count
\enterProblemHeader{\homeworkProblemName} % Header and footer within the environment
}{
\exitProblemHeader{\homeworkProblemName} % Header and footer after the environment
}

\newcommand{\problemAnswer}[1]{ % Defines the problem answer command with the content as the only argument
\noindent\framebox[\columnwidth][c]{\begin{minipage}{0.98\columnwidth}#1\end{minipage}} % Makes the box around the problem answer and puts the content inside
}

\newcommand{\homeworkSectionName}{}
\newenvironment{homeworkSection}[1]{ % New environment for sections within homework problems, takes 1 argument - the name of the section
\renewcommand{\homeworkSectionName}{#1} % Assign \homeworkSectionName to the name of the section from the environment argument
\subsection{\homeworkSectionName} % Make a subsection with the custom name of the subsection
\enterProblemHeader{\homeworkProblemName\ [\homeworkSectionName]} % Header and footer within the environment
}{
\enterProblemHeader{\homeworkProblemName} % Header and footer after the environment
}
   
%----------------------------------------------------------------------------------------
%	NAME AND CLASS SECTION
%----------------------------------------------------------------------------------------

\newcommand{\hmwkTitle}{Homework\ \# 3 } % Assignment title
\newcommand{\hmwkDueDate}{Friday,\ September \ 12,\ 2014} % Due date
\newcommand{\hmwkClass}{MATH-505A} % Course/class
\newcommand{\hmwkClassTime}{10:30am} % Class/lecture time
\newcommand{\hmwkAuthorName}{Saket Choudhary} % Your name
\newcommand{\hmwkAuthorID}{2170058637} % Teacher/lecturer
%----------------------------------------------------------------------------------------
%	TITLE PAGE
%----------------------------------------------------------------------------------------

\title{
\vspace{2in}
\textmd{\textbf{\hmwkClass:\ \hmwkTitle}}\\
\normalsize\vspace{0.1in}\small{Due\ on\ \hmwkDueDate}\\
%\vspace{0.1in}\large{\textit{\hmwkClassTime}}
\vspace{3in}
}

\author{\textbf{\hmwkAuthorName} \\
	\textbf{\hmwkAuthorID}
	}
\date{} % Insert date here if you want it to appear below your name

%----------------------------------------------------------------------------------------

\begin{document}

\maketitle

%----------------------------------------------------------------------------------------
%	TABLE OF CONTENTS
%----------------------------------------------------------------------------------------

%\setcounter{tocdepth}{1} % Uncomment this line if you don't want subsections listed in the ToC

\newpage
\tableofcontents
\newpage

\begin{homeworkProblem}[Exercise \# 1.7] % Custom section title
	\begin{homeworkSection}{(1)} % Section within problem
		\problemAnswer{ % Answer
			\textbf{Given:} Two roads $r1_{AB}$, $r2_{AB}$ connectinng points A and B and $s1_{BC}$, $s2_{BC}$ connecting B and C. 
			%$p(X)$ denotes the probability that the road X is blocked. Then, $p(r1_{AB}) = p(r2_{AB}) = p(s1_{AB}) = p(s2_{AB}) = p$.
			
			Let $p(AB)$ denote the probability that path between A$\longrightarrow$B is open and let $p({AB}^c)$ denote the probability that there is(are)  no path open  b/w A and B.
			%Alternatively $p(AB)$ denotes that road(s) between A and B are open.
			
			\textbf{To find:}$Y=P(AB | AC^c)$.
			
		Y is equal to the probability that path between A and B is open AND(given that) the path between A and C is closed $\implies$ Path between B and C is closed AND between A and B is open
		
		p(AB) $=$ Path b/w A,B is open $=$ 1 - Path b/w A,B is closed $=$  $1 - p*p$
		
		$p(AC^C|AB)$ = Path b/w A,C is closed given A,B is open = Path b/w B,C is closed given A,B is open. =  $p(BC^C)$\\
		Thus \begin{equation}
		\label{1c1}
		p(AB) = 1 - p^2
		\end{equation}
		Also, 
		\begin{equation}
		\label{1c2}
			p(AB) = p(BC)
		\end{equation}
		
		
		
		$p(AC^C) = $ 1 - Probability A,C is open = 1 - Probability AB is open AND BC is open. % = $1-p(AB)*p(BC)$
		Thus,
			\begin{equation}
			\label{1c3}
			p(AC^c) = 1- p(AB)p(AC) = 1-(1-p^2)^2
			\end{equation}
			
			
		\begin{equation}
		\label{1c4}
		p(AB \cap AC^C) = p(AC^C | AB)p(AB) = p(BC^C)p(AB) = p^2(1-p^2)
		\end{equation}
			
		\begin{equation}
		\label{1c5}
		p(AB|AC^c) = \frac{P(AB \cap AC^c)}{p(AC^c)} = \frac{p(AC^C | AB)p(AB)}{p(AC^c)} = \frac{p^2(1-p^2)}{1-(1-p^2)^2} 
		\end{equation}
		
		\textbf{Part 2:} Additional direct road from A to C. Find $p(AB | AC^c)$: \\
		$p(AC^c|AB) = $ Probability that path b/w A,C is closed given path b/w A,B are open = Probability path b/w A,C(direct) are closed AND path b/w B,C are closed
		
		\begin{equation}
		\label{1c6}
		p(AC^C|AB) = p*p(BC^c)p(AB)
		\end{equation}
		
		where the extra p in \ref{1c6} as compared to \ref{1c4} is because the direct path A,C should be blocked too.
		
		\begin{equation}
		\label{1c7}
				p(AC^c) =  1 - (p(1-p^2)^2 - (1-p))
		\end{equation}
		
		where the second term is the probability that direct A $\longrightarrow$ C is blocked, but A $\longleftrightarrow$ B and $B \longrightarrow$ is open(both). The third term represents the probaility that A $\longrightarrow$ C is open directly(does not matter if the other paths are open or blocked), 
		%where the extra $(1-p)$ factor in \ref{1c7} as compared to \ref{1c3} accounts for the fact that direct path AC is open.
		
		Thus, for part 2:
		
		\begin{equation}
		\label{1c8}
		p(AB|AC^c) = \frac{p^3(1-p^2)}{1-(1-p^2)^2-(1-p)} = \frac{p^2(1-p^2)}{1-(1-p^2)^2} 
		\end{equation}

		
		 
		}
	\end{homeworkSection}
		\begin{homeworkSection}{(2)} % Section within problem
			\problemAnswer{ % Answer
				\begin{equation}
					p(2K \cap 1A ) = \frac{\binom{4}{2} * \binom{4}{1} * \binom{52-4-4}{10}}{\binom{52}{13}}
					%= \frac{24 * 44! * 13!}{10! * 52!} = 1.357*10^{-9}
			\end{equation}
				
				$p(1A | 2K) = \frac{p(1A \cap 2K)}{p(2K)}$
				
				\begin{equation}
					\label{2c2}
						p(2K ) = \frac{\binom{4}{2}*\binom{52-4}{11}}{\binom{52}{13}} = \frac{6 * 48! * 13!}{52! * 11!} = \frac{6 * 12 * 13}{49 * 50 * 51 * 52} = 2.52*10^-4
				\end{equation}
				
				Thus,
				
				$p(1A | 2K ) = \frac{ \binom{4}{2} * \binom{4}{1} * \binom{44}{10}}{\binom{4}{2} * \binom{48}{11} }= \frac{4*11*37!}{46*47*48*34!} = 0.0399$

				
			}
		\end{homeworkSection}	
		
			\begin{homeworkSection}{(4)} % Section within problem
				\problemAnswer{ % Answer
					
					\textbf{To prove/disprove: } $p(x|C) > p(y|C) AND p(x|C^c) > p(y|C^c) \implies p(x) > p(y) $
					
					\begin{equation}
						\label{4ca1}
						p(x|C) - p(y|C)  > 0
					\end{equation}
					
					\begin{equation}
						\label{4ca1}
						p(x|C^c) - p(y|C^c) > 0
					\end{equation}
					
					\begin{equation}
					\label{4c1}
					p(x) = p(x|C)p(C) + p(x|C^c)p(C^c)
					\end{equation}
					
					Also,
					
					\begin{equation}
					\label{4c2}
					p(y) = p(y|C)p(C) + p(y|C^c)p(C^c)
					\end{equation}
					
					Consider $p(x)-p(y)$ : \\
					
					\begin{equation}
					\label{4c3}
					p(x)-p(y) = (p(x|C)-p(y|C))p(C) + (p(x|C^c)-p(y|C^c))p(C^c)
					\end{equation}
					
					From the  \ref{4ca1}, \ref{4ca2} and \ref{4c3}:
					
					\begin{equation}
						p(x) - p(y) > 0 \forall x,y
					\end{equation}
					
					\center {\textbf{Thus, x is always prefered over y.}}
					 
				}
			\end{homeworkSection}	
			
			\begin{homeworkSection}{(5)} % Section within problem
				\problemAnswer{ % Answer
					
					Let $X_i$ represent the $i^{th}$ ccard draw
					
					\textbf{Given:} $X_k > X_i, \  \forall i \in [1,k-1] and k \in [1,m]$
					
					$p(X_k = m | X_k > X_i) = \frac{p(X_k = m \cap X_k > X_i)}{p(X_k > X_i)} = \frac{p(X_k=m)}{p(X_k > X_i)} = \frac{\frac{1}{m}}{\frac{1}{k}} $
					
					Where the equality in the last step comes from the fact that the probability of choosing cards such that $p(X_k > X_i)$ is simply to choose the largest card, i.e. $k$ among the rest $i$.
					
					
					
					Thus $p(X_k=m | X-k > X_i) = \frac{k}{m}$.
					
				}
			\end{homeworkSection}				
\end{homeworkProblem}

\begin{homeworkSection}[Exercise \# 1.8] % Custom section title
	\begin{homeworkProblem}[1] % Section within problem
		\begin{homeworkSection}{(a)} % Section within problem
			\problemAnswer{ % Answer
				\textbf{Six turns up exactly once}
				The dice on which 6 should appear is choosen in $\binom{2}{1}$ way and has just one option(6) while the other has $6-1=5$  options. Total outcomes = $6*6=36$
				
				Thus: \\
				$ p = \frac{1*5}{36} = \frac{5}{36} $
			}
		\end{homeworkSection}	
		\begin{homeworkSection}{(b)} % Section within problem
			\problemAnswer{ % Answer
				
				\textbf{Both numbers are odd.}
				3 out of 6 numbers are odd and the outcome of odd on both dice are independent. Thus, \\
				
				
				$p = \frac{3*3}{36} = \frac{1}{12} $
				
			}
		\end{homeworkSection}	
		\begin{homeworkSection}{(c)} % Section within problem
			\problemAnswer{ % Answer
				\textbf{Sum of scores is 4}.
				
				Possible configurations: $(1,3);(2,2);(3,1)$
				
				Thus:\\
				$p=\frac{3}{36} = \frac{1}{12}$
			}
		\end{homeworkSection}	
		\begin{homeworkSection}{(d)} % Section within problem
			\problemAnswer{ % Answer
				\textbf{Sum of scores is divisible by 3}.
				
				Possible choices for sum to be divisible by 3: \\
				$3,6,9,12$: 
				
				$3$: $(1,2),(2,1)$\\
				$6$: $(1,5),(2,4),(3,3),(4,2),(5,1)$\\
				$9$: $(3,6),(4,5),(5,4),(6,3)$ \\
				$12$: $(6,6)$
				
				Thus $p = \frac{2+5+4+1}{36} = \frac{1}{3}$
				
			}
		\end{homeworkSection}			
	\end{homeworkProblem}	

	\begin{homeworkProblem}[2] % Section within problem
		\begin{homeworkSection}{(a)} % Section within problem
			\problemAnswer{ % Answer
				Head appears the first time on $n^{th}$ throw.
				Thus a series of $n-1$ consecutive tails followed by head.
				
				$p= \binom{n}{1}\frac{1}{2^{n-1}}*\frac{1}{2} = \frac{n}{2^n}$ 
			}
		\end{homeworkSection}			
		\begin{homeworkSection}{(b)} % Section within problem
			\problemAnswer{ % Answer
				The number of heads and dates to date are equal:
				
				\textbf{Case1:} $n$ is odd.
				
				Clearly the probability of heads and tails being equal in this case is zero!
				
			    \textbf{Case2:} $n$ is even.
				
				A series of H,T such that $|H| = |T|$. Since the probability of occurence of either a H or a T is $\frac{1}{2}$, the result will still be: \\
				$ p = \binom{n}{n/2}\frac{1}{2^n}$
			}
		\end{homeworkSection}
		\begin{homeworkSection}{(c)} % Section within problem
			\problemAnswer{ % Answer
				Exactly two heads have appeared altogether.
				
			
				
				$p = \binom{n}{2} * \frac{1}{2^2} * \frac{1}{2^{n-2}} = \binom{n}{2} * \frac{1}{2^n}$
}
		\end{homeworkSection}
		\begin{homeworkSection}{(d)} % Section within problem
			\problemAnswer{ % Answer
				At least two heads have appeared.
				
				This is same as the one minus probability that 0 or 1 heads have appeared so far:
				
				$p = 1 - p(0\ heads) - p(1\ head\ only)$
				$p = 1 - \binom{n}{0}\frac{1}{2^0}\frac{1}{2^n} - \binom{n}{1}\frac{1}{2}\frac{1}{2^n-1} = 1 - \frac{1}{2^n} - \frac{n}{2^n} $
				
				
			}
		\end{homeworkSection}
	\end{homeworkProblem}			
	
	
	\begin{homeworkProblem}[4] % Section within problem
		\begin{homeworkSection}{(a)} 
			\problemAnswer{
			\textbf{Biased coin tossed three times:} \\
			{Sample space}: HHH,HHT,HTH,HTT,THH,THT,TTH,TTT
		}
		\end{homeworkSection}
		\begin{homeworkSection}{(b)} 
			\problemAnswer{
			\textbf{Balls drawn without replacement from 2U,2V balls} \\
			\textbf{Sample space}: $(U1,V1),(U2,V1),(V1,U1), (V1, U2), (U1,V2), (U2,V2), (V2,U1), (V2,U2)$
		}
		\end{homeworkSection}
		\begin{homeworkSection}{(c)}
			\problemAnswer{
\textbf{			Biased coin tossed till H tunrs up} \\
Sample Space: \\
 		$\{H\}, \{T,H\}, \{T,T,H\}, \{T,T,T,H\}.....\{T,T,T,....,T,H\}$
 	}
		\end{homeworkSection}
		
	\end{homeworkProblem}
	\newpage
	\begin{homeworkSection}{5}
		\problemAnswer{
			\textbf{To find:} Probablity that exactly one of A,B occurs\\
			
			Probability that exactly one of A,B occurs is equal to the probability that either \textbf{only A} occurs or \textbf{only B} occurs. This is simply given by:
			
			$P(A \cup B) - P(A \cap B)$. That is either of A,B occurs removing the portion when both A,B occur.
		
			Thus, the probability that exactly one of A,B occurs is : \\
			$ p(A \cup b )-p(A\cap B) = p(A) + p(B) - p(A \cap B) - p(A \cap B) = p(A) + p(B) - 2p(A \cap B)$
		}
	\end{homeworkSection}
	
	\begin{homeworkSection}{6}
		\problemAnswer{
			\textbf{To Prove:} $P(A\cup B \cup C) = 1 - P(A^C | B^C \cap C^C) p(B^C | C^C) p(C^C)$
			$
			RHS = 1 - P(A^C | B^C \cap C^C)P(B^C | C^C) p(C^C)
			$
			
			$ RHS = 1 - P(A^C | B^C \cap C^C)\frac{P(B^C \cap C^C)}{p(C^C)}p(C^C) $
			
			Thus, expanding further $p(A^C | B^C \cap C^C) = \frac{p(A^C \cap B^C \cap C^C)}{p(B^C \cap C^C)}$
				
			Thus,
			
			$ RHS = 1 - \frac{p(A^C \cap B^C \cap C^C)}{p(B^C \cap C^C)} * \frac{P(B^C \cap C^C )}{p(C^C)}*p(C^C) 
			$
			
			$ RHS = 1 - P(A^C \cap B^C \cap C^C) $ which is same as $LHS$ 
			}
	\end{homeworkSection}
			
			\begin{homeworkSection}{14}
				\problemAnswer{
					
				Consider only two events: A,$B_j$. 
				
				From the definition of conditional probability: \\
				\begin{equation}
				\label{14c1}
				 p(A_j|B) = \frac{p(A_j \cap B)}{p(B)}
				\end{equation}
				 
				Also: \\
				\begin{equation}
				\label{14c2}
				p(B|A_j) = \frac{p(A_j \cap B)}{p(A_j)}
				\end{equation}
				
				 Substituing for $p( A_j\cap B)$ from \ref{14c1} in \ref{14c2}: \\
				 
				 \begin{equation}
				 \label{14c3}
				 p(B|A_j) = \frac{p(A_j|B)p(B)}{p(A_j)}
				 \end{equation}
				
				or equivalently,
				
				 \begin{equation}
				 \label{14c4}
				 p(A_j|B) = \frac{p(B|A_j)p(A_j)}{p(B)}
				 \end{equation}
				 
				 where p(B) can be expressed as(using the law of partition): \\
				 \begin{equation}
				 \label{14c5}
				 p(B) = p(B|A_1)P(A_1) + P(B|A_2)P(A_2) + ...+P(B | A_n)P(A_n)
				 \end{equation}
				 
				 Thus $p(B) = \sum_{1}^{n}p(B|A_i)p(A_i)$
				
				Ans hence,
				
				\center $p(A_j|B) = \frac{p(B|A_j)p(A_j)}{\sum_{1}^{n}p(B|A_i)p(A_i)}$
					}
			\end{homeworkSection}
\end{homeworkSection}


\end{document}
